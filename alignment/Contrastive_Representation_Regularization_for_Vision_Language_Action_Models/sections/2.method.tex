\section{Method}
In this section, we introduce \textit{Robot State-aware Contrastive Loss (RS-CL)}, which enhances the action prediction capability of VLA models by guiding the representation to capture low-level robotic signals, particularly the proprioceptive states.
We describe the VLA training framework in Sec.~\ref{subsec:vla} and present our proposed method, RS-CL, in Sec.~\ref{subsec:RS-CL}. An overview of our method is shown in Fig.~\ref{figure:method}.

\subsection{Vision-Language-Action model}
\label{subsec:vla}

VLA models are trained to predict the next action chunk $\mathbf{A}_t =[\rva_t, \rva_{t+1}, \dots, \rva_{t+H}]$ of horizon $H$ at current timestep $t$, from a set of observation images from $V$ different views $\mathbf{O}_t^V$ = $\{\rvo_t^1, \rvo_t^2, \dots, \rvo_t^V\}$, a task instruction $\rvc$, and the robot's proprioceptive state $\rvq$.
A standard framework for VLA models~\citep{pi0, gr00tn1} encodes multimodal inputs $[\mathbf{O}_t^V, \rvc]$ using a pre-trained VLM into a hidden representation, and pass it to the action decoder.
In practice, we train a lightweight adapter module $f_{\phi}$ upon the VLM and freeze the VLM, following \citet{gr00tn1_5}.
$f_{\phi}$ processes the output of the VLM as $\rvh = f_{\phi}\big(\textrm{VLM}(\mathbf{O}_t^V, \rvc)\big) \in \mathbb{R}^{N \times d_{\textrm{model}}} $, where $N$ is the number of input tokens for the VLM and $d_{\textrm{model}}$ is the size of the hidden dimension. 

An action decoder $D_\theta$ generates $\mathbf{A}_t$ conditioned on $\rvh$ with the current robot state $\rvq$. 
Similar to prior works~\citep{pi0, gr00tn1}, we adopt the DiT ~\citep{diffusiontransformer} architecture for the $D_\theta$ and train with the flow-matching objective~\citep{flowmatching, rectifiedflow}:
\begin{equation}
\mathcal{L}_{\textrm{FM}}(\theta, \phi) = \mathbb{E}_{s} \Big[ \| D_\theta(\rvh, \mathbf{A}_t^{s}, \rvq) - (\epsilon - \mathbf{A}_t) \|_2^2 \Big],
\label{eq:fm}
\end{equation}
where $\mathbf{A}_{t}^{s} = s \mathbf{A}_{t} + (1-s)\epsilon$ is an interpolated action chunk at the flow-matching timestep $s \in [0,1]$ sampled from a prior distribution $p(s)$. 
After training, $D_\theta$ generates $\mathbf{A}_t$ through an iterative denoising process starting from a random Gaussian noise $\epsilon \sim \mathcal{N}(\bm{0}, \mathbf{I})$. 

\begin{figure}[t]
    \centering\small
    \begin{subfigure}[t]{0.30\textwidth}
        \includegraphics[width=\linewidth]{assets/observation_trajectories.pdf}
        \caption{Task trajectories.}
        \label{subfig:trajectories}
    \end{subfigure}
    ~
    \begin{subfigure}[t]{0.33\textwidth}
        \includegraphics[width=\linewidth]{assets/observation_tsne_vlm.pdf}
        \caption{Pre-trained VLM representations.}
        \label{subfig:vlmrep}
    \end{subfigure}
    \begin{subfigure}[t]{0.33\textwidth}
        \includegraphics[width=\linewidth]{assets/observation_tsne_rscl.pdf}        
        \caption{RS-CL aligned representations.}
        \label{subfig:rsclrep}
    \end{subfigure}
    \caption{\textbf{Training VLM representations for action prediction.} 
    \textbf{(a)} We visualize VLM embeddings of robot episodes performing the same task ``Open the microwave / cabinet door'' across different scenes in  RoboCasa-Kitchen. 
    \textbf{(b)} Pre-trained VLM representations are dominated by the visual appearance (\emph{e.g.}, distractor objects).
    \textbf{(c)} RS-CL guides embeddings to align with the robot’s proprioceptive states, yielding representations that capture common robotic signals (\emph{e.g.}, the robot's current pose, next control action) across environments, therefore aligning all episodes by the task progress.
    }
\vspace{-1.0em}
\label{figure:motivation}
\end{figure}

\subsection{Robot State-aware Contrastive Loss}
\label{subsec:RS-CL}
While VLMs acquire rich semantic representations from Internet-scale vision–language data, they lack exposure to robotic modalities such as low-level control actions and proprioceptive states.
As a result, their embeddings are strongly shaped by the visual appearance and often fail to capture signals relevant to robot control.
This misalignment is evident when we visualize the VLM embeddings of robot trajectories for the same manipulation task (\emph{e.g.}, Open the microwave / cabinet) across different environments in RoboCasa-Kitchen (see Fig.~\ref{subfig:trajectories}).
We observe that VLM embeddings are dominated by the visual cues, such as presence of large objects or background textures (see Fig.~\ref{subfig:vlmrep}), rather than control-relevant factors like the robot’s current pose or the next action needed to complete the task.

\begin{wrapfigure}[17]{R}{0.40\textwidth}
\centering\small
\resizebox{0.38\textwidth}{!}{
\includegraphics[width=\linewidth]{assets/method_viewcutoff.pdf}
}
\vspace{-0.5em}
\captionof{figure}{
    \textbf{Representation-level augmentation for contrastive pairs.}
    \mbox{\textit{View cutoff}} is an simple augmentation that randomly masks out the embedding slice of one observation view from the VLM representation.
}
\label{fig:viewcutoff}
\end{wrapfigure}

This misalignment motivates our central hypothesis: explicitly aligning VLM representations with their physical state will improve action prediction.
Based on this hypothesis, we introduce \textit{Robot State-aware Contrastive Loss (RS-CL)},
an auxiliary objective for VLAs that regularizes the VLM's representation space using supervision from the robot's proprioceptive states.
Our key idea is a contrastive loss that uses the distances between proprioceptive states to assign soft weights to similarity scores, which effectively guides the representation space to be aligned with robotic signals.
As an auxiliary objective, RS-CL complements the original action prediction loss, enabling the entire model to be trained end-to-end in a single stage.
Concretely, RS-CL consists of three key components: a \textit{learnable summarization token} that amortizes long VLM outputs, a \textit{weighting scheme} for robot state supervision, and a \textit{representation-level augmentation} strategy for lightweight representation learning.


\vspace{0.01in}
\textbf{Amortizing VLM embeddings for representation learning.}~
Applying contrastive learning on the full sequence of VLM embeddings $\rvh \in \mathbb{R}^{N \times d_{\textrm{model}}}$ is impractical as the sequence length $N$ is typically large, leading to high computational cost and diluted learning signals. 
To address this, we introduce a \textit{learnable summarization token} $\rvu \in \mathbb{R}^{1 \times d_{\textrm{model}}}$ to produce a compact representative embedding of the sequence. 
Specifically, $\rvu$ is appended to the VLM output and processed by the adapter $f_\phi$:
\begin{equation}
[\rvh, \rvw] = f_\phi \big(\textrm{VLM}(\mathbf{O}_t^V, \rvc) \oplus \rvu \big),
\end{equation}
where $\rvw$ denotes the output corresponding to the summarization token and $\oplus$ denotes concatenation along the sequence dimension. 
Finally, $\rvw$ is projected by a lightweight projector $g_\psi$ into \mbox{$\rvz = g_\psi(\rvw)$}, providing a compact summary for contrastive  learning~\citep{simclr}, while the original embeddings $\rvh$ serves as the conditioning input to the action decoder.


\vspace{0.01in}
\textbf{Incorporating robot states into contrastive learning.}~
To effectively restructure the VLM representation space to capture robotic signals, we introduce a supervised contrastive learning objective assigned with \textit{soft weights}~\citep{supervisedcl, notallnegsoftcl}, that incorporate the distance between proprioceptive states. 
Conceptually, embeddings associated with similar proprioceptive states receive higher weights, are pulled closer in the representation space.
We consider InfoNCE~\citep{cpc} for the contrastive loss, which is widely used in practice~\citep{curl, r3m, vip}.
Formally, our \textit{Robot State-aware Contrastive Loss (RS-CL)} is defined as a weighted variant of the InfoNCE loss:
\begin{equation}
\mathcal{L}_{\textrm{RS-CL}}\big(\{\rvz\}_{i=1}^B, \{\tilde{\rvz}\}_{j=1}^B; \phi, \psi\big)
= - \sum_{i=1}^{B} \sum_{j=1}^{B}
w_{ij} \,
\log
\frac{\exp\!\big(\mathrm{sim}(\rvz_i, \tilde{\rvz}_j)/\tau\big)}
{\sum_{k=1}^{B} \exp\!\big(\mathrm{sim}(\rvz_i, \tilde{\rvz}_k)/\tau\big)}\text{,}
\end{equation}
where $\{\tilde{\rvz}\}_{j=1}^B$ is the augmented batch of $\{\rvz\}_{i=1}^B$ , \textrm{sim} denotes the cosine similarity, and $\tau>0$ is a temperature that controls the sharpness of similarity.
The soft weights $w_{ij}$ are computed from the relative distance between proprioceptive states $\rvq_i, \rvq_j$. 
In practice, we use the Euclidean distance and formulate $w_{ij}$ as follows:
\begin{equation}
w_{ij} = \frac{\exp(- \| \rvq_i - \rvq_j \|_2 / \beta )}{\sum_{k=1}^{B} \exp(- \| \rvq_i - \rvq_k \|_2 / \beta)},
\end{equation}
where $\beta >0$ is a temperature that controls the sharpness of the mapping from distance to weight.
The complete training objective integrates the proposed RS-CL with the action prediction objective, implemented as the flow-matching loss in \Eqref{eq:fm}:
\begin{equation}
\mathcal{L} = \mathcal{L}_{\textrm{FM}} + \lambda \, \mathcal{L}_{\textrm{RS-CL}},
\end{equation}
where we jointly optimize $\theta$, $\phi$, and $\psi$.


\vspace{0.01in}
\textbf{Representation augmentation for contrastive pairs.}~
The primary goal of our augmentation strategy is to generate diverse contrastive pairs while preserving the semantics tied to the robot’s proprioceptive states.
In line with this goal, we exploit the property that VLA models commonly process observations of the same scene from multiple views,
and propose \textit{view cutoff} (See Fig.~\ref{fig:viewcutoff}), a simple representation-level augmentation inspired by cutoff~\citep{cutoff}.  
\textit{View cutoff} randomly selects a single view index $i \in \{1, \dots, V\}$ and masks out the corresponding feature slice from the VLM output $\textrm{VLM}(\mathbf{O}_t^V, \rvc)$.
Unlike data-level augmentations requiring additional forward passes through the VLM for each augmented batch, view cutoff operates at the representation level, obtaining alternative representations with minimal overhead. 
As a result, only the lightweight adapter $f_\phi$ and projector $g_\psi$ are required to process the augmented variants, making the method substantially more efficient, yet still providing diverse pairs for contrastive learning.
