\clearpage

%------------------------------------------------------------
\section{Hyperparameters and Implementation Details}
\label{appendix:implementationdetails}
\subsection{Hyperparameters}
\label{appendixsubsec:hyperparams}
For the weighting coefficient for $\mathcal{L}_{\textrm{RS-CL}}$,  $\lambda$, we initialize to 1.0 and decayed to 0 using a cosine schedule by maximum training steps, such that representation refinement is emphasized in early training while accurate action prediction becomes the main focus later. 
For similarity temperature $\tau$ and soft weight temperature $\beta$, we use 0.2 and 1.0, respectively.

\subsection{Algorithm}
\label{appendixsubsec:pseudocode}


\begin{algorithm}[H]
\caption{Training VLA with Robot State-aware Contrastive Loss (RS-CL)}
\label{alg:rscl}
\begin{algorithmic}[1]
    \Require Observations $\mathbf{O}_t^V$, instruction $\rvc$, robot state $\rvq$, ground-truth actions $\mathbf{A}_t$, hyperparameters $(\lambda, \beta, \tau)$
    \Ensure Trained parameters $\theta, \phi, \psi$

    \For{each training step}
        \State $\rvh \gets f_{\phi}(\mathrm{VLM}(\mathbf{O}_t^V, \rvc))$ \Comment{Encode inputs with frozen VLM + adapter}
        \State $[\rvh, \rvw] \gets f_{\phi}(\mathrm{VLM}(\mathbf{O}_t^V, \rvc) \oplus \rvu)$ \Comment{Append summarization token}
        \State $\rvz \gets g_{\psi}(\rvw)$ \Comment{Project summarization output}
        \State $\tilde{\rvz} \gets \texttt{ViewCutoff}(\rvz)$ \Comment{View cutoff; Representation-level augmentation}

        \State $\mathcal{L}_{\textrm{FM}} \gets \|D_\theta(\rvh, \mathbf{A}_t^s, \rvq) - (\epsilon - \mathbf{A}_t)\|_2^2$ \Comment{Flow-matching loss}
        
        \State $w_{ij} \gets \frac{\exp(-\|\rvq_i - \rvq_j\|_2 / \beta)}{\sum_k \exp(-\|\rvq_i - \rvq_k\|_2 / \beta)}$ \Comment{Robot state-aware contrastive loss}
        \State $\mathcal{L}_{\textrm{RS-CL}} \gets -\sum_{i,j} w_{ij} \log \frac{\exp(\mathrm{sim}(\rvz_i, \tilde{\rvz}_j)/\tau)}{\sum_k \exp(\mathrm{sim}(\rvz_i, \tilde{\rvz}_k)/\tau)}$ \Comment{Contrastive loss}

        \State $\mathcal{L} \gets \mathcal{L}_{\textrm{FM}} + \lambda \mathcal{L}_{\textrm{RS-CL}}$ \Comment{Final joint objective}
        \State Update parameters $\theta, \phi, \psi$ via gradient descent
    \EndFor
\end{algorithmic}
\end{algorithm}

\subsection{Implementation Details for From-Scratch VLA Training}
\label{appendix:fromscratch}
We attach a randomly initialized action decoder to various pre-trained VLMs, with a lightweight adapter module $f_\phi$ in between. 
Following \citet{gr00tn1_5}, we define $\mathrm{VLM}(\mathbf{O}_t^V, \rvc)$ as the hidden representation from layer 12 out of 36 layers for Qwen2.5-VL-3B variants and the GR00T N1.5 backbone. 
For Qwen2.5-VL-7B, we extract $\mathrm{VLM}(\mathbf{O}_t^V, \rvc)$ from layer 18 out of 28, which yields higher performance in our layer ablation study on LIBERO (see Table~\ref{appendixtable:hiddenlayerablation}).
For SigLIP, we instead use the final hidden representation as the condition embedding. 

As the action decoder, we adopt a 16-layer DiT with 0.5B parameters. 
Empirically, we find that omitting a projection layer to reduce embedding dimensionality before conditioning improves performance (see Table~\ref{appendixtable:hiddenlayerablation}). 
Accordingly, we do not apply such a layer. 
Instead, for Qwen2.5-VL-7B variants, we use a larger attention dimension that matches its hidden size $d_\textrm{model}=3584$, while Qwen2.5-VL-3B uses $d_\textrm{model}=2048$.



\begin{table}[h]
\centering\small
\captionof{table}{
\textbf{Hidden representation layer ablations on Qwen2.5-VL-7B backbone.}~
We report success rates (\%) on the LIBERO benchmark, varying the hidden layer index used as the conditioning representation for VLA models trained from scratch.
}
\begin{tabular}{l ccccc}
    \toprule
    {Layer} 
    & {Spatial} 
    & {Object} 
    & {Goal} 
    & {Long} 
    & {Avg.} \\ 
    
    \midrule
    12 (with projection)& 87.4 & 94.2 & 41.8 & 40.4 & 66.0 \\
    18 (with projection)& 86.8 & 83.4 & 61.6 & 44.0 & 69.0 \\
    \rowcolor{black!10}
    \textbf{18 (no projection)} & 85.2 & 89.4 & 73.2 & 36.2 & \textbf{71.0} \\
    24 (with projection)& 85.2 & 89.4 & 73.2 & 36.2 & 57.0 \\
    \arrayrulecolor{black}\bottomrule
\end{tabular}
\label{appendixtable:hiddenlayerablation}
\end{table}




\clearpage
%------------------------------------------------------------
\section{Simulation Experiment Details}
\label{appendix:simulation}
\subsection{Dataset}
\label{appendix:dataset}
For RoboCasa-Kitchen, we use the publicly available dataset~\footnote{\scriptsize\url{https://huggingface.co/datasets/nvidia/PhysicalAI-Robotics-GR00T-X-Embodiment-Sim}} containing 3000 demonstrations generated with MimicGen~\citep{mimicgen}.
For LIBERO, we use the publicly available dataset~\footnote{\scriptsize\url{https://huggingface.co/datasets/physical-intelligence/libero}}, consisting of all 270K samples from LIBERO-Spatial, LIBERO-Object, LIBERO-Goal, and LIBERO-Long, re-rendered by \citet{openvla}.


\subsection{Training and Evaluation Details}
\label{appendix:trainingdetails}
For fine-tuning experiments on GR00T N1.5~\citep{gr00tn1_5}, we employ the publicly available pre-trained checkpoint~\footnote{\scriptsize\url{https://huggingface.co/nvidia/GR00T-N1.5-3B}}.
We follow the original training and inference recipe of \citet{gr00tn1_5}, including the prior distribution $p(s) = \mathcal{\textrm{Beta}}(\frac{a-s}{a};1.5,1), a = 0.999$ for sampling the flow-matching timestep $s$ in \eqref{eq:fm}. 
All models are trained with the \textit{new\_embodiment} tag. 
We omit the use of future tokens~\citep{flare}, as they are beyond the scope of this work.


For RoboCasa-Kitchen, we train for 60K gradient steps with a global batch size of 64, using AdamW with a learning rate of 1e-4 under a cosine decay schedule and 3K warmup steps. 
For LIBERO, we adopt a smaller global batch size of 32, as this setting yields better performance in practice.


For $\pi_0$ and $\pi_0$-FAST, we use the pre-trained checkpoints~\footnote{\scriptsize\url{gs://openpi-assets/checkpoints/pi0_base}} ~\footnote{\scriptsize\url{gs://openpi-assets/checkpoints/pi0_fast_base}} to reproduce fine-tuned performance on RoboCasa-Kitchen.
We train $\pi_0$ for 60K steps and $\pi_0$-FAST for 30K steps, both with a global batch size of 64. 
We set the learning rate to 2.5e-5 with cosine decay to 2.5e-6 and 1K warmup steps. 
At inference, we use an action horizon $H = 16$ and execute all actions without re-planning.

For RoboCasa-Kitchen, we evaluate all models with 1200 trials. 
For LIBERO, we evaluate 50 trials for each task, following~\citet{openvla}.


\subsection{Analysis Details}
\label{appendix:analysisdetail}
\textbf{Soft label target distance metric.}~
For the ablation study on soft label targets in Sec.~\ref{subsec:ablationstudy}, we define distances as follows. 
For next single action and current state, we use Euclidean distance. 
For next action sequence, we use Dynamic Time Warping (DTW), which measures similarity between temporal sequences that may vary in speed. 
DTW requires an additional temperature hyperparameter $\gamma$, which we set to 10.0. 
The soft weight temperature $\beta$ and similarity temperature $\tau$ are fixed at 1.0 and 0.2, respectively.

\textbf{CKNNA measurement.}~
CKNNA~\citep{cknna} is a nearest-neighbor variant of kernel alignment~\citep{cka}. 
We randomly sample 10 trajectories per task in RoboCasa-Kitchen, totaling 240 trajectories. 
Each trajectory is processed with a window size of 16, yielding 4415 transitions. 
We extract the embeddings from the adapter module $f_{\phi}$ (used as conditioning inputs to the action decoder) along with the corresponding proprioceptive states. 
We follow the implementation of \citet{cknna} and report results with $k=10$, measuring the alignment between proprioceptive states and conditional representations in the VLA model.
\newpage




%------------------------------------------------------------
\section{Real World Experiment Details}
\label{appendix:realworld}
\subsection{Hardware Platform}
We use Franka Research 3, a 7-DoF robotic arm equipped with a Robotiq 2F-85 gripper. For visual perception, we utilize the dual camera setup: a movable Stereolabs ZED 2 provides a global view, and a wrist-mounted ZED Mini captures a close-range view. Teleoperated demonstrations are collected using an Oculus Quest 2, and we log time-synchronized RGB images, joint states, and gripper width for training and evaluation. Demonstrations are recorded at 10 Hz.


\subsection{Real-World Tasks}
The in-domain and generalization tasks (visual, physical generalization, and language grounding) along with their corresponding prompts and representative key frames from the real-world evaluation, are shown in Fig.~\ref{figure:realtasks-in-domain}--~\ref{figure:realtasks-lang-gen}.

\textbf{In-domain tasks.}~
We introduce four pick-and-place tasks (Box to Bowl, Box to Plate, Basket to Bowl, Plate to Basket), with varied objects (teddy bear, blue cube, blue cup, yellow sponge) for each task (see Fig.~\ref{figure:realtasks-in-domain}).

\textbf{Visual generalization.}~
We use in-domain objects differing in  color (\emph{e.g.}, changing a blue cube to a green cube, or a yellow sponge to a blue sponge). 
We further introduce background variations by changing the tabletop covering or the target container (see Fig.~\ref{figure:realtasks-vis-gen}).

\textbf{Physical generalization.}~
We evaluate with unseen objects not used in training, including a yellow banana, purple grapes, red strawberry, and a yellow cup (different shape and texture from the blue cup used in training) (see Fig.~\ref{figure:realtasks-phy-gen}). 



\textbf{Language grounding.}~
We place two in-domain objects at the pick up location, and specify which one to pick up (see Fig.~\ref{figure:realtasks-lang-gen}).



\subsection{Real-World Training and Evaluation Details}
\textbf{Dataset.}~
We collect 60 demonstrations for each pick-and-place task and and for the close-lid task.


\textbf{Training.}~
We jointly train a model with the 4 pick-and-place tasks, and another model for the close-lid task.
For pick-and-place, we employ a cartesian action space with proprioceptive states, and for the close-lid task we use a joint action space to cover various configurations in manipulation.

\textbf{Evaluation.}~ 
For real-robot evaluation, we report the average success rate over 24 trials for each pick-and-place task, with varied objects. In the close-lid task, outcomes are classified as full success (lid fully closed), partial success (partially closed), or failure (not closed). 
For physical generalization, we evaluate on unseen objects (yellow banana, purple grapes, red strawberry, yellow cup), with success defined as the accurate completion of the pick-and-place.
We define language following as whether the gripper approaches the correct object, and task success as completing the instructed pick-and-place.

\clearpage

\begin{figure}[H]
    \centering
    \includegraphics[width=.95\linewidth]{assets/real_world_in_domain_tasks.pdf}
    \caption{Real-world in-domain tasks.}
    \label{figure:realtasks-in-domain}
\end{figure}
\vspace{-2.0em}
\begin{figure}[H]
    \centering
    \includegraphics[width=.95\linewidth]{assets/real_world_vis_gen_tasks.pdf}
    \caption{Real-world visual generalization tasks.}
    \label{figure:realtasks-vis-gen}
\end{figure}
\vspace{-2.0em}
\begin{figure}[H]
    \centering
    \includegraphics[width=.95\linewidth]{assets/real_world_phy_gen_tasks.pdf}
    \caption{Real-world physical generalization tasks.}
    \label{figure:realtasks-phy-gen}
\end{figure}
\vspace{-2.0em}
\begin{figure}[H]
    \centering
    \includegraphics[width=.95\linewidth]{assets/real_world_lang_gen_tasks.pdf}
    \caption{Real-world language grounding tasks.}
    \label{figure:realtasks-lang-gen}
\end{figure}
\vspace{-2.0em}




%------------------------------------------------------------
\section{Further Analysis}
\subsection{Contrastive Representation Regularization}
\label{appendix:clvsrscl}


\begin{table}[h]
\centering\small
\caption{
\textbf{RoboCasa-Kitchen benchmark success rate (\%)}. 
}
\vspace{-1.0em}
\label{appendixtable:robocasa}
\begin{tabular}{l ccc ccc ccc}
    \toprule
    \multirow{2.5}{*}{Method} &  \multicolumn{3}{c}{30 demos}  & \multicolumn{3}{c}{100 demos} & \multicolumn{3}{c}{300 demos} \\
    \cmidrule(lr){2-4} \cmidrule(lr){5-7} \cmidrule(lr){8-10} 
    & PnP  & Others & Avg. & PnP  & Others & Avg. & PnP  & Others & Avg. \\
    
    \midrule
    GR00T N1.5 \citep{gr00tn1_5} 
        & 30.8 & 56.9 & 48.2
        & 51.8 & 70.0 & 63.9 
        & 55.3 & 70.9 & 65.7 \\
    \textbf{\quad+ CL (Ours)}  
    & 36.0 & 55.0 & 48.1 
    & \textbf{59.3} & 69.0 & 65.0
    & 57.0 & 72.6 & 67.3 \\
    \textbf{\quad+ RS-CL (Ours)}  
    & \textbf{41.5} & \textbf{58.8} & \textbf{53.0}
    & 58.0 & \textbf{71.8} & \textbf{67.2} 
    & \textbf{59.8} & \textbf{74.6} & \textbf{69.7} \\
    
    \arrayrulecolor{black}\bottomrule
\end{tabular} 
\vspace{-1.0em}
\end{table}

\begin{table}[h]
\centering\small
\captionof{table}{
\textbf{LIBERO benchmark success rate (\%).} 
}
\vspace{-1.0em}
\begin{tabular}{l ccccc}
    \toprule
    {Method} 
    & {Spatial} 
    &  {Object} 
    & {Goal} 
    & {Long} 
    & {Avg.} \\ 
    
    \midrule
    GR00T N1.5 \citep{gr00tn1_5} 
    & 98.2 & \textbf{99.4} & 97.2 & 87.8 & 95.7 \\
    \textbf{\quad + CL (Ours)}  & 97.4 & 99.0 & 97.2& 87.4 & 95.3 \\
    \textbf{\quad + RS-CL (Ours)}  & \textbf{98.4} & 98.6 & \textbf{98.2 }& \textbf{90.4} & \textbf{96.4} \\

    \arrayrulecolor{black}\bottomrule
\end{tabular}
%}

\label{appendixtable:libero}
\end{table}

On RoboCasa-Kitchen, a contrastive representation regularization, without other supervision from low-level robotic signals (\emph{i.e.}, InfoNCE) improves the performance of GR00T N1.5 (\textbf{CL} at Table~\ref{appendixtable:robocasa}).
This result indicates the effectiveness of our proposed training framework, together with the augmentation strategy \textit{view cutoff}.
With further supervision from the robot’s proprioceptive states (\textbf{RS-CL} at Table~\ref{appendixtable:robocasa}), the performance further improves, highlighting the complementary benefit of incorporating proprioceptive information into VLM representations.


On LIBERO, CL performs comparably to the baseline (95.7\% vs. 95.3\%), but not improvements like RoboCasa-Kichen. 
This is likely due to the smaller batch size, where we train RoboCasa-Kitchen with a global batch size of 64, we train LIBERO with a global batch size of 32, for better performance of baseline GR00T N1.5 (bs64: 93.40 \% vs. bs32: 95.65 \%).
This reduces the number samples calculated in the contrastive path, leading to lower improvement.
However, with supervision of proprioceptive states (\textbf{RS-CL} at Table~\ref{appendixtable:robocasa}), the performance improves over baseline, despite the constraints.

\subsection{More Quantitative Results}
We report further results of our RS-CL on a VLA trained from SigLIP2~\citep{siglip2}, with varying number of demonstrations, and detailed results of our fine-tuning experiments in this section.


\begin{table}[H]
\centering\small
\vspace{-1.5em}
\caption{\textbf{Detailed results on RoboCasa-Kitchen.}~
Task-wise success rates of GR00T N1.5~\citep{gr00tn1_5} trained with, and without RS-CL, by different number of demonstrations.
} 
\label{tab:performance_comparison}
\resizebox{.9\textwidth}{!}{
    \begin{tabular}{lcccccc}
    \toprule
    \multirow{2.5}{*}{Task} & \multicolumn{3}{c}{GR00T N1.5 ($\mathcal{L_\textrm{FM}}$)} & \multicolumn{3}{c}{GR00T N1.5 ($\mathcal{L_\textrm{FM}}+\lambda\mathcal{L_\textrm{RS-CL}}$)} \\
    \cmidrule(lr){2-4} \cmidrule(lr){5-7}
    & 30 demos & 100 demos & 300 demos & 30 demos & 100 demos & 300 demos \\
    \midrule
    \multicolumn{7}{l}{RoboCasa Kitchen (24 tasks, PnP = Pick-and-Place)} \\
    \cmidrule(lr){1-7}
    Close Double Door &44.0& 86.0&80.0 &54.0&78.0&86.0\\
    Close Drawer &96.0& 96.0&96.0&96.0&96.0&96.0 \\
    Close Single Door &98.0& 94.0&98.0&88.0&98.0&98.0  \\
    Coffee Press Button &70.0& 82.0&90.0&86.0&94.0&92.0 \\
    Coffee Serve Mug &64.0& 72.0&58.0&74.0&66.0&70.0  \\
    Coffee Setup Mug &28.0& 34.0 &24.0&30.0&54.0&46.0 \\
    Open Double Door &80.0& 92.0&82.0&72.0&80.0&84.0  \\
    Open Drawer &46.0& 58.0&74.0&44.0&54.0&76.0 \\
    Open Single Door &64.0& 58.0&78.0&66.0&60.0&74.0  \\
    PnP from Cab to Counter &28.0& 42.0&54.0&38.0&54.0&60.0 \\
    PnP from Counter to Cab &36.0& 54.0&54.0&40.0&58.0&68.0 \\
    PnP from Counter to Microwave &30.0& 36.0&32.0&34.0&40.0&40.0  \\
    PnP from Counter to Sink &28.0& 66.0 &58.0&40.0&60.0&68.0 \\
    PnP from Counter to Stove &38.0& 60.0&66.0&38.0&74.0&72.0 \\
    PnP from Microwave to Counter &24.0& 44.0 &50.0&46.0&50.0&48.0 \\
    PnP from Sink to Counter &40.0& 52.0 &60.0&54.0&62.0&68.0 \\
    PnP from Stove to Counter &22.0& 60.0&68.0&42.0&66.0&54.0 \\
    Turn Off Microwave &62.0& 86.0&94.0&62.0&84.0&94.0  \\
    Turn Off Sink Faucet &72.0& 86.0&92.0&70.0&94.0&88.0 \\
    Turn Off Stove &10.0& 14.0&28.0&10.0&8.0&28.0 \\
    Turn On Microwave &44.0& 58.0 &44.0&48.0&72.0&66.0 \\
    Turn On Sink Faucet &60.0& 90.0 &86.0&72.0&90.0&90.0 \\
    Turn On Stove &34.0& 56.0 &32.0 &36.0&58.0&36.0 \\
    Turn Sink Spout &38.0& 58.0 &78.0 &32.0&62.0&70.0 \\
    \midrule
    \textbf{Average} & \textbf{48.2} & \textbf{63.9} & \textbf{65.7} & \textbf{53.0} & \textbf{67.2} & \textbf{69.7} \\
    \bottomrule
    \end{tabular}
}
\vspace{-2.5em}
\end{table}

\begin{table}[H]
\centering\small
\caption{\textbf{Detailed results on RoboCasa-Kitchen.}~
Task-wise success rates (\%) of reproduced $\pi_0$~\citep{pi0} and $\pi_0$-FAST~\citep{pi0fast}, by different number of demonstrations.
} 
\label{tab:pi_robocasaperofrmance}
\resizebox{.9\textwidth}{!}{
    \begin{tabular}{lcccccc}
    \toprule
    \multirow{2.5}{*}{Task} & \multicolumn{3}{c}{$\pi_0$} & \multicolumn{3}{c}{$\pi_0$-FAST} \\
    \cmidrule(lr){2-4} \cmidrule(lr){5-7}
    & 30 demos & 100 demos & 300 demos & 30 demos & 100 demos & 300 demos \\
    \midrule
    \multicolumn{7}{l}{RoboCasa Kitchen (24 tasks, PnP = Pick-and-Place)} \\
    \cmidrule(lr){1-7}
    Close Double Door       & 68.0 & 86.0 & 86.0 & 44.0 & 84.0 & 78.0 \\
    Close Drawer            & 94.0 & 94.0 & 96.0 & 84.0 & 96.0 & 94.0 \\
    Close Single Door       & 94.0 & 98.0 & 96.0 & 84.0 & 90.0 & 72.0 \\
    Coffee Press Button     & 66.0 & 80.0 & 88.0 & 20.0 & 82.0 & 90.0 \\
    Coffee Serve Mug        & 80.0 & 66.0 & 64.0 & 44.0 & 66.0 & 68.0 \\
    Coffee Setup Mug        & 20.0 & 32.0 & 38.0 & 2.0  & 34.0 & 38.0 \\
    Open Double Door        & 92.0 & 90.0 & 84.0 & 26.0 & 68.0 & 78.0 \\
    Open Drawer             & 44.0 & 56.0 & 62.0 & 36.0 & 58.0 & 68.0 \\
    Open Single Door        & 58.0 & 64.0 & 70.0 & 44.0 & 70.0 & 66.0 \\
    PnP Cab $\to$ Counter   & 14.0 & 22.0 & 18.0 & 12.0 & 22.0 & 30.0 \\
    PnP Counter $\to$ Cab   & 32.0 & 44.0 & 46.0 & 8.0  & 58.0 & 48.0 \\
    PnP Counter $\to$ Microwave & 26.0 & 30.0 & 18.0 & 10.0 & 32.0 & 20.0 \\
    PnP Counter $\to$ Sink  & 32.0 & 44.0 & 58.0 & 2.0  & 46.0 & 56.0 \\
    PnP Counter $\to$ Stove & 14.0 & 32.0 & 60.0 & 10.0 & 50.0 & 64.0 \\
    PnP Microwave $\to$ Counter & 16.0 & 20.0 & 24.0 & 4.0  & 38.0 & 46.0 \\
    PnP Sink $\to$ Counter  & 22.0 & 24.0 & 66.0 & 12.0 & 56.0 & 62.0 \\
    PnP Stove $\to$ Counter & 10.0 & 46.0 & 44.0 & 18.0 & 62.0 & 60.0 \\
    Turn Off Microwave      & 64.0 & 84.0 & 96.0 & 68.0 & 98.0 & 96.0 \\
    Turn Off Sink Faucet    & 72.0 & 86.0 & 94.0 & 48.0 & 76.0 & 94.0 \\
    Turn Off Stove          & 14.0 & 10.0 & 22.0 & 0.0  & 18.0 & 22.0 \\
    Turn On Microwave       & 58.0 & 82.0 & 70.0 & 52.0 & 68.0 & 88.0 \\
    Turn On Sink Faucet     & 80.0 & 82.0 & 86.0 & 40.0 & 66.0 & 74.0 \\
    Turn On Stove           & 26.0 & 68.0 & 42.0 & 12.0 & 52.0 & 38.0 \\
    Turn Sink Spout         & 50.0 & 68.0 & 72.0 & 36.0 & 54.0 & 76.0 \\
    \midrule
    \textbf{Average}        & \textbf{47.8} & \textbf{58.7} & \textbf{62.5} & \textbf{29.8} & \textbf{60.2} & \textbf{63.6} \\
    \bottomrule
    \end{tabular}
}
\vspace{-2.0em}
\end{table}

\begin{table}[H]
\centering\small
\caption{
\textbf{RoboCasa-Kitchen benchmark success rate (\%)}. Employing SigLIP2 as our VLM backbone, we train a VLA model from scratch and report the average success rate by different number of demonstrations.
}

\begin{tabular}{l c c c}
    \toprule
    \multirow{2.5}{*}{Method} &  \multicolumn{3}{c}{ \# of Demos} \\
    \cmidrule(lr){2-4}
    & {30 demos}  & {100 demos} & {300 demos} \\
    
    \midrule
    SigLIP2 backbone VLA
        &2.7 & 2.4 & \phantom{0}4.0 \\
    \rowcolor{green!10}
    \textbf{\quad+ RS-CL (Ours)}  
    &\textbf{8.0} & \textbf{9.1} & \textbf{14.1} \\
    
    \bottomrule
\end{tabular} 

\end{table}

\begin{table}[ht!]
\centering\small
\caption{
\textbf{Detailed results of from-scratch experiments.} 
Task success rate (\%) on the RoboCasa-Kitchen benchmark trained with 300 demonstrations. All models train a VLA from scratch, starting from each pre-trained VLM backbone. Best results within the same backbone indicated in \textbf{bold}.
}
\vspace{-1.0em}
\label{maintable:vlmtraining}
\begin{tabular}{l ccc}
    \toprule
    \multirow{2.5}{*}{Backbone Model} & \multicolumn{3}{c}{Success Rate} \\
    \cmidrule{2-4}
    & PnP & Others & Avg. \\
    
    \midrule
    Qwen2.5-VL-3B~\citep{qwen2_5vl}
        & \phantom{0}2.5 
        & \phantom{0}8.6
        & \phantom{0}6.6 \\
    \textbf{\quad+ RS-CL (Ours)} 
        & \textbf{\phantom{0}3.5}
        & \textbf{16.0} 
        & \textbf{11.8} \\
    NORA~\citep{nora} 
        & \phantom{0}1.5 
        & 11.4 
        & \phantom{0}8.1 \\
    \textbf{\quad+ RS-CL (Ours)} 
        & \textbf{\phantom{0}3.5} 
        & \textbf{23.3} 
        & \textbf{16.7} \\
    RoboBrain2.0-3B~\citep{robobrain20}  & \phantom{0}2.8 & 13.9 & 10.2 \\
    \textbf{\quad+ RS-CL (Ours)} 
        & \textbf{\phantom{0}5.8} 
        & \textbf{19.6} 
        & \textbf{15.0} \\ 
    \midrule
    Qwen2.5-VL-7B~\citep{qwen2_5vl}  
        & \phantom{0}2.5 
        & 12.4 
        & \phantom{0}9.1 \\
    \textbf{\quad+ RS-CL (Ours)} 
        & \textbf{\phantom{0}9.8}
        & \textbf{21.1}
        & \textbf{17.3} \\

    RoboBrain2.0-7B~\citep{robobrain20}  & \phantom{0}2.3 & 12.8 & \phantom{0}9.3 \\
    \textbf{\quad+ RS-CL (Ours)} 
        & \textbf{12.0} 
        & \textbf{25.9} 
        & \textbf{21.3} \\

    VeBrain-7B~\citep{vebrain}  & \phantom{0}3.0 & 10.9 & \phantom{0}8.3 \\
    \textbf{\quad+ RS-CL (Ours)} 
        & \textbf{\phantom{0}7.8}
        & \textbf{20.3}
        & \textbf{17.6} \\

    Cosmos-Reason-7B~\citep{cosmosreason}  & \phantom{0}1.0 & \phantom{0}5.5 & \phantom{0}4.0 \\
    \textbf{\quad+ RS-CL (Ours)} 
        & \textbf{\phantom{0}7.3}
        & \textbf{15.9}
        & \textbf{13.0} \\
    \midrule
    SigLIP2~\citep{siglip2}
        & \phantom{0}0.3
        & \phantom{0}2.9 
        & \phantom{0}2.0 \\
    \textbf{\quad+ RS-CL (Ours)} 
        & \textbf{\phantom{0}0.8} 
        & \textbf{\phantom{0}3.5} 
        & \textbf{\phantom{0}2.6} \\
    SigLIP2, unfrozen backbone
        & \phantom{0}3.3
        & \phantom{0}4.4 
        & \phantom{0}4.0 \\
    \textbf{\quad+ RS-CL (Ours)} 
        & \textbf{17.3} 
        & \textbf{12.5} 
        & \textbf{14.1} \\
    GR00T N1.5 VLM~\citep{gr00tn1_5} \ & 37.5 & 62.0 & 53.8 \\
    \textbf{\quad+ RS-CL (Ours)} 
        & \textbf{37.8} 
        & \textbf{66.3} 
        & \textbf{56.8} \\
    
    \bottomrule
\end{tabular} 
\end{table}


\section{Discussion}
\vspace{0.01in}
\textbf{Limitations.}~
While RS-CL explicitly leverages proprioceptive states to align the representation space, it does not incorporate further signals in robotic manipulation, such as object poses or contact forces. 
These modalities often provide complementary information that is captured by robot's proprioception state. 
Extending RS-CL to integrate such modalities into the representations, represents an promising direction for future research.

\textbf{Future directions.}~
One promising extension is to apply RS-CL to settings with more complex proprioceptive spaces, such as humanoid robots or dexterous hand manipulation tasks. 
These domains involve high-dimensional and complex state representations, where aligning VLM embeddings with proprioceptive signals may be even more beneficial for accurate action prediction.
%------------------------------------------------------------
\section{Use of Large Language Models}
Large language models were used to assist with drafting and polishing the writing of this paper.


\clearpage
