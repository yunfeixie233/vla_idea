\documentclass[conference]{IEEEtran}
\usepackage{times}

% numbers option provides compact numerical references in the text. 
\usepackage[numbers]{natbib}
\usepackage{multicol}
\usepackage[bookmarks=true]{hyperref}
\usepackage{xcolor}
\usepackage{listings}

\newcommand\myshade{85}
\colorlet{mylinkcolor}{violet}
\colorlet{mycitecolor}{orange}
\colorlet{myurlcolor}{blue}

\definecolor{dkgreen}{rgb}{0,0.6,0}
\definecolor{gray}{rgb}{0.5,0.5,0.5}
\definecolor{mauve}{rgb}{0.58,0,0.82}

\lstset{frame=tb,
  language=Python,
  aboveskip=3mm,
  belowskip=3mm,
  showstringspaces=false,
  columns=flexible,
  basicstyle={\small\ttfamily},
  numbers=none,
  numberstyle=\tiny\color{gray},
  keywordstyle=\color{blue},
  commentstyle=\color{dkgreen},
  stringstyle=\color{mauve},
  breaklines=true,
  breakatwhitespace=true,
  tabsize=3
}

\hypersetup{
  linkcolor  = mylinkcolor!\myshade!black,
  citecolor  = mycitecolor!\myshade!black,
  urlcolor   = myurlcolor!\myshade!black,
  colorlinks = true,
  linktoc = none,
}


\pdfinfo{
  /Author (Mahi Shafiullah)
  /Title  (The Surprising Effectiveness of Representation Learning for Visual Imitation)
  /CreationDate (D:20211123120000)
  /Subject (Visual Imitation Through Nearest Neighbors)
  /Keywords (Visual Imitation;Imitation Learning;VINN;Robot learning;Robot imitation;Representation Learning)
}

\IEEEoverridecommandlockouts                              % This command is only needed if 
\overrideIEEEmargins                                      % Needed to meet printer requirements.
\pdfminorversion=4


\input{math_commands.tex}

% The following packages can be found on http:\\www.ctan.org
%\usepackage{epsfig} % for postscript graphics files
%\usepackage{mathptmx} % assumes new font selection scheme installed
%\usepackage{times} % assumes new font selection scheme installed
%\usepackage{amssymb}  % assumes amsmath package installed
% \usepackage[noadjust]{cite}
\usepackage{graphicx}
\usepackage{amsmath}
\usepackage{algpseudocode}
\usepackage{algorithm}
\usepackage[algo2e]{algorithm2e}
\usepackage{multirow}
\usepackage[normalem]{ulem}
% \usepackage{hyperref}
\usepackage{booktabs}
% \usepackage{natbib}
\usepackage{graphicx}
% \def\hidenotes
% Feel free to change the color of your note comments
\newcommand{\lpnote}[1]{{\xxnote{LP}{blue}{#1}}}
\newcommand{\mnote}[1]{{\xxnote{MS}{red}{#1}}}
% \newcommand{\TK}[1]{{\xxnote{TK}{green}{#1}}}
% \newcommand{\wynote}[1]{{\xxnote{WY}{magenta}{#1}}}
% \newcommand{\ywnote}[1]{{\xxnote{YW}{yellow}{#1}}}
\newcommand{\xxnote}[3]{}
\ifx\hidenotes\undefined
  \renewcommand{\xxnote}[3]{\color{#2}{#1: #3}}
\fi

% \renewcommand{\baselinestretch}{0.995}
\hypersetup{
    colorlinks=true,
    linkcolor=blue,
    filecolor=magenta,      
    urlcolor=blue,
}


\title{\LARGE \bf The Surprising Effectiveness of Representation Learning \\for Visual Imitation}
\author{%
\\
Jyothish Pari$^\ast$\\
New York University\\
\texttt{jp5981@nyu.edu}
\and
Nur Muhammad\\(Mahi) Shafiullah$^\ast$\\
New York University\\
\texttt{mahi@cs.nyu.edu}
\and
Sridhar Pandian\\Arunachalam\\
New York University\\
\texttt{sa5914@nyu.edu}
\and
\\
Lerrel Pinto \\
New York University\\
\texttt{lerrel@cs.nyu.edu}
}%

% \author{%
% Jyothish Pari
% \and
% Nur Muhammad\\(Mahi) Shafiullah
% \and
% Sridhar Pandian\\Arunachalam
% \and
% Lerrel Pinto\\
% }

%
% \affil{New York University}

% \author{
%     \IEEEauthorblockN{Jyothish Pari}
%     \and
%     \IEEEauthorblockN{Nur Muhammad (Mahi) Shafiullah}
%     \IEEEauthorblockN{Sridhar Pandian Arunachalam, Lerrel Pinto}
%     \IEEEauthorblockA{New York University
%     \\\{jyo, mahi\}@nyu.edu}
% }


\begin{document}
\maketitle
\thispagestyle{empty}
\pagestyle{empty}

\def\thefootnote{*}\footnotetext{The first two authors contributed equally to this work.}\def\thefootnote{\arabic{footnote}}
%%%%%%%%%%%%%%%%%%%%%%%%%%%%%%%%%%%%%%%%%%%%%%%%%%%%%%%%%%%%%%%%%%%%%%%%%%%%%%%%
\begin{abstract}
While visual imitation learning offers one of the most effective ways of learning from visual demonstrations, generalizing from them requires either hundreds of diverse demonstrations, task specific priors, or large, hard-to-train parametric models.
One reason such complexities arise is because standard visual imitation frameworks try to solve two coupled problems at once: learning a succinct but good representation from the diverse visual data, while simultaneously learning to associate the demonstrated actions with such representations.
Such joint learning causes an interdependence between these two problems, which often results in needing large amounts of demonstrations for learning.
To address this challenge, we instead propose to decouple representation learning from behavior learning for visual imitation. First, we learn a visual representation encoder from offline data using standard supervised and self-supervised learning methods. Once the representations are trained, we use non-parametric Locally Weighted Regression to predict the actions.
We experimentally show that this simple decoupling improves the performance of visual imitation models on both offline demonstration datasets and real-robot door opening compared to prior work in visual imitation. All of our generated data, code, and robot videos are publicly available at \url{https://jyopari.github.io/VINN/}.
\end{abstract}

%%%%%%%%%%%%%%%%%%%%%%%%%%%%%%%%%%%%%%%%%%%%%%%%%%%%%%%%%%%%%%%%%%%%%%%%%%%%%%%%
\section{Introduction}
% \lpnote{TODO: Lerrel}

Imitation learning serves as a powerful framework for getting robots to learn complex skills in visually rich environments~\cite{zhang2018deep, stadie2017third,duan2017one,zhu2018reinforcement,young2020visual}. 
Recent works in this area have shown promising results in generalization to previously unseen environments for robotic tasks such as pick and place, pushing, and rearrangement~\cite{young2020visual}. 
However, such generalization is often too narrow to be directly applied in the diverse real-world application. 
For instance, policies trained to open one door rarely generalize to opening different doors~\cite{DBLP:journals/corr/abs-1908-01887}.
This lack of generalization is further exacerbated by the plethora of different options to achieve generalization: either needing hundreds of diverse demonstrations, task-specific priors, or large parametric models. 
This begs the question: What really matters for generalization in visual imitation?

An obvious answer is visual representation -- generalizing to diverse visual environments should require powerful representation learning. 
Prior work in computer vision~\cite{byol,simclr,moco2,swav,bardes2021vicreg} have shown that better representations significantly improve downstream performance for tasks such as image classification.
However, in the case of robotics, evaluating the performance of visual representations is quite complicated.
Consider behavior cloning~\cite{torabi2018behavioral}, one of the simplest methods of imitation. 
Standard approaches in behavior cloning fit convolutional neural networks on a large dataset of expert demonstrations using end-to-end gradient descent.
Although powerful, such models conflate two fundamental problems in visual imitation: (a) representation learning, i.e. inferring information-preserving low-dimensional embeddings from high-dimensional observations and (b) behavior learning, i.e. generating actions given representations of the environment state. This joint learning often results in large dataset requirements for such techniques. 

\begin{figure}[t]
  \begin{center}
    \includegraphics[width = \linewidth]{figures/VINN_intro.pdf}
  \end{center}
  \vspace{-0.1in}
  \caption{Consider the task of opening doors from visual observations. VINN, our visual imitation framework first learns visual representations through self-supervised learning. Given these representations, non-parametric weighted nearest neighbors from a handful of demonstrations is used to compute actions, which results in robust door-opening behavior.}
  \vspace{-0.1in}
\label{fig:intro}
\end{figure}

One way to achieve this decoupling is to use representation modules pre-trained through standard proxy tasks such as image classification, detection, or segmentation~\cite{sax2019learning}.
However, this relies on large amounts of labelled human data on datasets that are often significantly out of distribution to robot data~\cite{chen2020robust}.
A more scalable approach is to take inspiration from recent work in computer vision, where visual encoders are trained using self-supervised losses~\cite{moco2, simclr, byol}.
These methods allow the encoders to learn useful features of the world without requiring human labelling.
There has been recent progress in vision-based Reinforcement Learning (RL) that improves performance by creating this explicit decoupling \cite{stooke2021decoupling, yarats2021reinforcement}.
Visual imitation has a significant advantage over RL settings: learning visual representations in RL is further coupled with challenges in exploration~\cite{yarats2021mastering}, which has limited its application in real-world settings due to poor sample complexity. 

In this work we present a new and simple framework for visual imitation that decouples representation learning from behavior learning. First, given an offline dataset of experience, we train visual encoders that can embed high-dimensional visual observations to low-dimensional representations. Next, given a handful of demonstrations, for a new observation, we find its associated nearest neighbors in the representation space. For our agent's behavior on that new observation, we use a weighted average of the nearest neighbors' actions. This technique is inspired by Locally Weighted Regression~\cite{atkeson1997locally}, where instead of operating on state estimates, we operate on self-supervised visual representations. Intuitively, this allows the behavior to roughly correspond to a Mixture-of-Experts model trained on the visual demonstrations. Since nearest neighbors is non-parametric, this technique requires no additional training for behavior learning. We will refer to our framework as Visual Imitation through Nearest Neighbors (VINN).

Our experimental analysis demonstrates that VINN can successfully learn powerful representations and behaviors across three manipulation tasks: Pushing, Stacking, and Door Opening. 
Surprisingly, we find that non-parametric behavior learning on top of learned representations is competitive with end-to-end behavior cloning methods. On offline MSE metrics, we report results on par with competitive baselines, while being significantly simpler. 
To further test the real-world applicability of VINN, we run robot experiments on opening doors using 71 visual demonstrations. Across a suite of generalization experiments, VINN succeeds 80\% on doors present in the demonstration dataset and 40\% on opening the door in novel scenes. 
In contrast, our strongest baselines have success rates of 53.3\% and 3.3\% respectively.

To summarize, this paper presents the following contributions. 
First, we present VINN, a novel yet simple to implement visual imitation framework that derives non-parametric behaviors from learned visual representations. 
Second, we show that VINN is competitive to standard parametric behavior cloning and can outperform it on a suite of manipulation tasks. Third, we demonstrate that VINN can be used on real robots for opening doors and can achieve high generalization performance on novel doors. Finally, we extensively ablate over and analyze different representations, amount of training data, and other hyperparameters to demonstrate the robustness of VINN.
% on our project website. 

% since the performance of a visual robotics system is not just dependent on its representations but also on how the representations are used by action prediction. While prior work in Reinforcement Learning provides approaches to tackle this coupling, they are further conflated with challenges in exploration~\cite{}. 

% The coupling of representation and behavior learning is perhaps most evident in behavior cloning, one of the simplest methods of imitation and hence serves as a reproducible testbed for our analysis. Standard approaches in behavior cloning fit convolutional neural networks on a large dataset of expert demonstrations using end-to-end gradient descent. Although powerful, such models conflate two fundamental problems in visual imitation -- representation learning, i.e. inferring low-dimensional encodings from high-dimensional observations and behavior learning, i.e. generating actions given representations of the environment. 

% Behavior cloning from observation one of the simplest methods of imitation, behavior cloning from observations~\cite{}. Popular techniques fit convolutional neural networks on a large dataset of expert demonstrations using end-to-end gradient descent. Given sufficient data and modelling capacity, behavior cloning is capable of solving challenging manipulation and navigation problems. Although powerful, such models conflate two fundamental problems in visual imitation -- representation learning, i.e. inferring low-dimensional encodings from high-dimensional observations and behavior learning, i.e. generating actions given representations of the environment. 


% % para 1: Introducing the challenge of generalization for visual imitation. What works, what doesnt. Allude to decoupling representation with the control aspects.

% para 2: However, generalization is narrow. Models trianed to open one door rarely generalize to opening different doors. This is in contrast to human vision. Inspiration from human vision. Our representations are largely task-agnostic. We dont have independent representations for solving each narrow task distribution.

% para 3: The solution is to decouple representation learning from control and to question the end-to-end vision to action control paradigm for visual imitation. What does decoupling representations from control actually mean? What are the challenges associated with this decoupling? How do you learn representations without.

% para 4: In this work we present a new approach for visual imitation that focuses on showcasing the power of self-supervised representation learning. Unlike supervised representation learning methods~\cite{}, we do not need human labels for providing expensive image annotation. But how do we test out the learned representations.

% para 5: We use the non-parameteric weighted Nearest Neighbours. Simple to implement and interpretable technique. And this works -- surpringly well.

% para 6: Summarize contributions

%%%%%%%%%%%%%%%%%%%%%%%%%%%%%%%%%%%%%%%%%%%%%%%%%%%%%%%%%%%%%%%%%%%%%%%%%%%%%%%%
\section{Related Work}

\begin{figure*}[h]
  \begin{center}
    \includegraphics[width = 1.0\textwidth]{figures/VINN_arch.pdf}
  \end{center}
  \caption{Overview of our VINN algorithm. During training, we use offline visual data to train a BYOL-style self-supervised model as our encoder. During evaluation, we compare the encoded input against the encodings of our demonstration frames to find the nearest examples to our query. Then, our model's predicted action is just a weighted average of the associated actions from the nearest images.}
\label{fig:arch}
\end{figure*}

\subsection{Imitation via Cloning}
Imitation learning is frequently used to learn skills and behaviors from human demonstrations~\cite{piaget2013play, meltzoff1977imitation, meltzoff1983newborn, tomasello1993imitative}. 
In the context of manipulation, such techniques have successfully solved a variety of problems in pushing, stacking, and grasping~\cite{zhang2018deep,DBLP:journals/corr/abs-1802-09564,argall2009survey,hussein2017imitation}. Behavioral Cloning (BC) \cite{torabi2018behavioral} is one of the most common techniques. 
% A comprehensive review of imitation learning can be found in \cite{argall2009survey, hussein2017imitation}.
If the agent's morphology or viewpoint is different than the demonstrations', the model needs to involve techniques such as transfer learning to resolve this domain gap~\cite{stadie2017third,sermanet2016unsupervised}. 
To close this unintended domain gap, \cite{zhang2018deep} has used tele-operation methods, while \cite{song2020grasping, young2020visual} have used assistive tools. 
Using assistive tools provides us the benefit of being a able to scalably collect diverse demonstrations. In this paper, we follow the DemoAT \cite{young2020visual} framework to collect expert demonstrations.

\subsection{Visual Representation Learning}

In computer vision, interest in learning a good representation has been longstanding, especially when labelled data is rare or difficult to collect \cite{simclr, moco2, byol, swav}.
This large class of representation learning techniques aim to extract features that can help other models improve their performance in some downstream learning tasks, without needing to explicitly learn a label.
In such tasks, first a model is trained on one or more pretext tasks with this unlabeled dataset to learn a representation.
Such tasks generally include instance invariance, or predicting some image transformation parameters (e.g. rotation and distortion), patches, or frame sequence~\cite{gidaris2018unsupervised,dosovitskiy2015discriminative,doersch2016unsupervised,misra2016shuffle,simclr,moco2,wu2018unsupervised}.
% Some works have proposed simultaneously training these pretext tasks alongside the main objective~\cite{zhai2019s4l, sun2019unsupervised}.
In representation learning, the performance of the model on the pretext task is usually disregarded.
Instead, the focus is on
% the model pretrained through these pretext tasks and 
the input domain to representation mapping that these models have learned.
Ideally, to solve such pretext tasks, the pretrained model may have learned some useful structural meaning and encoded it in the representation.
Thus, intuitively, such a model can be used in downstream tasks where there is not enough data to learn this structural meaning directly from the available task-relevant data.
Unsupervised representation learning, in works such as \cite{simclr, moco2, byol, swav, bardes2021vicreg, dwibedi2021little}, has shown impressive performance gains on difficult benchmarks since they can harness a large amounts of unlabelled data unavailable in task-specific datasets.

Recently, interest in unsupervised or semi-supervised representation learning technique has grown within robotics~\cite{manuelli2020keypoints} due to the availability of unlabeled data and its effectiveness in visual imitation tasks ~\cite{young2021playful,zhan2020framework}.
We follow a BYOL-style~\cite{byol} self-supervised representation learning framework in our experiments.

\subsection{Non-parametric Control}
Non-parametric models are those, which instead of modeling some parameters about the data distribution, tries to express it in terms of previously observed training data. Non-parametric models are significantly more expressive, but as a downside to this, they usually require a large number of training examples to generalize well. A popular and simple example of non-parametric models is Locally Weighted Learning (LWL)~\cite{atkeson1997locally}. LWL is a form of instance-based, non-parametric learning that refers to algorithms whose response to any query is a weighted aggregate of similar examples. 
% There, similarity is measured in distance in some representation space.
Simple nearest neighbor models are an example of such learning, where all weight is put on the closest neighbor to the input point. Nearest neighbor methods have been successfully used in previous works for control tasks ~\cite{mansimov2018simple}
More sophisticated, $k$-NN algorithms base their predictions on an aggregate of the nearest $k$ points ~\cite{10.1007/978-1-4612-2660-4_33}.

Uses of LWL based methods in supervised learning, robotics, and reinforcement learning is quite old.
In works like~\cite{snell2017prototypical, wang2019simpleshot}, effectiveness of LWL algorithms like k-nearest neighbor has shown competitive success in difficult, high dimensional tasks like classifying the miniImageNet.
LWL has also shown success for robotic control problems~\cite{atkeson1997locally}, although it requires an accurate state-estimator to obtain low-dimensional states.
In~\cite{lee2016robust,pritzel2017neural,rajeswaran2018generalization}, elements of non-parametric learning is weaved into the reinforcement learning algorithms to create models which can adjust their complexity based on the amount of available data.
Finally, in works like~\cite{shah2018qlearning} non-parametric k-Nearest Neighbor regression based Q-functions are shown to give a good approximation of the true Q function under some theoretical assumptions.
Our work, VINN, draws inspiration from the simplicity of LWL and demonstrates the usefulness of this idea by using Locally Weighted Regression in challenging visual robotic tasks.

% In this work, we use a form of Locally Weighted regression with a weighting function~\cite{atkeson1997locally} to compute the particular action for an observation given similar observations among our demonstrations. 



%%%%%%%%%%%%%%%%%%%%%%%%%%%%%%%%%%%%%%%%%%%%%%%%%%%%%%%%%%%%%%%%%%%%%%%%%%%%%%%%
% \section{Background}
% I dont think we need background. We can describe high level in related work.

% \subsection{Representation Learning through Bootstrapping}

% \subsection{Weighted Nearest Neighbours}
\section{Approach}
% \lpnote{TODO: Mahi}

In this section, we describe the components of our algorithms and how they fit together to create VINN. As seen in Fig.~\ref{fig:arch}, VINN consists of two parts: (a) training an encoding network on offline visual data, and (b) querying against the provided demonstrations for a nearest-neighbor based action prediction. 

\subsection{Visual Representation Learning}

Given an offline dataset of visual experience from the robot, we first learn a visual representation embedding function.
In this work, we use two key insights for learning our visual representation: first, we can learn a good vision prior using existing large but unrelated real world datasets, and then, we can fine-tune starting from that prior using our demonstration dataset, which is small but relevant to the task at hand.

For the first insight, whenever possible, we initialize our models from an ImageNet-pretrained model. Such models are provided with the PyTorch~\cite{paszke2019pytorch} library that we use and can be achieved by simply adding a single parameter to the model initialization function call. 

Then, we use self supervised learning and train this visual encoder on the all the frames in our offline training dataset.
In this work, we use Bootstrap Your Own Latent (BYOL)~\cite{byol} as the self-supervision objective.
As illustrated in Fig.~\ref{fig:arch}, BYOL uses two versions of the same encoder network: one normally updating online network, and a slow moving average of the online network called the target network.
The BYOL self-supervised loss function tries to reduce the discrepancy in the two heads of the network when they are fed with differently augmented version of the same image.
Although we use BYOL in this work, VINN can also work with other self-supervised representation learning methods~\cite{simclr,moco2, swav,bardes2021vicreg} (Table~\ref{tab:mse-table}).

In practice, we initialize both the BYOL online and target networks with an ImageNet-pretrained encoder. 
Then, using the BYOL objective, we finetune them to better fit our image distribution.
Once the self-supervised training is done, we encode all our training demonstration frames with the encoder to obtain a set of their embeddings, $E$.

\subsection{$k$-Nearest Neighbors Based Locally Weighted Regression}
\begin{figure*}[ht]
  \begin{center}
    \includegraphics[width = \linewidth]{figures/VINN_opt.pdf}
  \end{center}
  \caption{Nearest neighbor queries on the encoded demonstration dataset; the query image is on the first column, and the found nearest neighbors are on the next three columns. The associated action is shown with a green arrow. The bottom right set of nearest neighbors demonstrates the advantage of performing a weighted average over nearest neighbors' actions instead of copying the nearest neighbor's action.}
\label{figure:sanity_check}
\end{figure*}

The set of embeddings $E$ given by our encoder holds compact representations of the demonstration images.
Thus, during test time, given an input we search for demonstration frames with similar features.
We find the nearest neighbors of the encoded input $e$ from the set of demonstration embeddings, $E$.
In Fig.~\ref{figure:sanity_check}, we see that these nearest neighbors are visually similar to the query image.
Our algorithm implicitly assumes that a similar observation must result in a similar action. 
Thus, once we have found the $k$ nearest neighbors of our query, we set the next action as an weighted average of the actions associated with those $k$ nearest neighbors.

Concretely, this is done by performing nearest neighbors search based on the distance between embeddings: $\|e - e^{(i)}\|_2$, where $e^{(i)}$ is the $i^{th}$ nearest neighbor. Once we find the $k$ nearest neighbors and their associated actions, namely $(e^{(1)}, a^{(1)}), (e^{(2)}, a^{(2)}), \cdots, (e^{(k)}, a^{(k)})$, we set the action as the Euclidean kernel weighted average \cite{atkeson1997locally} of those examples' associated actions:
% \begin{equation}
\[\hat a = \frac{\sum_{i=1}^k \exp \left ( {- \|e - e^{(i)}\|_2}\right ) \cdot a^{(i)} } {\sum_{i=1}^k \exp \left ( {- \|e - e^{(i)}\|_2}\right )}\]
% \end{equation}
In practice, this turns out to be the average of the observations' associated actions weighted by the SoftMin of their distance from the query image in the embedding space.

% \subsection{Representation Learning for Visual Imitation}

\subsection{Deployment in real-robot door opening}
\label{sec:demoat}
% \lpnote{Lerrel: Maybe this should go into experimental details?}
% \lpnote{Maybe this can involve the changes we made to the original VIME setup. Like the gripper type, camera, action extraction?}
For our robotic door opening task, we collect demonstrations using the DemoAT~\cite{young2020visual} tool.
Here, a reacher-grabber is mounted with a GoPro camera to collect a video of each trajectory.
We pass the series of frames into a structure from motion (SfM) method which outputs the camera's location in a fixed frame ~\cite{ozyesil2017survey}.
From the sequence of camera poses, which consist of coordinate and orientation, we extract translational motion which becomes our action.
To extract the gripper state, we train a gripper network that outputs a distribution over four classes (open, almost open, almost closed, closed), which represent various stages of gripping.
Then, we feed 
these images and their corresponding actions into our imitation learning method.

To train our visual encoders, we train ImageNet-pretrained BYOL encoders on individual frames in our demonstration dataset without action information. This same dataset with action information serves as the demonstration dataset for the $k$-NN based action prediction. Note that although we use task-specific demonstrations for representation learning, our framework is compatible with using other forms of unlabelled data such offline datasets~\cite{gulcehre2020rl,fu2020d4rl} or task-agnostic play data~\cite{young2021playful}.

To execute our door-opening skill on the robot, we run our model on a closed loop manner. After resetting the robot and the environment, on every step, we retrieve the robot observation and query the model with it. The model returns a translational action $\hat a$ as well as the gripper state $g$, and the robot moves $c \odot \hat a$ where the vector $c$ is a hyper-parameter with each element $< 1$ to mitigate our SfM model's inaccuracies and improve transfer from human demonstrations to robot execution. In addition, for nearest neighbor based methods, we have hyper-parameters that map the floating value $g$ into a gripper state which was tuned per experiment. 

%%%%%%%%%%%%%%%%%%%%%%%%%%%%%%%%%%%%%%%%%%%%%%%%%%%%%%%%%%%%%%%%%%%%%%%%%%%%%%%%
\section{Experimental Evaluation}\label{sec:experiments}
% \lpnote{TODO: Mahi}

\begin{figure*}[!htb]
\minipage{0.33\textwidth}
   \includegraphics[width=\linewidth]{figures/comparison_push.pdf}
  %\caption{A really Awesome Image}\label{fig:awesome_image1}
\endminipage\hfill
\minipage{0.33\textwidth}
  \includegraphics[width=\linewidth]{figures/comparison_stack.pdf}
  %\caption{A really Awesome Image}\label{fig:awesome_image2}
\endminipage\hfill
\minipage{0.33\textwidth}%
  \includegraphics[width=\linewidth]{figures/comparison_handle.pdf}
  %\caption{A really Awesome Image}\label{fig:awesome_image3}
\endminipage
 \caption{Mean Squared Error for the Pushing, Stacking and Door Opening (left to right) datasets of different algorithms trained on subsamples of the original dataset. End-to-end behavior cloning initialized with ImageNet-trained features perform as well as VINN for larger datasets, but fixed representation based methods outperforms it largely on small datasets.}
 \label{fig:mse_loss}
\end{figure*}

% Overview of the key research questions we are looking to answer.
In the previous sections we have described our framework for visual imitation, VINN. 
In this section, we seek to answer our key question: how well does VINN imitate human demonstrations? 
To answer this question, we will evaluate both on offline datasets and in closed-loop real-robot evaluation settings.
Additionally, we will probe into the generalization with few demonstrations ability of VINN in settings where imitation algorithms usually suffer.

\subsection{Experimental Setup}\label{sec:exp_setup}
We conduct two different set of experiments: the first on the offline datasets for Pushing, Stacking and Door-Opening and the second on real-robot door opening.

\paragraph{Offline Visual Imitation Datasets} Data for Pushing 
and Stacking tasks are taken from \cite{young2020visual}. 
% We use \lpnote{Jyo} examples from the dataset provided in and use \lpnote{Jyo} of them in the training set.
The goal in the pushing task is to slide an object on a surface into a red circle.
In the stacking task, the goal is to grasp an object present in the scene and move it on top of another object also in the scene, and release. To avoid confusion, in the expert demonstrations for stacking, the closest object is always placed on top of the distant object. 
The action labels are end-effector movements, which in this case is the translation vector in between the current frame and the subsequent one. 
In each case, there are a diverse set of backgrounds and objects that make up the scene and the task, making the tasks difficult.

For Door Opening, data is collected by 3 data-collectors in their kitchens. 
This amounts to a total of 71 demonstrations for training and 21 demonstrations for testing. 
We normalize all actions from the dataset to account for scale ambiguity from SfM. 
For all three tasks, we calculate the MSE loss between the ground truth actions and the actions predicted by each of the methods. 
Note that the number of demonstrations collected for this Door Opening task is an order of magnitude smaller than the ones used for Stacking and Pushing, which contain around 750 and 930 demonstrations respectively.
To understand the performance on the various model in low data settings, we create subsampled Pushing and Stacking datasets containing 71 demonstrations on each for training and 21 for testing. This subsampling makes all three our datasets have the same size. 

\paragraph{Closed-loop control} We conduct our robot experiments on a loaded cabinet door opening task (see Fig.~\ref{fig:intro}), where the goal of the robot is to grab hold of the cabinet handle and pull open the cabinet door. 
We use the Hello-Robot Stretch~\cite{kemp2021design} for this experiment.  
% For our demonstrations, we followed the DemoAt \cite{young2020visual} method of attaching a GoPro to a Reacher-Grabber, and collecting demonstrations with it. 
% We extracted the ground truth actions from the GoPro metadata.
When evaluations start, the arm resets to $\approx0.15$ meters away from the cabinet door, with a random lateral translation within $0.05$ meters parallel to the cabinet to evaluate generalization to varying starting states.

\subsection{Baselines}
We run our experiments for baseline comparison using the following methods:
\begin{itemize}
    \item \textit{Random Action:} In this baseline, we sample a random action from the action space.
    \item \textit{Open Loop:} We find the maximum-likelihood open loop policy given all our demonstration, which is the average action $\overline a(t)$ over all actions $a_i(t)$ seen in the dataset at timestep $t$. In a Bayesian sense, if standard behavioral cloning is trying to approximate $p(a \mid s)$, this model is trying to approximate $p(a \mid t)$.
    \item \textit{Behavioral Cloning (BC) end to end:} We train a ResNet-50 model with augmentated demonstration frames similar to \cite{torabi2018behavioral,young2020visual}. We initialize the model with  weights derived from ImageNet pretraining.
    \item \textit{BC on Representations (BC-rep):} We use a self-supervised BYOL model to extract the encoding of each of our demonstration frames, and perform behavioral cloning on top of the representations.
    This baseline is similar to \cite{young2021playful} and performs better than end-to-end BC on the real robot (Table~\ref{table:real_robot}).
    \item \textit{Implicit Behavioral Cloning:} We train Implicit BC \cite{florence2021implicit} models on the tasks, modifying the official code. 
    \item \textit{ImageNet features + NN:} Instead of self-supervision, here we use the image representation generated by a pretrained ImageNet encoder akin to \cite{chen2020robust}. The difference between this baseline and our method is simply forgoing the finetuning step on our dataset.
    This baseline highlights the importance of self-supervised pre-training on the domain related dataset.
    
    \item \textit{Self-supervised learning method + NN:} This is our method; we compare three different ways of learning self-supervised representations features from our dataset -- BYOL~\cite{byol}, SimCLR~\cite{simclr}, and VICReg~\cite{bardes2021vicreg}, starting from an ImageNet pretrained ResNet-50, and then we use locally weighted regression to find the action.
    % However, the official implementation requires the action space to be products of real intervals, while in our tasks are normalized actions on the unit 2-sphere $S^2$, which is why IBC suffers in these tasks.
    % \item \textit{ImageNet Features + NN:} We use a similar approach as BC-rep, but instead of using representations trained with BYOL, we use representations generated by a ResNet50 pretrained on ImageNet only. 
    % \item \textit{Random-NN:} Similar to ImageNet-NN, we consider a randomly initialized ResNet-50 model for this baseline, get encodings from it, and get an action by performing a locally weighted $k$-NN averaging like VINN. This baseline, once again, highlights the importance of the pre-training.
    % \item \textit{BC-Random:} We use a similar approach as BC-rep, but instead of using representations from BYOL, we use representations generated by a ResNet50 pretrained on ImageNet only.
\end{itemize}

\subsection{Training Details}
Each encoder network used in this paper follows the ResNet-50 architecture~\cite{he2015deep} with the final linear layer removed. 
Unless specified otherwise, we always initialize the weights of the ResNet-50 encoder with a pretrained model on ImageNet dataset.
For VINN, we train our self-supervised encodings with the BYOL~\cite{byol} loss.
For standard end-to-end BC, we replace the last linear layer with a three-layer MLP and train it with the MSE loss.
For BC-rep, we freeze the encoding network to the weights trained by BYOL on our dataset, and train just the final layers with the MSE loss.
Additionally, for all visual learning, we use random crop, random color jitter, random grayscale augmentations and random blurring. 
We trained the self-supervised finetuning methods for 100 epochs on all three datasets. 
% What type of networks used? information to reproduce this learning portion of this paper.

% \subsection{Can VINN Compete with Standard Behavior Cloning?}
% % Experimental results on Push MSE, Stack MSE and Door MSE.
% \begin{figure}
%     \centering
%     \includegraphics[width=\linewidth]{figures/comparison_push.pdf}
%       \vspace{-0.2in}
%     % \caption{Caption}
%     %   \vspace{-0.2in}
%     \includegraphics[width=\linewidth]{figures/comparison_stack.pdf}
%       \vspace{-0.2in}
%     \includegraphics[width=\linewidth]{figures/comparison_handle.pdf}
%       \vspace{-0.1in}
%     \caption{Comparison of the performance of three candidate algorithms (BC on representations, ImageNet pretraining features + NN, and VINN (which is BYOL-fine-tuned features + NN) on the stack, push and handle datasets as the size of the dataset changes.}
%       \vspace{-0.2in}
%     \label{fig:scaling_data}
% \end{figure}
% \begin{figure}
%     \centering

% \end{figure}

\subsection{How does VINN Perform on Offline Datasets?}
For our first evaluation, we compare our method against the baselines on their Mean-Squared Error loss for the Pushing, Stacking, and Door-Opening tasks in Fig.~\ref{fig:mse_loss}. To understand the impact of the training dataset size on the algorithms, we train models on multiple subsamples of different sizes from each dataset. We see that while end-to-end Behavioral Cloning starting from pretrained ImageNet representations can be better with a large amounts of training demonstrations, Nearest Neighbor methods are either competitive or better performing in low data settings.

On the Stacking and Door-Opening tasks, VINN is significantly better when the number of training demonstrations are small ($<20$). While on the Pushing task, we notice that the task might be too difficult to solve with small number of demonstrations. One reason for this is that BYOL might not be able to extract the most relevant representations for this task. Further experiments in Table~\ref{tab:mse-table} show that using other forms of self-supervision such as VICReg can significantly improve performance on this task. Overall, these experiments supports our hypothesis that provided with good representations, nearest-neighbor techniques can provide a competitive alternative to end-to-end behavior cloning.

% it remains on par with the baselines. 
% As we can see, VINN achieves comparable MSE loss in the test set in Pushing, Stacking, and Door Opening. Thus, in simple offline datasets, the predicted action from VINN is similarly close to the ground truth as BC and BC variants. \lpnote{this statement may not be true.}

\subsection{How does VINN Perform on Robotic Evaluation?}

\begin{figure*}[t]
  \begin{center}
    \includegraphics[width = \textwidth]{figures/VINN_generalization.png}
  \end{center}
  \caption{Sample frames from the rollouts from our model on the real robot experiments, with artificial occlusions added to the cabinet to test generalization. Under the maximum occlusion, our model fails to ever open the cabinet door, while in all other cases, the robot is able to succeed (Table~\ref{table:generalization}.)}
\label{fig:rollouts}
\end{figure*}

% Experimental results on door opening without variations.
Next, we run VINN and the baselines on our real robot environment. 
In this setting, our test environment comprises of the same three cabinets where training demonstrations were collected presented without any visual modifications.
For each of our models, we run 30 rollouts with the robot in the real world with three different cabinets.
On each rollout, the starting position of the robot is randomized as detailed in (Sec.~\ref{sec:exp_setup}).
In Table~\ref{table:real_robot}, we show the percentage of success from the 30 rollouts of each model, where we record both the number of time the robot successfully grasped the handle, as well as the number of time it fully opened the door.

\begin{table}[!ht]
\centering
\caption{Success rate over 30 trials (10 trials on three cabinets each) on the robotic door opening task.}
\label{table:real_robot}
\begin{tabular}{@{}ccc@{}}
\toprule
Method                  & Handle grasped & Door opened   \\ \midrule
BC (end to end)        & 0\%            & 0\%           \\
BC on representations  & 56.7\%         & 53.3\%        \\
Imagenet features + NN & 20\%           & 0\%           \\
VINN (BYOL + NN)       & \textbf{80}\%  & \textbf{80}\% \\ \bottomrule
\end{tabular}
\end{table}

As we see from Table~\ref{table:real_robot}, VINN does better than all BC variants in successfully opening the cabinet door when there is minimal difference between the test and the train environments. 
Noticeably, it shows that depending on self-supervised features on augmented data make the models much more robust.
% We believe this success is due such models being more robust to the out of distribution queries, since the door opening task is multi-step and long horizon. 
BC, as an end-to-end parameteric model, does not have a strong prior on the actions if the robot makes a wrong move causing the visual observations to quickly goes out-of-distribution~\cite{dagger2010}. 
% However, pretrained representation based models do not overfit much to the combination of visual feature and the task objective.
% This lack of co-optimization between visual features and policy goals enforces a bottleneck that results in better visual representations in runtime.
On the other hand, VINN can recover up to certain degree of deviation using the nearest neighbor prior, since the translation actions typically tend to re-center the robot instead of pushing it further out of distribution. 
% \lpnote{Explanations in this section are quite hand-wavy. Mahi, can you edit?}

\subsection{To What Extent does VINN Generalize to Novel Scenes?}\label{sec:generalize}
% Experiments on generalization. Where does it work and where does it not.

To test generalization of our robot algorithms to novel scenes in the real world, we modified one of our test cabinets with various levels of occlusion. We show frames from a sample rollouts in each environment in Fig.~\ref{fig:rollouts}, which also shows the cabinet modifications.

\begin{table}[ht!]
\centering
\caption{Success rate over 10 trials on robotic door opening with visual modifications on one cabinet door.}
\label{table:generalization}
\begin{tabular}{@{}ccc@{}}
\toprule
Modification                               & BC-rep        & VINN (ours)    \\ \midrule
Baseline (no modifications)          & \textbf{90}\% & 80\%           \\
Covered signs and handle             & 10\%          & \textbf{70\% } \\
Covered signs, handle, and one bin   & 0\%           & \textbf{50\%}  \\
Covered signs, handle, and both bins & 0\%           & 0\%            \\ \bottomrule
\end{tabular}
\end{table}

In Table~\ref{table:generalization}, we see that VINN only completely fails when all the visual landscape on the cabinet is occluded. This failure is expected, because without coherent visual markers, the encoder fails to convey information, and thus the k-NN part also fails.
Even then, we see that VINN succeeds at a higher rate even with significant modifications to the cabinet while BC-rep fails completely. 

Over all the real robot experiments, we find the following phenomenon: while a good MSE loss is not sufficient for a good performance in the real world, the two are still correlated, and a low MSE loss seems to be necessary for good real world performance.
This observation let us test hypotheses offline before deploying and testing them in a real robot, which can be time-consuming and expensive.
We hypothesize that this gap between performance on the MSE metric (Table~\ref{tab:mse-table}) and real world performance (Table~\ref{table:real_robot}, ~\ref{table:generalization}) comes from variability in different models' ability to perform well in situations off the training manifold, where they may need to correct previous errors.
% Interestingly, although the MSE metric does not directly correspond to online performance, our experiments in Table~\ref{fig:mse_loss} suggest strong correlation. Specifically, methods that have high MSE on online tasks, mainly those that have no pretraining for representations, yield poor online performance on the Door Opening task. 

\begin{table*}[!bhtp]
\caption{\label{table:mse_loss} Test MSE $(\times 10^{-1})$ on predicted actions for a set of baseline methods and ablations. Standard deviations, when reported, are over three randomly initialized runs.}
\label{tab:mse-table}
% \resizebox{\textwidth}{!}{%
\centering
\begin{tabular}{@{}cccccccccc@{}}
\toprule
             &        &                                                     & \multicolumn{2}{c}{No Pretraining}                & \multicolumn{5}{c}{With ImageNet Pretraining}                                                                                                                                                                                                      \\ \cmidrule(l){4-5}\cmidrule(l){6-10} 
Tasks        & Random & \begin{tabular}[c]{@{}c@{}}Open\\ Loop\end{tabular} & \begin{tabular}[c]{@{}c@{}}Implicit\\ BC\end{tabular}   & \begin{tabular}[c]{@{}c@{}}BYOL\\ + NN\end{tabular} & BC-Rep          & \begin{tabular}[c]{@{}c@{}}VINN\\(BYOL + NN)\end{tabular} & \begin{tabular}[c]{@{}c@{}}VICREG\\ + NN\end{tabular} & \begin{tabular}[c]{@{}c@{}}SimCLR\\ + NN\end{tabular} & \begin{tabular}[c]{@{}c@{}}ImageNet\\ + NN\end{tabular} \\ \midrule
Door Opening & $6.34$ & $2.27$                                              & $1.8$ & $1.52$                                              & $1.19 \pm 0.05$ & $0.92$                                              & $1.05$                                                & $0.95$                                                & $0.98$                                                  \\
Stacking     & $6.13$ & $2.83$                                              & $7.1$ & $2.82$                                              & $3.45 \pm 0.29$ & $2.58$                                              & $2.74$                                                & $2.63$                                                & $2.85$                                                  \\
Pushing      & $6.15$ & $2.12$                                              & $5.6$ & $2.43$                                              & $2.20 \pm 0.20$ & $2.43$                                              & $1.50$                                                & $2.21$                                                & $2.35$                                                  \\ \bottomrule
\end{tabular}
% }
\end{table*}

\subsection{How Important are the Design Choices Made in VINN for Success?}
% In this section, we examine some of the design choices made for VINN and empirically show their impact compared to possible alter-tives. 

% \lpnote{This section needs better organization. Currently it reads like a block of text.} 
VINN comprises of two primary components, the visual encoder and the nearest-neighbor based action modules.
In this section, we consider some major design choices that we made for each of them. 

\paragraph{Choosing the Right Self-supervision} While we use a BYOL-based self-supervised encoding in our algorithm, there are multiple other self-supervised methods such as SimCLR and VICReg ~\cite{simclr,bardes2021vicreg}. On a small set of experiments we noticed similar MSE losses compared to SimCLR~\cite{simclr} and VICReg~\cite{bardes2021vicreg}. From Table~\ref{table:mse_loss}, we see that BYOL does the best in Door-Opening and Stacking, while VICReg does better in Pushing. However, we choose BYOL for our robot experiments since it requires less tuning overall. 
% \lpnote{On Pushing VICReg does better. I think it is good to highlight that k-NN depends on the quality of representations, and if it is good, it will do well.}

\paragraph{Ablating Pretraining and Fine-tuning} Another large gain in our algorithm is achieved by initializing our visual encoders with a network trained on ImageNet. In Table~\ref{table:mse_loss}, we also show MSE losses from models that resulted from ablating this components of VINN. Removing this component achieves the column BYOL + NN (No Pretraining), which performs much worse than VINN. 
Similarly, the success of VINN depends on the self-supervised fine-tuning on our dataset, ablating which results in the model shown in ImageNet + NN column of Table~\ref{table:mse_loss}. 
This model performs only slightly worse than VINN on the MSE metric.
However, in Table~\ref{table:real_robot}, we see that this model performs poorly on the real world.
These ablations show that the performance of our locally weighted regression based policy depends on the quality of the representation, where a good representation leads to better nearest neighbors, which in turn lead to a better policy both offline and online.

\paragraph{Performing Implicit instead of Explicit Imitation}
Moving away from the explicit forms of imitation where the models try to predict the actions directly, we run baselines with Implicit Behavioral Cloning (IBC)~\cite{florence2021implicit}. 
As we see on Table~\ref{table:mse_loss}, this baseline fails to learn behaviors significantly better than the random or open loop baselines. We believe this is caused by two reasons. First, the implicit models have to model the energy for the full space (action space $\times$ observation space), which requires more data than the few demonstrations that we have in our datasets. Second, the official implementation of IBC supports $[-1, 1]^3$ as the action space instead of its much smaller subspace of normalized 3d vectors $S^2$. This much larger action space, over which IBC tried to model the action, might have resulted in worse performance for IBC. While VINN makes the implicit assumption that the locally-weighted average of valid actions also yield a valid action, it can be freely projected to any relevant space without further processing, which makes it more flexible.

\paragraph{Learning a Parametric Policy on Representations}
Our Behavioral Cloning on representations (BC-Rep) baseline in all our experiments (Sec.~\ref{sec:experiments}) show the performance of a baseline where we use learned representations to learn a parametric behavioral policy. In the MSE losses (Table~\ref{tab:mse-table}) and real world experiments (Table~\ref{table:real_robot},~\ref{table:generalization}.) This is the baseline that achieves the closest performance to VINN. However, the difference between BC-rep and VINN becomes more pronounced as the gap between training and test domain or the policy horizon grows. These experimental results indicate that using a non-parametric policy may be enabling us to be robust to out-of-distribution samples.

\begin{figure}[ht]
  \begin{center}
    \includegraphics[width = \linewidth]{figures/k_graph.pdf}
  \end{center}
%   \vspace{-0.2in}
  \caption{Value of $k$ in the $k$-nearest neighbor weighted regression in VINN vs normalized MSE loss achieved by the model.}
%   \vspace{-0.275in}
\label{fig:knn_ablation}
\end{figure}

\paragraph{Choosing the Right $k$ for $k$-Nearest Neighbors} Finally, in VINN, we study the effect of different values of $k$ for the $k$-NN based locally weighted controller. This parameter is important because with too small of a $k$, the predicted action may stop being smooth. On the other hand, with too large of a $k$, unrelated examples may start influencing the predicted action. By plotting our model's normalized MSE loss in the validation set against the value of $k$ in Fig.~\ref{fig:knn_ablation}, we find that around $10$, $k$ seems ideal for achieving low validation loss while averaging over only a few actions. 
Beyond $k=20$, we didn't notice any significant improvement to our model from increasing $k$.

\subsection{Computational Considerations}
While the datasets we used for our experiments were not large, we recognize that our current nearest neighbor implementation is a $O(n)$ algorithm dependant linearly on the size of the training dataset with a naive algorithm. 
However, we believe VINN to be practical, since firstly, it was designed mostly for the small demonstration dataset regime where $O(n)$ is quite small, 
and secondly, this search can be sped up with a compiled index beyond the naive method using open-source libraries such as FAISS~\cite{faiss} which were optimized to run nearest neighbor search on the order of billion examples~\cite{matsui2018survey}. 
Currently, our algorithm takes $\approx 0.074$ seconds to encode an image, and $\approx 0.038$ seconds to perform nearest neighbors regression, which is only a small speed penalty for the robotic tasks we consider.
% \lpnote{Can we say that more efficient methods do exist for faster retrieval? and cite them as well.}

%%%%%%%%%%%%%%%%%%%%%%%%%%%%%%%%%%%%%%%%%%%%%%%%%%%%%%%%%%%%%%%%%%%%%%%%%%%%%%%%
\section{Conclusion}
In this paper, we present \methodname{}, a unified framework for vision–language–action modeling that bridges heterogeneous modalities through a shared token space and models them autoregressively. The proposed unified design facilitates deeper cross-modal integration and inherently supports flexible multimodal tasks. By leveraging a world model trained to capture dynamics and causality from videos, we observe significant improvements in downstream policy learning, both in terms of performance and efficiency. Extensive simulation experiments further demonstrate the model’s strong generalization ability, efficient policy learning, and broad applicability across diverse domains.
These findings highlight the great potential of our method as a new paradigm for vision–language–action modeling.
\vspace{-2mm}
\paragraph{Limitations and Future Work.}
Due to limited computational resources, our investigation into post-training scalability is still in its early stages. Nonetheless, initial results are promising and indicate potential for scaling to larger video datasets. Furthermore, while the unified multimodal framework exhibits strong capabilities in cross-modal learning, further research is needed to fully integrate it with reinforcement learning paradigms, enabling more robust and adaptive policy learning.

%%%%%%%%%%%%%%%%%%%%%%%%%%%%%%%%%%%%%%%%%%%%%%%%%%%%%%%%%%%%%%%%%%%%%%%%%%%%%%%%
\section*{Acknowledgements}
We thank Rohith Mukku for his help with writing code and running ablation experiments. We thank Dhiraj Gandhi, Pete Florence, and Soumith Chintala for providing feedback on an early version of this paper. This work was supported by grants from Honda, Amazon, and ONR award numbers N00014-21-1-2404 and N00014-21-1-2758.

%%%%%%%%%%%%%%%%%%%%%%%%%%%%%%%%%%%%%%%%%%%%%%%%%%%%%%%%%%%%%%%%%%%%%%%%%%%%%%%%
% \section*{ACKNOWLEDGEMENTS}

%% Use plainnat to work nicely with natbib. 

\bibliographystyle{plainnat}
\bibliography{references}

\clearpage

\setcounter{table}{0}
\renewcommand{\thetable}{A\arabic{table}}
\setcounter{figure}{0}
\renewcommand{\thefigure}{A\arabic{figure}}
\renewcommand\theHtable{Appendix.\thetable}


\section{Text Prompts}
Below we listed the text prompts we used for adaptation and task evaluation.

\begin{table}[h]
\centering
\small

\setlength{\tabcolsep}{4.8pt}
\scalebox{0.8}{

\begin{tabular}{@{}llll@{}}
\toprule
Task             & In-Domain Prompts                         & AnimateDiff Prompts                                        & DreamBooth Identifier                 \\ \midrule
Dog Walking      & a dog/pharaoh hound walking          & a dog/pharaoh hound walking                                & a {[}D{]} dog                         \\
Humanoid Walking & a(n) humanoid/action figure  walking & a(n) humanoid/action figure walking                        & a {[}D{]} action figure               \\
\midrule
Assembly$^*$     & assembly                         & a robot arm placing a ring over a peg                               & \multirow{14}{*}{a {[}D{]} robot arm} \\
Dial Turn$^*$     & dial turn                         & a robot arm turning a dial                               &                                       \\
Reach$^*$     & reach                         & a robot arm reaching a red sphere                               &                                       \\
Peg Unplug Side$^*$     & peg unplug side                         & a robot arm unplugging a gray peg                               &                                       \\
Lever Pull$^*$       & lever pull                           & a robot arm pulling a lever                                &                                       \\ 
Coffee Push$^*$      & coffee push                          & a robot arm pushing a white cup towards a coffee machine &                                       \\
Door Close$^*$       & door close                           & a robot arm closing a door                                 &                                   \\
Door Open        & door open                            & a robot arm opening a door                                 &                                       \\
Window Close     & window close                         & a robot arm closing a window                               &                                       \\
Window Open      & window open                          & a robot arm opening a window                               &                                       \\
Drawer Close     & drawer close                         & a robot arm closing a drawer                               &                                       \\
Drawer Open      & drawer open                          & a robot arm open a drawer                                  &                                       \\
Soccer           & soccer                               & a robot arm pushing a soccer ball into the net             &                                       \\
Button Press     & button press                         & a robot arm pushing a button                               &                                       \\\bottomrule
\end{tabular}
}

\caption{\textbf{Task-Prompt Pairs.} We include a comprehensive list of tasks and their text prompts for adaptation and evaluation. ``$*$'' denotes tasks seen during adaptation.}
\label{table:text_prompts}
\end{table}

\begin{table}[h]
\centering
\small

\setlength{\tabcolsep}{4.8pt}

\begin{tabular}{@{}lll@{}}
\toprule
Task             & In-Domain Prompts                         & AnimateDiff Prompts           \\ \midrule
Spatula in Kitchen$^{*}$      & spatula                      & find the spatula              \\
Toaster in Kitchen$^{*}$       & toaster                   & find the toaster                    \\
Painting in Living Room$^*$     & painting                    & find the painting             \\
Blinds in Bedroom$^*$           & blinds                         & find the blinds             \\
ToiletPaper in Bathroom$^*$     & toilet paper                         & find the toilet paper   \\
Pillow in Living Room          & pillow                         & find the pillow           \\
DeskLamp in Living Room       & desk lamp                           & find the desk lamp     \\ 
Mirror in Bedroom               & mirror                          & find the mirror         \\
Laptop in Bedroom             & laptop                           & find the laptop           \\
\bottomrule
\end{tabular}


\caption{\textbf{Task-Prompt Pairs for iTHOR.} We include a comprehensive list of iTHOR tasks and their text prompts for adaptation and evaluation. ``$*$'' denotes tasks seen during adaptation.}
\label{table:text_prompts_ithor}
\end{table}








\section{Continued Denoising}
\label{sec:cont_denoising}

In diffusion-based policy supervision, rewards are extracted from the procedure of corrupting frames achieved by the policy with some level of Gaussian noise and then making denoising predictions using the video model~\citep{huang2023diffusion, luo2024text}.  For additional insight, we propose a visualization technique called \textbf{\textit{continued denoising}}, and report FVD scores for videos generated in such a manner.  In continued denoising, rather than extracting a scalar from components of the denoising prediction as in Video-TADPoLe, we treat the noised video as an initialization and iteratively continue sampling to produce a final clean video prediction - thus, “continuing” the denoising procedure.  In our experiments we perform continued denoising conditioned on a desired text prompt, a noise level of 700, a total frame length of 16, and 10 denoising steps.

As mentioned in Section~\ref{subsec:policy_supervision_method}, policy supervision does not necessarily require strong free-form generation of in-domain videos; rather it evaluates observed frames achieved by following the current policy.  For qualitative purposes, continued denoising provides us a visual sense of how this evaluation of achieved frames is done (examples in Figure~\ref{fig:mw_continued_denoising_400_unseen}), as well as a sanity check on the integration of in-domain information through adaptation.  Furthermore, it enables quantitative comparison through FVD scores, which provides an idea on the capability of adapted video models to reconstruct in-domain-like videos conditioned on text.  It is intuitive to hypothesize that a lower FVD score correlates with better in-domain adaptation, as it understands how to accurately complete the provided in-domain frames from a heavy noise corruption.

In Table~\ref{table:fvd_scores_contd}, we report the FVD scores for the same set of seen and unseen tasks that are evaluated in free-form generation experiments. 
We discover that the lowest FVD score for continued denoising is achieved by the in-domain model, which is unsurprising as it was explicitly trained on such examples.  The next-best FVD scores are achieved by probabilistic adaptation and its inverse.  This is significant because it supports the finding that with adaptation, generalization to unseen tasks is possible, and suggests that accurate domain-specific rewards can be supplied through policy supervision.  Indeed, this aligns with our result in Table~\ref{table:mw_videotadpole}, where Inverse Probabilistic Adaptation achieves the best overall task performance through policy supervision. 

\begin{table}[h]
\centering
\setlength{\tabcolsep}{4.8pt}
\resizebox{\textwidth}{!}{
\begin{tabular}{@{}lcccccc@{}}
\toprule
FVD Scores (MetaWorld) & Vanilla AnimateDiff & In-Domain-Only & Direct Finetuning & Subject Customization & Prob. Adaptation & Inverse Prob. Adaptation \\ \midrule
Seen                   & 2700.4              & \textbf{602.8}          & 1004.6            & 1078.9                & 622.6            & 627.4                    \\
Unseen                 & 2643.2              & \textbf{610.1}          & 978.5             & 1711.8                & 630.6            & 681.8                    \\ \bottomrule
\end{tabular}
}
\caption[]{\textbf{FVD Scores with Continued Denoising.} We report FVD scores for videos of MetaWorld tasks, produced by Continued Denoising via the video generative models of interest. This is computed for both seen and unseen task sets, each with 7 tasks, aggregating results over 1000 synthetic videos.}
\label{table:fvd_scores_contd}
\end{table}

\begin{figure}[h]
    \centering
    \includegraphics[width=\linewidth]{figures/mw_continued_denoising_400_unseen.pdf}
    \vspace{-15pt}
    \caption{\textbf{Continued Denoising.}  We visualize frames from a task unseen during adaptation, corrupted with a level of Gaussian noise (top row).  We then show the result of continued denoising using an inverse probabilistic adaptation model to verify it can visually generalize to fill in novel in-domain information.  Despite not having seen a button, it is able to reconstruct it conditioned on text.  This figure is for intuition; in practice, a much higher noise level is used, shown in Figure~\ref{fig:mw_continued_denoising_700_unseen}.}
    \label{fig:mw_continued_denoising_400_unseen}
    \vspace{-10pt}
\end{figure}


\begin{figure}[h]
    \centering
    \includegraphics[width=\linewidth]{figures/mw_continued_denoising_700_unseen.pdf}
    \vspace{-1em}
    \caption{\textbf{Continued Denoising (in practice).}  In practice, an aggressive level of Gaussian corruption is usually used on achieved frames for reward computation (700 for MetaWorld).  However, because to the human eye this may look virtually indistinguishable from pure noise, we supply an illustrative example in Figure~\ref{fig:mw_continued_denoising_400_unseen} using a noise level of 400.  Here, we showcase visuals of the same unseen task corrupted with a practical noise level of 700.  We then show the result of continued denoising to visually verify the model integrates adapted in-domain information successfully.  When performing continued denoising from such a high corruption, conditioned on the text prompt ``a robot arm pushing a button”, it is therefore quite surprising the level of detail with which the adapted text-to-video model is able to reconstruct novel in-domain features such as the button - which it has not even seen during adaptation.  The resulting continued denoising video can also be evaluated against in-domain examples via FVD for further insights.}
    \label{fig:mw_continued_denoising_700_unseen}
    \vspace{-1em}
\end{figure}

\section{Video-TADPoLe Reward Computation}
\label{sec:videotadpole_equations}

Video-TADPoLe~\citep{luo2024text} rewards are densely computed for a trajectory achieved by a policy, in terms of their rendered frames.  For arbitrary start index $i$ and end index $j$ inclusive of the trajectory, for $i \leq j$, let $\mathbf{o}_{[i+1:j+1]}$  denote the associated sequence of rendered frames.  Video-TADPoLe then utilizes a source noise vector $\boldsymbol{\epsilon}_0 \sim \mathcal{N}(\boldsymbol{\epsilon};\boldsymbol{0}, \textbf{I}_{j-i+1})$ of the same dimensionality as a Gaussian corruption to produce noisy observation $\tilde{\mathbf{o}}_{[i+1:j+1]}$.  Then, Video-TADPoLe computes a batch of \textit{alignment reward} terms through one inference step of the text-to-video diffusion model as:
$$r_{[i:j]}^\text{align} = \left\lVert\boldsymbol{\hat{\epsilon}}_{\boldsymbol{\phi}}(\tilde{\boldsymbol{o}}_{[i+1:j+1]}; \texttt{t}_\text{noise}, y) - \boldsymbol{\hat{\epsilon}}_{\boldsymbol{\phi}}(\tilde{\boldsymbol{o}}_{[i+1:j+1]}; \texttt{t}_\text{noise})\right\rVert_2^2,$$
and a batch of \textit{reconstruction reward} terms as:
$$r_{[i:j]}^\text{rec} = \left\lVert\boldsymbol{\hat{\epsilon}}_{\boldsymbol{\phi}}(\tilde{\boldsymbol{o}}_{[i+1:j+1]}; \texttt{t}_\text{noise}) - \boldsymbol{\epsilon}_0\right\rVert_2^2 - \left\lVert\boldsymbol{\hat{\epsilon}}_{\boldsymbol{\phi}}(\tilde{\boldsymbol{o}}_{[i+1:j+1]}; \texttt{t}_\text{noise}, y) - \boldsymbol{\epsilon}_0\right\rVert_2^2.$$
For a provided context window of size $n$, Video-TADPoLe calculates the reward at each timestep $t$ utilizing each context window that involves achieved observation $\mathbf{o}_{t+1}$:
$$r_t = \frac{1}{n}\sum_{i = 1}^{n}\texttt{symlog}\left(w_1*r_{[t-i+1:t-i+n]}^\text{align}[i-1]\right) + \texttt{symlog}\left(w_2 * r_{[t-i+1:t-i+n]}^\text{rec}[i-1]\right).$$
A stride term $s$ can be used to make this computation tractable across long trajectories, where the context window skips $s$ timesteps before computing a sequence of Video-TADPoLe rewards again.  The context window $n$, stride $s$, and noise level $\texttt{t}_\text{noise}$ are hyperparameters to be set by the user; in practice, good settings for such hyperparameters can be found in an offline manner through \textit{policy discrimination} (Section~\ref{sec:policy_discrimination}).











\section{Implementation Details}
\label{sec:implementation_details}


We include the default hyperparameters from the TD-MPC implementation in Table~\ref{tab:tdmpc-hparams} for completeness.  We do not modify the default recommended settings for both Humanoid and Dog environments, as well as the Meta-World experiments.








\begin{table}[h]
\centering
\small
\begin{tabular}{@{}ll@{}}
\toprule
Hyperparameter     & Value                           \\ \midrule
Training Objective       &   \texttt{pred\_noise}           \\
Number of Training Steps &  60000                            \\
Loss Type                & L2                                \\
Learning Rate            &  1e-4                              \\
Beta Schedule            &  Linear schedule (0.0085, 0.012)   \\
Timesteps                &  1000                              \\
EMA Decay                &  0.99                              \\
EMA Update Steps         &  10                                \\ 
\bottomrule
\end{tabular}


\caption[]{\textbf{Hyperparameters for In-Domain Model Training.} }
\label{table:hparams_in_domain_model_training}
\end{table}

\begin{table}[]
\centering
\begin{tabular}{@{}lcc@{}}
\toprule
Noise Level              & Humanoid Walking & Dog Walking \\ \midrule
In-Domain Only           & 600              & 600         \\
Direct Finetuning        & 700              & 700         \\
Subject Customization    & 500              & 600         \\
Prob. Adaptation         & 700              & 700         \\
Inverse Prob. Adaptation & 600              & 500         \\ \bottomrule
\end{tabular}
\caption[]{\textbf{VideoTADPoLe Noise Levels for DeepMind Control.}}
\label{table:videotadpole_noise_level_dmc}
\end{table}


\textbf{Visual Planning Hyperparameters:} To generate a video plan with adapted video models, we perform DDIM~\citep{song2021ddim} sampling for 25 steps. We use 7.5 as the text-conditioning guidance scale for directly finetuned AnimateDiff, and use 2.5 for other adaptation techniques. Additionally, we use 0.1 as the prior strength for probabilistic adaptaion and 0.5 for its inverse version.


\textbf{Inverse Dynamics:} We employ a small MLP network as our inverse dynamics model. The model takes in the embeddings of two consecutive video frames, which are extracted using VC-1~\citep{majumdar2023vc1}, and predicts the action that enables the transition between the provided frames. We train the inverse dynamics model on a dataset comprising a mixture of expert and suboptimal trajectories rendered from the environment, using the same set of tasks and data volumn as used for adaptation. For fairness, we reuse the same dynamics model across all adaptation techniques during evaluation. We provide the detailed hyperparameters of inverse dynamics training in Table~\ref{table:inv_dyn_hparams}.


\begin{table}[h]
\centering
\small

\begin{tabular}{@{}ll@{}}
\toprule
Hyperparameter    & Value \\ \midrule
Input Dimension  & 1536      \\
Output Dimension & 4      \\
Training Epochs  & 20      \\
Learning Rate    & 3e-5      \\
Optimizer        & AdamW     \\ \bottomrule
\end{tabular}

\caption{\textbf{Hyperparamters of Inverse Dynamics Model Training}}
\label{table:inv_dyn_hparams}
\end{table}


\begin{table}[h!]
\centering
\begin{minipage}{0.48\textwidth}
\label{tab:sd-hparams}
\vspace{0.05in}
\centering
\begin{tabular}{@{}ll@{}}
\toprule
Component   & \# Parameters (Millions) \\
\midrule
VAE (Encoder) & 34.16 \\
VAE (Decoder) & 49.49 \\
U-Net & 865.91 \\
Text Encoder & 340.39 \\
\bottomrule
\end{tabular}%

\caption{\textbf{StableDiffusion Components.} For completeness, we list sizes of the components of the StableDiffusion v2.1 checkpoint used in Video-TADPoLe experiments. The checkpoint is used purely for inference, and is not modified or updated in any way. Note that the VAE Decoder is not utilized in our framework.}

\hfill

\vspace{0.45in}


\label{tab:ad-hparams}
\vspace{0.05in}
\centering
\begin{tabular}{@{}ll@{}}
\toprule
Component   & \# Parameters (Millions) \\
\midrule
VAE (Encoder) & 34.16 \\
VAE (Decoder) & 49.49 \\
U-Net & 1312.73 \\%865.91 \\
Text Encoder & 123.06 \\
\bottomrule
\end{tabular}%
\caption{\textbf{AnimateDiff Components.} For completeness, we list sizes of the components of the AnimateDiff checkpoint used in Video-TADPoLe experiments.  The checkpoint is used purely for inference, and is not modified or updated in any way. Note that the VAE Decoder is not utilized in our framework.}
\end{minipage}\hfill
\begin{minipage}{0.48\textwidth}
\vspace{0.05in}
\centering
\resizebox{\linewidth}{!}{%
\begin{tabular}{@{}ll@{}}
\toprule
Hyperparameter   & Value \\
\midrule
Discount factor ($\gamma$) & 0.99 \\
Seed steps & $5,000$ \\
Replay buffer size & Unlimited \\
Sampling technique & PER ($\alpha=0.6, \beta=0.4$) \\
Planning horizon ($H$) & $5$ \\
Initial parameters ($\mu^{0}, \sigma^{0}$) & $(0, 2)$ \\
Population size & $512$ \\
Elite fraction & $64$ \\
Iterations & 12 (Humanoid)\\
 & 8 (Dog)\\
Policy fraction & $5\%$ \\
Number of particles & $1$ \\
Momentum coefficient & $0.1$ \\
Temperature ($\tau$) & $0.5$ \\
MLP hidden size & $512$ \\
MLP activation & ELU \\
Latent dimension & 100 (Humanoid, Dog) \\
Learning rate & 3e-4 (Dog)\\
 & 1e-3 (Humanoid) \\
Optimizer ($\theta$) & Adam ($\beta_1=0.9, \beta_2=0.999$) \\
Temporal coefficient ($\lambda$) & $0.5$ \\
Reward loss coefficient ($c_{1}$) & $0.5$ \\
Value loss coefficient ($c_{2}$) & $0.1$ \\
Consistency loss coefficient ($c_{3}$) & $2$ \\
Exploration schedule ($\epsilon$) & $0.5\rightarrow 0.05$ (25k steps) \\
Planning horizon schedule & $1\rightarrow 5$ (25k steps) \\
Batch size & 2048 (Dog) \\
 & 512 (Humanoid) \\
Momentum coefficient ($\zeta$) & $0.99$ \\
Steps per gradient update & $1$ \\
$\theta^{-}$ update frequency & 2 \\
\bottomrule
\end{tabular}%
}
\caption{\textbf{TD-MPC hyperparameters.} We use the official implementation TD-MPC~\citep{hansen2022temporal} with no adjustments to the hyperparameters, but list it below for completeness. We set the number of training steps to 2 million for continuous control experiments using TD-MPC, and 700k steps for MetaWorld experiments.}
\label{tab:tdmpc-hparams}
\end{minipage}
\end{table}


\section{Policy Discrimination}
\label{sec:policy_discrimination}
Rather than performing an expensive sweep over Video-TADPoLe hyperparameters directly by launching policy supervision experiments across each adapted video model technique, which can be expensive, we look for an offline method to determine reasonable hyperparameter settings.  For each environment, we therefore utilize an example expert quality demonstration video as well as an example poor quality demonstration video (with arbitrary quality levels in-between, if available).  Then, we can perform a search over Video-TADPoLe parameters by computing Video-TADPoLe rewards for these trajectories using an adapted video model, conditioned on the task-relevant text prompt, with respect to different context window, stride, and noise level settings.  We seek parameter settings that, through the adapted video model's Video-TADPoLe reward computation, can correctly distinguish between the expert, text-aligned video demonstration from the poor, text-unaligned video demonstration; this can be done by comparing the predicted Video-TADPoLe rewards.  Once identified in this offline manner, we can subsequently use the discovered settings of context window, stride, and noise level for learning text-conditioned policies.  In practice, we have found that these settings can be reused for novel text-conditioning within the same environment without issue.


\section{Policy Supervision with Additional Pretrained Video Models}

\begin{table*}[h]
\centering
\small

\setlength{\tabcolsep}{4.8pt}
 \scalebox{0.9}{
\begin{tabular}{lccccc}\toprule
Success Rate (\%) w/ & Door Close$^*$ & Door Open & Window Close & Window Open & Drawer Close  \\
\midrule
In-Domain-Only & 100.0 $\pm$ 0.0  & 31.1 $\pm$ 44.0  & 0.0 $\pm$ 0.0  & 33.3 $\pm$ 47.1  & 74.4 $\pm$ 36.2    \\
Vanilla AnimateLCM & 100.0 $\pm$ 0.0  & 0.0 $\pm$ 0.0  & 98.9 $\pm$ 1.9  & 33.3 $\pm$ 29.1   & 100.0 $\pm$ 0.0    \\
\midrule
Prob. Adaptation & 100.0 $\pm$ 0.0  & 0.0 $\pm$ 0.0 & 66.7 $\pm$ 57.7  & 0.0 $\pm$ 0.0 & 100.0 $\pm$ 0.0   \\
Inverse Prob. Adaptation & 100.0 $\pm$ 0.0 & 100.0 $\pm$ 0.0  & 100.0 $\pm$ 0.0  & 94.4 $\pm$ 9.6  & 100.0 $\pm$ 0.0    \\
\midrule
\midrule
Success Rate (\%) w/ & Drawer Open & Coffee Push$^*$ & Soccer & Button Press &  \textbf{Overall}  \\
\midrule
In-Domain-Only &  0.0 $\pm$ 0.0 & 0.0 $\pm$ 0.0 & 0.0 $\pm$ 0.0 & 33.3 $\pm$ 47.1  & 30.2    \\
Vanilla AnimateLCM & 0.0 $\pm$ 0.0 & 5.6 $\pm$ 9.6  & 0.0 $\pm$ 0.0  & 0.0 $\pm$ 0.0  &  37.5    \\
\midrule
Prob. Adaptation & 0.0 $\pm$ 0.0 & 32.2 $\pm$ 28.0  & 4.4 $\pm$ 7.7  & 0.0 $\pm$ 0.0  &  33.7     \\
Inverse Prob. Adaptation & 16.7 $\pm$ 29.0 & 41.1 $\pm$ 15.0 & 4.4 $\pm$ 5.1 & 30.0 $\pm$ 52.0  &  \textbf{65.2}   \\
\bottomrule
\end{tabular}
}
\vspace{-5pt}
\caption[]{\textbf{Policy Learning on MetaWorld with AnimateLCM.} We report the mean success rate across 9 manipulation tasks in MetaWorld, aggregated over 3 seeds.}
\label{table:mw_policysupervision_animatelcm}
\vspace{-5pt}
\end{table*}

We also provide policy supervision results on MetaWorld with AnimateLCM~\citep{wang2024animatelcm} in Table~\ref{table:mw_policysupervision_animatelcm}. Similar to AnimateDiff, vanilla AnimateLCM is also able to achieve decent success rates through Video-TADPoLe. Furthermore, we discover that inverse probabilistic adaptation consistently achieves the best performance with both AnimateDiff and AnimateLCM. With AnimateLCM, inverse probabilistic adaptation obtains the highest overall success rate of $\textbf{65.2\%}$, surpassing all other evaluated video models and adaptation techniques, with non-zero success rates across all evaluated tasks.


\section{Visual Planning with Additional Pretrained Video Models}

\begin{table*}[h]
\centering
\small

\setlength{\tabcolsep}{4.8pt}
 \scalebox{0.9}{
\begin{tabular}{lccccc}\toprule
Success Rate (\%) w/ & Door Close$^*$ & Door Open & Window Close & Window Open & Drawer Close  \\
\midrule
In-Domain-Only & 93.3 $\pm$ 14.9 & 0.0 $\pm$ 0.0 & 53.3 $\pm$ 29.8  & 6.7 $\pm$ 14.9 & 20.0 $\pm$ 29.8 \\
Vanilla AnimateLCM & 100.0 $\pm$ 0.0  & 0.0 $\pm$ 0.0  & 0.0 $\pm$ 0.0  & 20.0 $\pm$ 18.3   & 40.0 $\pm$ 27.9    \\
\midrule
Prob. Adaptation & 100.0 $\pm$ 0.0  & 0.0 $\pm$ 0.0 & 53.3 $\pm$ 38.0  & 0.0 $\pm$ 0.0 & 53.3 $\pm$ 29.8   \\
Inverse Prob. Adaptation & 100.0 $\pm$ 0.0 & 0.0 $\pm$ 0.0  & 40.0 $\pm$ 14.9  & 0.0 $\pm$ 0.0  & 93.3 $\pm$ 14.9    \\
\midrule
\midrule
Success Rate (\%) w/ & Drawer Open & Coffee Push$^*$ & Soccer & Button Press &  \textbf{Overall}  \\
\midrule
In-Domain-Only &  0.0 $\pm$ 0.0  & 0.0 $\pm$ 0.0 & 0.0 $\pm$ 0.0 & 40.0 $\pm$ 14.9 &   23.7  \\
Vanilla AnimateLCM & 0.0 $\pm$ 0.0 & 0.0 $\pm$ 0.0  & 0.0 $\pm$ 0.0  & 0.0 $\pm$ 0.0  &  17.8    \\
\midrule
Prob. Adaptation & 0.0 $\pm$ 0.0 & 0.0 $\pm$ 0.0  & 0.0 $\pm$ 0.0  & 6.7 $\pm$ 14.9  &  23.7     \\
Inverse Prob. Adaptation & 0.0 $\pm$ 0.0 & 0.0 $\pm$ 0.0 & 6.7 $\pm$ 14.9 & 26.7 $\pm$ 27.9  &  \textbf{29.6}   \\
\bottomrule
\end{tabular}
}
\vspace{-5pt}
\caption[]{\textbf{Visual Planning on MetaWorld with AnimateLCM.} We report the mean success rate across 9 manipulation tasks in MetaWorld. Each table entry shows the average success rate aggregated from $5$ seeds. }
\label{table:mw_visualplanning_animatelcm}
\vspace{-5pt}
\end{table*}

We provide visual planning results on MetaWorld with an additional video diffusion model, AnimateLCM~\citep{wang2024animatelcm}, in Table~\ref{table:mw_visualplanning_animatelcm}. We observe both probabilistic adaptation and its inverse version bring improvements in overall success rate compared to Vanilla AnimateLCM. Specifically, inverse probabilistic adaptation achieves the best overall performance and outperforms the in-domain-only baseline by 24.9\%, reconfirming the efficacy of adaptation in improving in-domain task performance.  This further demonstrates that adaptation as an approach can be applied flexibly across different backbone text-to-video models for successful downstream robotic applications.

\section{Visual Planning for Object Navigation in iTHOR Environments}

\begin{table*}[h]
\centering
\small

\setlength{\tabcolsep}{4.8pt}
\resizebox{\textwidth}{!}{
\begin{tabular}{lccccc}\toprule
Success Rate (\%) w/ & Spatula in \textit{Kitchen}$^*$ & Toaster in \textit{Kitchen}$^*$ & Painting in \textit{Living Room}$^*$ & Blinds in \textit{Bedroom}$^*$ & ToiletPaper in \textit{Bathroom}$^*$ \\
\midrule
In-Domain-Only & 13.3 $\pm$ 29.8 & 33.3 $\pm$ 33.3 & 0.0 $\pm$ 0.0  & 13.3 $\pm$ 29.8 & 40.0 $\pm$ 36.5 \\
\midrule
Prob. Adaptation & 20.0 $\pm$ 29.8  & 60.0 $\pm$ 27.9 & 0.0 $\pm$ 0.0  & 26.7 $\pm$ 36.5 & 73.3 $\pm$ 14.9   \\
Inverse Prob. Adaptation & 13.3 $\pm$ 18.3 & 33.3 $\pm$ 23.6  & 0.0 $\pm$ 0.0  & 33.3 $\pm$ 33.3  & 40.0 $\pm$ 14.9    \\
\midrule
\midrule
Success Rate (\%) w/ & Pillow in \textit{Living Room} & DeskLamp in \textit{Living Room} & Mirror in \textit{Bedroom} &  Laptop in \textit{Bedroom} &  \textbf{Overall}  \\
\midrule
In-Domain-Only &  6.7 $\pm$ 14.9  & 6.7 $\pm$ 14.9 & 0.0 $\pm$ 0.0 & 26.7 $\pm$ 27.9 &   15.6  \\
\midrule
Prob. Adaptation & 13.3 $\pm$ 18.3 & 13.3 $\pm$ 18.3  & 0.0 $\pm$ 0.0  & 53.3 $\pm$ 29.8  &  \textbf{28.9}     \\
Inverse Prob. Adaptation & 6.7 $\pm$ 14.9 & 13.3 $\pm$ 18.3 & 6.7 $\pm$ 14.9 & 60.0 $\pm$ 27.9  &  23.0   \\
\bottomrule
\end{tabular}
}
\vspace{-5pt}
\caption[]{\textbf{Visual Planning on iTHOR.} We report the mean success rate across 9 object navigation tasks in iTHOR. Each table entry shows the average success rate aggregated from $5$ seeds. ``$*$'' denotes seen tasks during adaptation.} %
\label{table:mw_videoplanning_ithor}
\vspace{-5pt}
\end{table*}

We provide additional experimentation of adaptation techniques on iTHOR~\citep{eric2017ai2thor}, in which a mobile robotic agent is asked to perform egocentric navigation to a specified target object in different scenes. This benchmark poses challenges of navigating in partially observable settings and allows us to further evaluate adaptation methods on in-domain video generation from egocentric views. To perform adaptation, we reuse the video dataset provided by AVDC~\citep{ko2024avdc}, which spans 12 target objects and includes 25 successful navigation trajectories for each object. We report the success rates of visual planning across 9 navigation tasks in Table~\ref{table:mw_videoplanning_ithor}, in which 4 tasks are unseen during adaptation. We provide a detailed list of iTHOR tasks along with their corresponding text prompts in Table~\ref{table:text_prompts_ithor}. In Table~\ref{table:mw_videoplanning_ithor}, we again observe that the overall performance of both probabilistic adaptation and its inverse outperform that of in-domain-only baseline by a large margin, highlighting that the internet knowledge of pretrained video models can be effectively utilized for various downstream robotic applications through proper adaptation.  This result further highlights how adaptation can be flexibly applied across varied robotic settings.



\section{Step Counts to Task Success in Close-loop Visual Planning}


\begin{table*}[h]
\centering
\small

\setlength{\tabcolsep}{4.8pt}
 \scalebox{0.9}{
\begin{tabular}{lccccc}\toprule
Step Count  w/ & Door Close$^*$ & Door Open & Window Close & Window Open & Drawer Close  \\
\midrule
In-Domain-Only & 80.0 & - & 176.0  & 344.0 & 25.3 \\
Vanilla AnimateDiff & 122.3  & -  & 323.0  & 217.3   & 160.9    \\
\midrule
Direct Finetuning & 96.0 & - & 159.2 & 333.3 & 63.8 \\
Subject Customization  & 150.5 & - & - & 297.3 & 238.7 \\
Prob. Adaptation & 75.4   & - & 171.5  & 312.0 & 31.0   \\
Inverse Prob. Adaptation & 87.0 & -  & 222.6  & -  & 35.8    \\
\midrule
\midrule
Step Count  w/ & Drawer Open & Coffee Push$^*$ & Soccer & Button Press &    \\
\midrule
In-Domain-Only &  -  & - & - & 204.0 &     \\
Vanilla AnimateDiff & - & -  & -  & -  &     \\
\midrule
Direct Finetuning & - & - & - & - & \\
Subject Customization & - & 155.0 & - & - & \\
Prob. Adaptation & 244.0 & 52.0  & -  & 183.2  &      \\
Inverse Prob. Adaptation & - & - & 144.0 & 192.0 &     \\
\bottomrule
\end{tabular}
}
\vspace{-5pt}
\caption[]{\textbf{Step Counts of Visual Planning on MetaWorld.} We report the average number of taken steps in successful evaluation rollouts across 9 manipulation tasks in MetaWorld. Unsuccessful rollouts are omitted. We observed that probabilistic adaptation in general achieves task success using fewer number of steps.} %
\label{table:mw_visualplanning_stepcount}
\vspace{-5pt}
\end{table*}







\end{document}


