\documentclass{article}

% if you need to pass options to natbib, use, e.g.:
%     \PassOptionsToPackage{numbers, compress}{natbib}
% before loading neurips_data_2023

% ready for submission
\PassOptionsToPackage{numbers, compress}{natbib}
\usepackage[final]{neurips_data_2023}

% to compile a preprint version, add the [preprint] option, e.g.:
%     \usepackage[preprint]{neurips_data_2023}
% This will indicate that the work is currently under review.

% to compile a camera-ready version, add the [final] option, e.g.:
%     \usepackage[final]{neurips_data_2023}

% to avoid loading the natbib package, add option nonatbib:
%    \usepackage[nonatbib]{neurips_data_2023}

% Submissions to the datasets and benchmarks are typically non anonymous,
% but anonymous submissions are allowed. If you feel that you must submit 
% anonymously, you can compile an anonymous version by adding the [anonymous] 
% option, e.g.:
%     \usepackage[anonymous]{neurips_data_2023}
% This will hide all author names.


\usepackage[utf8]{inputenc} % allow utf-8 input
\usepackage[T1]{fontenc}    % use 8-bit T1 fonts
\usepackage{hyperref}       % hyperlinks
\usepackage{url}            % simple URL typesetting
\usepackage{booktabs}       % professional-quality tables
\usepackage{amsfonts}       % blackboard math symbols
\usepackage{nicefrac}       % compact symbols for 1/2, etc.
\usepackage{microtype}      % microtypography
\usepackage{xcolor}         % colors
\usepackage{qiangstyle}
\usepackage{amsmath}
\usepackage{amsfonts}
\usepackage{hyperref}
\usepackage{amsmath}
\usepackage{amssymb}
\usepackage{mathtools}
\usepackage{amsthm}
\usepackage{hyperref}
\usepackage{url}
\usepackage{xcolor}
\usepackage{amssymb}
\usepackage{multirow}
\usepackage{bbding}
\usepackage{bbm}
\usepackage{listings}
\usepackage[T1]{fontenc}
\usepackage{booktabs, tabularx, colortbl} 
\usepackage{caption}
\usepackage{wrapfig,lipsum}
\usepackage{hhline}
\usepackage{tcolorbox}
\usepackage{graphicx}
\usepackage{bmpsize}

\definecolor{myRed}{rgb}{0.9, 0.5, 0.6}
\definecolor{myBlue}{rgb}{0.55, 0.70, 0.90}
\definecolor{myGreen}{rgb}{0.64, 0.83, 0.35}
\definecolor{myYellow}{rgb}{0.9, 0.75, 0.4}

\newcommand{\bo}[1]{{\textcolor{magenta}{[bo: #1]}}}
\newcommand{\commentp}[1]{{\textcolor[rgb]{0.9,0.1,0.1}{[Peter: #1]}}}
\newcommand{\yuke}[1]{{\textcolor{green}{[YZ: #1]}}}
\newcommand{\chongkai}[1]{{\textcolor{cyan}{[chongkai: #1]}}}
\newcommand{\yifeng}[1]{{\textcolor[rgb]{0.2,0.7,0.2}{[Yifeng: #1]}}}
\newcommand{\yifengchange}[1]{{\textcolor[rgb]{0.2,0.7,0.2}{[Yifeng's Change Suggestsions: #1]}}}
\newcommand{\yy}[1]{\small{\textcolor{blue}{Yihao: #1}}}

\newcommand{\loosepar}{\looseness=-1}

\newcommand{\lldm}{\textsc{LLDM}}
\newcommand{\lb}{\textsc{LIBERO}}
\newcommand{\liberohundred}{\textsc{LIBERO-100}}
\newcommand{\liberolong}{\textsc{LIBERO-Long}}
\newcommand{\liberoninety}{\textsc{LIBERO-90}}
\newcommand{\liberospatial}{\textsc{LIBERO-Spatial}}
\newcommand{\liberoobject}{\textsc{LIBERO-Object}}
\newcommand{\liberox}{\textsc{LIBERO-X}}
\newcommand{\liberogoal}{\textsc{LIBERO-Goal}}
\newcommand{\challengeknowledge}{\textsc{(T1)}}
\newcommand{\challengealgorithm}{\textsc{(T3)}}
\newcommand{\challengearchitecture}{\textsc{(T2)}}
\newcommand{\challengeordering}{\textsc{(T4)}}
\newcommand{\challengepretraining}{\textsc{(T5)}}
\newcommand{\qone}{\textbf{Q1}}
\newcommand{\qtwo}{\textbf{Q2}}
\newcommand{\qthree}{\textbf{Q3}}
\newcommand{\qfour}{\textbf{Q4}}
\newcommand{\qfive}{\textbf{Q5}}
\newcommand{\qsix}{\textbf{Q6}}

\newcommand{\bcrnn}{\textsc{ResNet-RNN}}
\newcommand{\bct}{\textsc{ResNet-T}}
\newcommand{\bcvilt}{\textsc{ViT-T}}

\newcommand{\er}{\textsc{ER}}
\newcommand{\ewc}{\textsc{EWC}}
\newcommand{\packnet}{\textsc{PackNet}}
\newcommand{\seql}{\textsc{SeqL}}
\newcommand{\mtl}{\textsc{MTL}}

\newcommand{\robomimic}{\texttt{Robomimic}}
\newcommand{\robosuite}{\texttt{Robosuite}}


\newcommand{\myparagraph}[1]{\textbf{#1}~~~}
\newcommand{\mysection}[1]{\section{#1}}
\newcommand{\mysubsection}[1]{\subsection{#1}}

\newcommand{\fs}[1]{\footnotesize $\pm$#1}

\usepackage[capitalize,noabbrev]{cleveref}
\definecolor{top1_boxit_color}{RGB}{186,51,121}
\definecolor{top2_boxit_color}{RGB}{186,51,121}
\definecolor{top3_boxit_color}{RGB}{186,51,121}
\newcommand{\toponeboxit}{\cellcolor{top1_boxit_color!90}}
\newcommand{\toptwoboxit}{\cellcolor{top2_boxit_color!50}}
\newcommand{\topthreeboxit}{\cellcolor{top3_boxit_color!20}}

\title{LIBERO: Benchmarking Knowledge Transfer for Lifelong Robot Learning}


% The \author macro works with any number of authors. There are two commands
% used to separate the names and addresses of multiple authors: \And and \AND.
%
% Using \And between authors leaves it to LaTeX to determine where to break the
% lines. Using \AND forces a line break at that point. So, if LaTeX puts 3 of 4
% authors names on the first line, and the last on the second line, try using
% \AND instead of \And before the third author name.

\author{%
  $^\dagger$Bo Liu\thanks{Equal contribution.}, $^\dagger$Yifeng Zhu$^*$, $^\ddagger$Chongkai Gao$^*$, $^{\dagger}$Yihao Feng\\\textbf{$^\dagger$Qiang Liu, $^{\dagger}$Yuke Zhu, $^{\dagger,\mathsection}$Peter Stone} \\
  $^\dagger$The University of Texas at Austin, $^{\mathsection}$Sony AI, $^\ddagger$Tsinghua University\\
  \texttt{\{bliu,yifengz,lqiang,yukez,pstone\}@cs.utexas.edu}\\
  \texttt{yihao.ac@gmail.com, gck20@mails.tsinghua.edu.cn}
}

\begin{document}

\maketitle

\begin{abstract}
Lifelong learning offers a promising paradigm of building a generalist agent that learns and adapts over its lifespan. 
%
Unlike traditional lifelong learning problems in image and text domains, which primarily involve the transfer of declarative knowledge of entities and concepts, lifelong learning in decision-making (\lldm{}) also necessitates the transfer of procedural knowledge, such as actions and behaviors.
%
To advance research in \lldm{}, we introduce \lb{}, a novel benchmark of lifelong learning for robot manipulation. Specifically, \lb{} highlights five key research topics in \lldm{}: \textbf{1)} how to efficiently transfer declarative knowledge, procedural knowledge, or the mixture of both; \textbf{2)} how to design effective policy architectures and \textbf{3)} effective algorithms for \lldm{}; \textbf{4)} the robustness of a lifelong learner with respect to task ordering; and \textbf{5)} the effect of model pretraining for \lldm{}.
We develop an extendible \emph{procedural generation} pipeline that can in principle generate infinitely many tasks. For benchmarking purpose, we create four task suites (130 tasks in total) that we use to investigate the above-mentioned research topics. To support sample-efficient learning, we provide high-quality human-teleoperated demonstration data for all tasks.
%
Our extensive experiments present several insightful or even \emph{unexpected} discoveries: sequential finetuning outperforms existing lifelong learning methods in forward transfer, no single visual encoder architecture excels at all types of knowledge transfer, and naive supervised pretraining can hinder agents' performance in the subsequent \lldm{}.\footnote{Check the website at \url{https://libero-project.github.io} for the code and the datasets.}
\end{abstract}
\section{Introduction}

Recent advancements in video generative models have made promising their application to interactive problems and decision-making.  Such video models trained explicitly on in-domain demonstrations have demonstrated accurate encoding of environment-specific visual details and dynamics, and have been popular choices to utilize for robotic learning~\citep{du2024video, huang2023diffusion, yang2023learning,  ko2024avdc, liang2024dreamitate}.  When optimized on expert video demonstrations, their encoded understanding of expert behavior can be directly used to supervise the learning of high-performing policies~\citep{huang2023diffusion, escontrela2024video}, and applied as performant visual planners~\citep{du2024learning} in robotic settings.  However, for arbitrary robotic environments, there is usually a severe difference in scale of tractably available expert demonstration data, especially with associated text labelling, in comparison with general internet-scale datasets of videos paired with natural language.  As a result, such in-domain video generative models usually suffer from weaker generalization capability, across novel text specifications and motions of interest.


\looseness=-1
Instead of training directly on in-domain demonstrations, stronger generalization performance can be obtained by using text-to-video models pretrained on internet-scale data.  Having summarized powerful priors over visual styles, natural motion, and alignment with natural language from large-scale data, such models can be leveraged to supervise policies that behave flexibly conditioned on text across multiple environment visual styles without modification~\citep{luo2024text}.  However, interaction is often performed in a fixed environment with specific visual characteristics and potentially unique interaction dynamics, which a general environment-agnostic video model may not inherently understand or respect.  Thus, directly applying a large-scale pretrained video generative model without modification comes with a potential drawback; they may not understand the intricacies of particular environments of interest to successfully facilitate high-performance behavior within them.


These considerations naturally motivate the investigation of ways to mutually cover the independent deficiencies of each approach.  In this work, we perform a thorough study on novel task generalization via adapting internet video knowledge; we seek to illuminate how in-domain information can be best integrated into large-scale pretrained text-to-video models, such that powerful zero-shot text-conditioned generalization capabilities are enabled while considering environment-specific knowledge pertaining to visual styles and interaction dynamics.  We compare the downstream robotic performance of multiple adaptation techniques and contrast their respective requirements on in-domain data samples, which range from utilizing only a few still-frames of the agent to text-labelled video demonstrations, and training resources, which span from direct finetuning of the large-scale video model to utilizing it only for inference without any updates.  Broadly, we provide this study of adaptation for facilitating action prediction (\textbf{Adapt2Act}) as valuable insight to the practitioner interested in balancing performance with resource availability.



We perform standardized evaluations across both robotic manipulation tasks~\citep{yu2020meta} and continuous control~\citep{tassa2018deepmind}, and demonstrate that adapted video generative models are able to successfully act as accurate video planners for novel text-conditioned specifications across a variety of robotic tasks, and can also supervise the learning of novel text-conditioned policies. Furthermore, in this work we propose a novel adaptation technique termed \textit{Inverse Probabilistic Adaptation}, which we highlight achieves consistently strong generalization performance across robotic environments and downstream evaluation approaches.  We also discover that even when only suboptimal in-domain demonstrations are provided, it can still effectively leverage web-scale priors and text conditioning to generate coherent video plans and successfully solve novel tasks.  Visualizations and code are provided at \href{https://diffusion-supervision.github.io/adapt2act/}{\nolinkurl{diffusion-supervision.github.io/adapt2act/}}.



\begin{figure}
    \centering
    \includegraphics[width=0.9\linewidth]{figures/adaptation_techniques_adjusted_new.png}
    \vspace{-5pt}
    \caption{\textbf{Adaptation Techniques.} We explore how in-domain information can be integrated into large-scale text-to-video models through three different adaptation techniques: Subject Customization, Probabilistic Adaptation, and Direct Finetuning.  Subject Customization only modifies the image and text encoder, rather than the motion module, and is lightweight in terms of data requirements: it only utilizes pairs of static images and text annotated with a special identifier.  Probabilistic Adaptation learns a small in-domain model from paired video data, which is then used through score composition with a large-scale video model that is kept frozen.  The small in-domain model can be flexibly parameterized to consider available training resources. Direct Finetuning seeks to update the motion module of the large-scale pretrained video model with in-domain paired video data.}
    \label{fig:adaptation}
    \vspace{-15pt}
\end{figure}




\mysection{Background}
This section introduces the problem formulation and defines key terms used throughout the paper.

\mysubsection{Markov Decision Process for Robot Learning}
A robot learning problem can be formulated as a finite-horizon Markov Decision Process:
$
\mathcal{M} = (\mathcal{S}, \mathcal{A}, \mathcal{T}, H, ~{\color{purple}{\mu_0, R}}).
$
Here, $\mathcal{S}$ and $\mathcal{A}$ are the state and action spaces of the robot. $\mu_0$ is the initial state distribution, $R: \mathcal{S}\times \mathcal{A} \rightarrow \mathbb{R}$ is the reward function, and $\mathcal{T}: \mathcal{S} \times \mathcal{A} \rightarrow \mathcal{S}$ is the transition function. In this work, we assume a sparse-reward setting and replace $R$ with a goal predicate $g: \mathcal{S} \rightarrow \{0, 1\}$. The robot's objective is to learn a policy $\pi$ that maximizes the expected return:
$
\max_\pi J(\pi) = \mathbb{E}_{s_t, a_t \sim \pi, \mu_0} [\sum_{t=1}^{H} g(s_t)].
$

\mysubsection{Lifelong Robot Learning Problem}
In a \emph{lifelong robot learning problem}, a robot sequentially learns over $K$ tasks $\{T^1, \dots, T^K\}$ with a single policy $\pi$. We assume $\pi$ is conditioned on the task, i.e., $\pi(\cdot \mid s; T)$. For each task, $T^k \equiv (\mu_0^k, g^k)$ is defined by the initial state distribution $\mu_0^k$ and the goal predicate $g^k$.\footnote{Throughout the paper, a superscript/subscript is used to index the task/time step.} We assume $\mathcal{S}, \mathcal{A}, \mathcal{T}, H$ are the same for all tasks. Up to the $k$-th task $T^k$, the robot aims to optimize
\begin{equation}
    \max_\pi~ J_{\text{LRL}}(\pi) = \frac{1}{k}\sum_{p=1}^k \bigg[ \mathop{\mathbb{E}}\limits_{s^p_t, a^p_t \sim \pi(\cdot;T^p),~ \mu_0^p} \bigg[\sum_{t=1}^L g^p(s_t^p) \bigg] \bigg].
    \label{eq:LRL}
\end{equation}
An important feature of the lifelong setting is that the agent loses access to the previous $k-1$ tasks when it learns on task $T^k$.

\myparagraph{Lifelong Imitation Learning} Due to the challenge of sparse-reward reinforcement learning,
 we consider a practical alternative setting where a user would provide a small demonstration dataset for each task in the sequence. Denote $D^k = \{ \tau^k_i \}_{i=1}^N$ as $N$ demonstrations for task $T^k$. Each $\tau^k_i = (o_0, a_0, o_1, a_1, \dots, o_{l^k})$ where $l^k \leq H$. Here, $o_t$ is the robot's sensory input, including the perceptual observation and the information about the robot's joints and gripper. In practice, the observation $o_t$ is often non-Markovian. Therefore, following works in partially observable MDPs~\citep{hausknecht2015deep}, we represent $s_t$ by the aggregated history of observations, i.e. $s_t \equiv o_{\leq t} \triangleq (o_0, o_1, \dots, o_t) $.
This results in the \emph{lifelong imitation learning problem} with the same objective as in Eq.~\eqref{eq:LRL}. But during training, we perform behavioral cloning~\citep{bain1995framework} with the following surrogate objective function:
\begin{equation}
 \min_\pi~ J_{\text{BC}}(\pi) = \frac{1}{k}\sum_{p=1}^k \mathop{\mathbb{E}}\limits_{o_t, a_t \sim D^p} \bigg[ \sum_{t=0}^{l^p} \mathcal{L}\big(\pi(o_{\leq t}; T^p), a^p_t\big)\bigg]\,,
    \label{eq:bc}
\end{equation}
where $\mathcal{L}$ is a supervised learning loss, e.g., the negative log-likelihood loss, and $\pi$ is a Gaussian mixture model. Similarly, we assume $\{D^p: p < k\}$ are not fully available when learning $T^k$.
\mysection{Research Topics in \lldm{}}
\label{sec:research-challenge}
We outline five major research topics in \lldm{} that motivate the design of \lb{} and our study.


\myparagraph{\challengeknowledge{} Transfer of Different Types of Knowledge} 
In order to accomplish a task such as \emph{put the ketchup next to the plate in the basket}, a robot must understand the concept \emph{ketchup}, the location of the \emph{plate/basket}, and how to \emph{put} the ketchup in the basket. 
Indeed, robot manipulation tasks in general necessitate different types of knowledge, making it hard to determine the cause of failure. We present four task suites in Section~\ref{sec:libero-suite}: three task suites for studying the transfer of knowledge about spatial relationships, object concepts, and task goals in a disentangled manner, and one suite for studying the transfer of mixed types of knowledge.

\myparagraph{\challengearchitecture{} Neural Architecture Design} 
An important research question in \lldm{} is how to design effective neural architectures to abstract the multi-modal observations (images, language descriptions, and robot states) and transfer only relevant knowledge when learning new tasks.

\myparagraph{\challengealgorithm{} Lifelong Learning Algorithm Design} Given a policy architecture, it is crucial to determine what learning algorithms to apply for \lldm{}. Specifically, the sequential nature of \lldm{} suggests that even minor forgetting over successive steps can potentially lead to a total failure in execution. As such, we consider the design of lifelong learning algorithms to be an open area of research in \lldm{}.

\myparagraph{\challengeordering{} Robustness to Task Ordering}
It is well-known that task curriculum influences policy learning \cite{bengio2009curriculum,narvekar2020curriculum}. A robot in the real world, however, often cannot choose which task to encounter first. Therefore, a good lifelong learning algorithm should be robust to different task orderings.

\myparagraph{\challengepretraining{} Usage of Pretrained Models} In practice, robots will be most likely pretrained on large datasets in factories before deployment~\citep{kaelbling2020foundation}. However, it is not well-understood whether or how pretraining could benefit subsequent \lldm{}. 
\mysection{LIBERO}
This section introduces the components in \lb{}: the procedural generation pipeline that allows the never-ending creation of tasks (Section~\ref{sec:procedural}), the four task suites we generate for benchmarking (Section~\ref{sec:libero-suite}), five algorithms (Section~\ref{sec:algo}), and three neural architectures (Section~\ref{sec:arch}). 
\begin{figure*}[t!]
    \centering
    \includegraphics[width=\textwidth]{figures/libero_fig2.png}

    \caption{\lb{}'s procedural generation pipeline:  Extracting behavioral templates from a large-scale human activity dataset \textbf{(1)}, Ego4D, for generating task instructions \textbf{(2)}; Based on the task description, selecting the scene and generating the PDDL description file \textbf{(3)} that specifies the objects and layouts \textbf{(A)}, the initial object configurations \textbf{(B)}, and the task goal \textbf{(C)}.
    % This figure highlights the procedural generation pipeline in \lb{}. \textbf{(1)} We extract behavioral templates from a large-scale human activity dataset, Ego4D; \textbf{(2)} We generate language instructions by selecting the objects whose models are available in the simulators; \textbf{(3)} We select a scene (shown on the left) that is appropriate for the language instruction, and programmatically generate a PDDL-based scene description file that specifies: \textbf{(A)} the initial configuration distribution $\mu_{0}$ which includes object categories, placement regions; \textbf{(B)} the initial states of objects in the format of a list of predicates; and \textbf{(C)} the goal in a logic proposition format that consists of predicates. The goal is satisfied when all the predicates are true. The upper right shows screenshots of initial configurations and the configuration when the goal is satisfied.
    }
    \label{fig:libero-procedural-generation}
\end{figure*}

\mysubsection{Procedural Generation of Tasks}
\label{sec:procedural}
Research in \lldm{} requires a systematic way to create new tasks while maintaining task diversity and relevance to existing tasks. \lb{} procedurally generates new tasks in three steps: \textbf{1)} extract behavioral templates from language annotations of human activities and generate sampled tasks described in natural language based on such templates; \textbf{2)} specify an initial object distribution given a task description; and \textbf{3)} specify task goals using a propositional formula that aligns with the language instructions.
%
Our generation pipeline is built on top of \robosuite{}~\citep{zhu2020robosuite}, a modular manipulation simulator that offers seamless integration. Figure~\ref{fig:libero-procedural-generation} illustrates an example of task creation using this pipeline, and each component is expanded upon below.

\myparagraph{Behavioral Templates and Instruction Generation} Human activities serve as a fertile source of tasks that can inspire and generate a vast number of manipulation tasks. 
%
We choose a large-scale activity dataset, Ego4D~\citep{grauman2022ego4d}, which includes a large variety of everyday activities with language annotations. We pre-process the dataset by extracting the language descriptions and then summarize them into a large set of commonly used language templates. After this pre-processing step, we use the templates and select objects available in the simulator to generate a set of task descriptions in the form of language instructions. For example, we can generate an instruction ``Open the drawer of the cabinet'' from the template ``Open ...''.

\myparagraph{Initial State Distribution ($\mu_0$)} 
To specify $\mu_0$, we first sample a scene layout that matches the objects/behaviors in a provided instruction. For instance, a kitchen scene is selected for an instruction \textit{Open the top drawer of the cabinet and put the bowl in it}. Then, the details about $\mu_0$ are generated in the PDDL language~\citep{mcdermott1998pddl,srivastava2022behavior}.
Concretely, $\mu_0$ contains information about object categories and their placement (Figure~\ref{fig:libero-procedural-generation}-\textbf{(A)}), and their initial status (Figure~\ref{fig:libero-procedural-generation}-\textbf{(B)}).

\myparagraph{Goal Specifications $(g)$} Based on $\mu_{0}$ and the language instruction, we specify the task goal using a conjunction of predicates. Predicates include \emph{unary predicates} that describe the properties of an object, such as \texttt{Open}(X) or \texttt{TurnOff}(X), and \emph{binary predicates} that describe spatial relations between objects, such as \texttt{On}(A, B) or \texttt{In}(A, B). An example of the goal specification using PDDL language can be found in Figure~\ref{fig:libero-procedural-generation}-\textbf{(C)}. The simulation terminates when all predicates are verified true.

\mysubsection{Task Suites}
\label{sec:libero-suite}
While the pipeline in Section~\ref{sec:procedural} supports the generation of an unlimited number of tasks, we offer fixed sets of tasks for benchmarking purposes.
\lb{} has four task suites: \liberospatial, \liberoobject, \liberogoal, and \liberohundred. The first three task suites are curated to disentangle the transfer of \emph{declarative} and \emph{procedural} knowledge (as mentioned in~\challengeknowledge{}), while \liberohundred{} is a suite of 100 tasks with entangled knowledge transfer. 
% Note that the procedural generation pipeline can scale up the number of tasks in the future with ease.

\myparagraph{\liberox{}} \liberospatial{}, \liberoobject{}, and \liberogoal{} all have 10 tasks\footnote{
A suite of 10 tasks is enough to observe catastrophic forgetting while maintaining computation efficiency.
} 
and are designed to investigate the controlled transfer of knowledge about spatial information (declarative), objects (declarative), and task goals (procedural).
Specifically, all tasks in \liberospatial{} request the robot to place a bowl, among the same set of objects, on a plate. But there are two identical bowls that differ only in their location or spatial relationship to other objects. Hence, to successfully complete \liberospatial{}, the robot needs to continually learn and memorize new spatial relationships.
% evaluates if a policy can associate visual features and language with knowledge of spatial understanding by using consistent object layouts except for two bowls with unique arrangements in each task. The goal for the robot is to identify one of the bowls based on its spatial location or relation to other objects and place it on the plate. 
% The robot is requested to pick-place a unique object in all the \liberoobject{} tasks.
All tasks in \liberoobject{} request the robot to pick-place a unique object. 
Hence, to accomplish \liberoobject{}, the robot needs to continually learn and memorize new object types.
%
All tasks in \liberogoal{} share the same objects with fixed spatial relationships but differ only in the task goal. Hence, to accomplish \liberogoal{}, the robot needs to continually learn new knowledge about motions and behaviors.
More details are in Appendix~\ref{appendix:task}.

\myparagraph{\liberohundred}  \liberohundred{} contains 100 tasks that entail diverse object interactions and versatile motor skills. 
In this paper, we split \liberohundred{} into 90 short-horizon tasks (\liberoninety{}) and 10 long-horizon tasks (\liberolong{}). \liberoninety{} serves as the data source for pretraining~\textbf{\challengepretraining{}} and \liberolong{} for downstream evaluation of lifelong learning algorithms.

\mysubsection{Lifelong Learning Algorithms}
\label{sec:algo}
We implement three representative lifelong learning algorithms to facilitate research in algorithmic design for \lldm{}. Specifically, we implement Experience Replay (\er{})~\citep{chaudhry2019tiny}, Elastic Weight Consolidation (\ewc{})~\citep{kirkpatrick2017overcoming}, and~\packnet{}~\citep{mallya2018packnet}. We pick \er{}, \ewc{}, and \packnet{} because they correspond to the memory-based, regularization-based, and dynamic-architecture-based methods for lifelong learning. In addition, prior research~\cite{Woczyk2021ContinualWA} has discovered that they are state-of-the-art methods. Besides these three methods, we also implement sequential finetuning (\seql{}) and multitask learning (\mtl{}), which serve as a lower bound and upper bound for lifelong learning algorithms, respectively. More details about the algorithms are in Appendix~\ref{appendix:llalgo}.
% To evaluate the efficacy of different algorithmic designs to retain knowledge in \lldm{}, we study three representative lifelong learning algorithms, Experience Replay (\er{})~\citep{chaudhry2019tiny}, Elastic Weight Consolidation (\ewc{})~\citep{kirkpatrick2017overcoming}, and~\packnet{}~\citep{mallya2018packnet}. \er{} is a memory-based method to store important past data for new task learning. \ewc{} is a regularization-based method that constrains the neural network update. \packnet{} is a method based on a dynamic architecture that updates the neural network architectures on the fly across different tasks. We additionally implement two baseline methods, sequential learning (\seql{}) and multitask learning (\mtl{}), which serve as a lower bound and upper bound for lifelong learning algorithms, respectively.

\mysubsection{Neural Network Architectures} 
\label{sec:arch}

\label{sec:method-architecture}
We implement three vision-language policy networks, \bcrnn{}, \bct{}, and \bcvilt{}, that integrate visual, temporal, and linguistic information for \lldm{}.
%
Language instructions of tasks are encoded using pretrained BERT embeddings~\cite{devlin2018bert}. 
% \bcrnn{}/\bct{}/\bcvilt{} uses ResNet/ResNet/Vision Transformer (ViT) to abstract knowledge from visual inputs and uses LSTM/Transformer/Transformer to integrate temporal information (See Figure~\ref{fig:architectures}, \ref{fig:encoders}). The task information is encoded using a task token from language embedding if a ViT is used or encoded using the FiLM method~\citep{perez2018film} if a ResNet is used.
The \bcrnn{}~\citep{mandlekar2021matters} uses a ResNet as the visual backbone that encodes per-step visual observations and an LSTM as the temporal backbone to process a sequence of encoded visual information. The language instruction is incorporated into the ResNet features using the FiLM method~\citep{perez2018film} and added to the LSTM inputs, respectively.
\bct{} architecture~\citep{zhu2022viola} uses a similar ResNet-based visual backbone, but a transformer decoder~\citep{vaswani2017attention} as the temporal backbone to process outputs from ResNet, which are a temporal sequence of visual tokens. The language embedding is treated as a separate token in inputs to the transformer alongside the visual tokens.
The \bcvilt{} architecture~\citep{kim2021vilt}, which is widely used in visual-language tasks, uses a Vision Transformer (ViT) as the visual backbone and a transformer decoder as the temporal backbone. The language embedding is treated as a separate token in inputs of both ViT and the transformer decoder. All the temporal backbones output a latent vector for every decision-making step. We compute the multi-modal distribution over manipulation actions using a Gaussian-Mixture-Model (GMM) based output head~\citep{bishop1994mixture, mandlekar2021matters, wang2023mimicplay}. In the end, a robot executes a policy by sampling a continuous value for end-effector action from the output distribution. Figure~\ref{fig:architectures} visualizes the three architectures.\loosepar{}

For all the lifelong learning algorithms and neural architectures, we use behavioral cloning (BC)~\citep{bain1995framework} to train policies for individual tasks (See \eqref{eq:bc}). BC allows for efficient policy learning such that we can study lifelong learning algorithms with limited computational resources. 
To train BC, we provide 50 trajectories of high-quality demonstrations for every single task in the generated task suites. The demonstrations are collected by human experts through teleoperation with 3Dconnexion Spacemouse.



\section{Experiments}
\subsection{Dataset}
\paragraph{CALVIN.}
\emph{CALVIN}~\cite{mees2022calvin} is a simulated benchmark tailored for evaluating long-horizon, language-conditioned robotic manipulation. It comprises four simulated environments (A, B, C, and D), each containing demonstration trajectories collected via human teleoperation. The benchmark encompasses 34 distinct manipulation tasks with a total of 1,000 unique language instructions. Performance is measured by the average number of successfully completed sub-tasks within a sequence. Standard evaluation protocols include the \emph{ABC$\rightarrow$D} and \emph{ABCD$\rightarrow$D} settings, which test a model’s ability to generalize to unseen environments and compositions of long-horizon tasks.
\vspace{-2mm}
\paragraph{LIBERO.} 
The \emph{LIBERO} benchmark~\cite{liu2023libero} is a comprehensive suite for lifelong robotic manipulation, comprising four task suites with 10 tasks and 50 human demonstrations each. These suites are designed to evaluate different generalization abilities: \emph{LIBERO-Spatial} tests spatial reasoning by varying layouts with fixed objects; \emph{LIBERO-Object} assesses object-level generalization with varying objects in a fixed scene; \emph{LIBERO-Goal} targets goal-conditioned behavior by varying task goals; and \emph{LIBERO-Long} (\emph{LIBERO-10}) features long-horizon, compositional tasks with diverse objects, layouts, and goals, challenging temporal and compositional reasoning.
\vspace{-2mm}
\paragraph{SimplerEnv.} 
SimplerEnv~\cite{li2024evaluating} serves as a simulation benchmark designed to evaluate the transferability and generalization capabilities of models trained on real-world video data. It incorporates diverse manipulation setups across both the WidowX and Google Robot platforms, encompassing variations in lighting conditions, object textures, color distributions, and camera viewpoints.

% ----------------------------------------------------------------------------------------------
\subsection{Implementation Details}
The model adopts a purely autoregressive Transformer architecture with 8.5 billion parameters, identical to Emu3~\cite{wang2024emu3}. Images are tokenized using a VQ-based image encoder with a spatial compression factor of 8. For action encoding, we use the relative differences between consecutive frames. We first apply 1st and 99th percentile normalization, and then utilize the FAST tokenizer~\cite{pertsch2025fast}, which has a vocabulary size of 1024 and replaces the final 1024 token IDs of the language tokenizer.

\vspace{-2mm}
\paragraph{Post-training Stage.}
In the post-training stage, we leverage large-scale robot-centric video datasets to study the effects of various post-training strategies on downstream policy learning. The model is initialized with pre-trained weights from the first stage of Emu3~\cite{wang2024emu3}. We curate a total of 622K videos from existing robotics datasets (details provided in the appendix), and identify the world model as the most effective post-training approach. During training, supervision is applied solely on the vision tokens. The model is trained for 30K steps with a batch size of 64.

\vspace{-2mm}
\paragraph{Fine-tuning Stage.}
During fine-tuning, the model is initialized with weights from the post-training stage and trained using a two-frame interleaved vision-action sequence with an action chunk size of 10. A cosine annealing learning rate schedule is applied, starting at \(8 \times 10^{-5}\), and the loss is computed solely over action tokens.
For the CALVIN benchmark, RGB observations from both third-person (\(200 \times 200\)) and wrist-view (\(80 \times 80\)) cameras are used. Training is conducted on A100 GPUs with a batch size of 192 for 8k steps.
For the LIBERO benchmark, third-person and wrist-view RGB images (both at \(200 \times 200\)) are used to train a unified model with a batch size of 192 for 8k steps. A single model is evaluated across four task suites.
For the SimplerEnv benchmark, single-view RGB observations are used with input resized to \(256 \times 256\). Training is conducted on the Bridge-WidowX setup using a batch size of 128 for 20k steps, with an action chunk size of 5.

Additional implementation details on the post-training strategy, real-robot fine-tuning procedures, and autonomous driving experiments are provided in the appendix.
% ----------------------------------------------------------------------------------------------
\begin{table}[ht]
    \centering
    \caption{\textbf{Long-horizon robotic manipulation evaluation on the CALVIN benchmark.}}
      \resizebox{0.86\textwidth}{!}{
    \begin{tabular}{l l c c c c c c}
        \toprule
        \textbf{Method}  & \textbf{Task} & \multicolumn{5}{c}{\textbf{Tasks Completed in a Row}} & \textbf{Avg. Len $\uparrow$} \\
        \cmidrule(lr){3-7}
        &  & 1 & 2 & 3 & 4 & 5 & \\
        \midrule
        MCIL~\cite{lynch2020language} & ABCD$\rightarrow$D &0.373 & 0.027 & 0.002 & 0.000 & 0.000 & 0.40\\
        RT-1~\cite{brohan2022rt}  & ABCD$\rightarrow$D & 0.844 & 0.617 & 0.438 & 0.323 & 0.227 & 2.45 \\
        Robo-Flamingo~\cite{li2024vision} & ABCD$\rightarrow$D & 0.964 & 0.896 & 0.824 & 0.740 & 0.660 & 4.09 \\
        GR-1~\cite{wu2023unleashing}  & ABCD$\rightarrow$D & 0.949 & 0.896 & 0.844 & 0.789 & 0.731 & 4.21 \\
        UP-VLA~\cite{zhang2025up}  & ABCD$\rightarrow$D & 0.962 & 0.921 & 0.879 & 0.842 & 0.812 & 4.42 \\
        RoboVLMs~\cite{li2024towards} & ABCD$\rightarrow$D & 0.967 & 0.930 & 0.899 & 0.865 & 0.826 & 4.49 \\
        \rowcolor[gray]{0.9} \textbf{\methodname{}} & ABCD$\rightarrow$D & \textbf{0.985} & \textbf{0.961} & \textbf{0.931} & \textbf{0.899} & \textbf{0.851} & \textbf{4.63}\\
        \midrule
        MCIL~\cite{lynch2020language} & ABC$\rightarrow$D &0.304	&0.013&	0.002	&0.000	&0.000&	0.31\\
        Robo-Flamingo~\cite{li2024vision} & ABC$\rightarrow$D & 0.824 & 0.619 & 0.466 & 0.331 & 0.235 & 2.47 \\
        SuSIE~\cite{black2023zero}& ABC$\rightarrow$D& 0.870 &0.690 &0.490 &0.380 &0.260 &2.69\\
        GR-1~\cite{wu2023unleashing}  & ABC$\rightarrow$D & 0.854 & 0.712 &0.596	&0.497	&0.401	&3.06 \\
        UP-VLA~\cite{zhang2025up}  & ABC$\rightarrow$D & 0.928 &0.865& 0.815& 0.769& 0.699& 4.08 \\
        RoboVLMs~\cite{li2024towards} & ABC$\rightarrow$D &
        0.980 &0.936 &0.854& 0.778 &0.704 &4.25 \\
        Seer-Large~\cite{tian2024predictive} & ABC$\rightarrow$D &0.963 &0.916 &0.861 &0.803 &0.740 & 4.28\\
         \rowcolor[gray]{0.9} \textbf{\methodname{}}  & ABC$\rightarrow$D & \textbf{0.989} & \textbf{0.948} & \textbf{0.890} & \textbf{0.828} & \textbf{0.751} & \textbf{4.41}\\
        \bottomrule
    \end{tabular}}
        \label{tab:calvin_results}
\end{table}


\subsection{Main Results}
In this section, we evaluate our method on three simulation benchmarks: CALVIN (long-horizon tasks), LIBERO (diverse generalization), and SimplerEnv (real-to-sim manipulation). Our approach consistently achieves state-of-the-art performance across all settings.
\vspace{-2mm}
\paragraph{CALVIN Simulation Evaluation.}
Table~\ref{tab:calvin_results} presents the experimental results in the CALVIN benchmark. Our method achieves the highest performance on both the ABC$\rightarrow$D and ABCD$\rightarrow$D tasks, significantly outperforming previous approaches and demonstrating strong capabilities in multi-task learning and long-horizon planning.
\vspace{-2mm}
\paragraph{LIBERO Simulation Evaluation.}
Following~\cite{zhao2025cot}, we report the average success rate over 500 episodes for each task suite (Spatial, Object, Goal, Long). As shown in Table~\ref{tab:libero}, \methodname{} achieves the best overall performance across all LIBERO benchmark suites, with particularly significant gains on long-horizon tasks—improving the previous state of the art from 69.0\% to 94.0\%. Compared to $\pi_0$~\cite{pertsch2025fast}, our method demonstrates superior performance on long-horizon tasks.
\begin{table}[ht]
\centering
\caption{\textbf{Comparison of different methods on the LIBERO benchmark.}}
\resizebox{0.68\textwidth}{!}{
\begin{tabular}{lccccc}
\toprule
\textbf{Method} & \textbf{SPATIAL} & \textbf{OBJECT} & \textbf{GOAL} & \textbf{LONG} & \textbf{Average} \\
\midrule
DP*~\cite{chi2023diffusion}         & 78.3\% & 92.5\% & 68.3\% & 50.5\% & 72.4\% \\
Octo~\cite{team2024octo}      & 78.9\% & 85.7\% & 84.6\% & 51.1\% & 75.1\% \\
OpenVLA~\cite{kim2024openvla}& 84.9\% & 88.4\% & 79.2\% & 53.7\% & 76.5\% \\
SpatialVLA~\cite{qu2025spatialvla} & 88.2\% & 89.9\% & 78.6\% & 55.5\% & 78.1\% \\
CoT-VLA~\cite{zhao2025cot} & 87.5\% & 91.6\% & 87.6\% & 69.0\% & 81.1\%\\
$\pi_0$-FAST~\cite{pertsch2025fast} &\textbf{96.4\%}	& 96.8\%&	88.6\%&	60.2\%&	85.5\% \\
% \rowcolor[gray]{0.9} \textbf{\methodname{}} &\textbf{96.6\%} & \textbf{99.4\%} & \textbf{91.6\%} & \textbf{89.2\%} & \textbf{94.2\%}\\
\rowcolor[gray]{0.9} \textbf{\methodname{}} &95.4\% & \textbf{98.8\%} & \textbf{93.6\%} & \textbf{94.0\%} & \textbf{95.5\%}\\
\bottomrule
\end{tabular}}
\label{tab:libero}
\vspace{-3mm}
\end{table}

\vspace{-2mm}
\paragraph{SimplerEnv Simulation Evaluation.}
Table~\ref{tab:simplerenv_bridge} summarizes the performance across various manipulation policies on the Bridge-WidowX setup. Our approach demonstrates a significant improvement over prior methods, raising the average success rate from 42.7\% to 69.8\%. In particular, it shows marked improvements on previously difficult tasks, including stack block, put carrot and put spoon.
\begin{table}[htbp]
\centering
\caption{\textbf{Evaluation on SimplerEnv-WidowX across various manipulation tasks.}}
\label{tab:simplerenv_bridge}
\resizebox{\textwidth}{!}{
\begin{tabular}{l|cc|cc|cc|cc|c}
\toprule
\multirow{2}{*}{\textbf{Model}} 
& \multicolumn{2}{c|}{\textbf{Put Spoon on Towel}} 
& \multicolumn{2}{c|}{\textbf{Put Carrot on Plate}} 
& \multicolumn{2}{c|}{\textbf{Stack Green on Yellow Block}} 
& \multicolumn{2}{c|}{\textbf{Put Eggplant in Yellow Basket}} 
& \textbf{Overall} \\
\cmidrule(lr){2-3} \cmidrule(lr){4-5} \cmidrule(lr){6-7} \cmidrule(lr){8-9}
& \multicolumn{1}{c}{Grasp} & \multicolumn{1}{c|}{Success} 
& \multicolumn{1}{c}{Grasp} & \multicolumn{1}{c|}{Success} 
& \multicolumn{1}{c}{Grasp} & \multicolumn{1}{c|}{Success} 
& \multicolumn{1}{c}{Grasp} & \multicolumn{1}{c|}{Success} 
& \multicolumn{1}{c}{Success} \\
\midrule
RT-1-X~\cite{brohan2023rt}     
& 16.7\% & 0.0\%   
& 20.8\% & 4.2\%   
& 8.3\%  & 0.0\%   
& 0.0\%  & 0.0\%   
& 1.1\% \\
Octo-Base~\cite{octo_2023} 
& 34.7\% & 12.5\%  
& 52.8\% & 8.3\%   
& 31.9\% & 0.0\%   
& 66.7\% & 43.1\%  
& 16.0\% \\
Octo-Small~\cite{octo_2023}
& 77.8\% & 47.2\%  
& 27.8\% & 9.7\%   
& 40.3\% & 4.2\%   
& 87.5\% & 56.9\%  
& 29.5\% \\
OpenVLA~\cite{kim2024openvla}  
& 4.1\%  & 0.0\%   
& 33.3\% & 0.0\%   
& 12.5\% & 0.0\%   
& 8.3\%  & 4.1\%   
& 1.0\% \\
RoboVLMs~\cite{li2024towards}
& 70.8\% & 45.8\%  
& 33.3\% & 20.8\%  
& 54.2\% & 4.2\%  
& 91.7\% & 79.2%  
& 37.5\% \\
SpatialVLA~\cite{qu2025spatialvla} 
& 20.8\% & 16.7\% 
& 29.2\% & 25.0\% 
& 62.5\% & 29.2\% 
& 100\% & \textbf{100\%} 
& 42.7\% \\
% \rowcolor[gray]{0.9} \textbf{\methodname{}} 
% & \textbf{91.7\%} & \textbf{70.8\%} 
% & \textbf{66.7\%} & \textbf{58.3\%} 
% & \textbf{79.2\%} & \textbf{33.3\%} 
% & 95.8\% & 95.8\%  
% & \textbf{64.6\%} \\
\rowcolor[gray]{0.9} \textbf{\methodname{}} 
& \textbf{83.3\%} & \textbf{83.3\%} 
& \textbf{74.0\%} & \textbf{66.7\%} 
& \textbf{95.8\%} & \textbf{33.3\%} 
& \textbf{100.0\%} & 95.8\%  
& \textbf{69.8\%} \\
\bottomrule
\end{tabular}}
\vspace{-3mm}
\end{table}

% \input{tables/simplerenv_google}
% -----------------------------------------------------------------------
\subsection{In-Depth Analysis}
In this section, we provide an in-depth analysis within our unified framework, which may offer key insights for the design of future VLA models.
We first analyze how post-training enhances downstream policy learning in terms of both performance (Table~\ref{abl:post_train}) and training efficiency (Table~\ref{tab:post_efficiency}), highlighting the potential of world models as a general post-training strategy for robotics.
We then investigate that even without post-training stage, incorporating visual prediction loss (Table~\ref{tab:calvin_abl_visual}) and historical context (Table~\ref{tab:calvin_abl_history}) still contributes positively to policy learning. 
\vspace{-2mm}
\paragraph{Effectiveness of World Model Post-Training.}
Table~\ref{abl:post_train} investigates the effects of different post-training strategies on downstream policy learning across various simulation benchmarks. The results reveal that, due to inconsistencies in the action space across tasks, action-only learning exhibits low transferability, leading to a negative impact on performance. In contrast, most post-training approaches significantly enhance policy learning, highlighting the crucial role of visual learning in transferability.
Among these, the world model post-training approach yields the most substantial gains, enhancing both generalization and long-horizon planning capabilities.
A comparison with text-to-image (T2I) training emphasizes the importance of modeling temporal dynamics in video data, while contrasting with video-only training highlights the essential role of textual guidance in state transitions. Notably, this world model training requires no action annotations, enabling scalable learning from large-scale video data and providing a promising direction for future VLA research.
\definecolor{DeepGreen}{rgb}{0.0, 0.8, 0.0}

\begin{table}[tbp]
\centering
\caption{\textbf{Effectiveness of World Model Post-Training.} We compare different post-training strategies by fine-tuning only with action prediction on the downstream benchmarks.}
\resizebox{\linewidth}{!}{
\begin{tabular}{ccc|cc|cc}
\toprule
\multicolumn{3}{c|}{\textbf{Post-training Stage}} & \multicolumn{2}{c|}{\textbf{Generalization}} & \multicolumn{2}{c}{\textbf{Long-horizon}} \\
\textbf{Strategy} & \textbf{Sequence} & \textbf{Supervision} & \textbf{LIBERO} & \textbf{SimplerEnv-WidowX} & \textbf{LIBERO-Long} & \textbf{CALVIN} \\
\midrule
& & &  48.5 & 0.0 & 17.4 & 1.46 \\
action prediction & $T, I, A$ & action &43.9 (\textcolor{red}{-4.6}) & 0.0 & 10.6 (\textcolor{red}{-6.8})& 0.52(\textcolor{red}{-0.94})\\
text-to-image    & $T,I$ & vision & 69.8 (\textcolor{DeepGreen}{+21.3}) & 6.3 (\textcolor{DeepGreen}{+6.3}) & 55.8 (\textcolor{DeepGreen}{+38.4}) & 3.79 (\textcolor{DeepGreen}{+2.33}) \\
video prediction & $I_1,...,I_t$ & vision & 78.9 (\textcolor{DeepGreen}{+30.4})& 17.7 (\textcolor{DeepGreen}{+17.7}) & 80.8 (\textcolor{DeepGreen}{+63.4}) & 3.59 (\textcolor{DeepGreen}{+2.13}) \\
\rowcolor[gray]{0.9} world model      & $T,I_1,...,I_t$ & vision & \textbf{94.2} (\textcolor{DeepGreen}{+45.7}) & \textbf{64.6} (\textcolor{DeepGreen}{+64.6}) & \textbf{89.2} (\textcolor{DeepGreen}{+71.8}) & \textbf{4.61} (\textcolor{DeepGreen}{+3.15}) \\
\bottomrule
\end{tabular}}
\label{abl:post_train}
\vspace{-5mm}
\end{table}

\vspace{-2mm}
\paragraph{Data and Training Efficiency.}
Table~\ref{tab:post_efficiency} shows that post-training substantially enhances downstream policy learning efficiency. On the CALVIN benchmark (Table~\ref{tab:data_efficiency}), our method achieves higher success rates using only 10\% of the fine-tuning data, outperforming prior approaches such as GR-1~\cite{wu2023unleashing} and RoboVLMs~\cite{li2024towards}. In addition, Table~\ref{tab:train_efficiency} highlights improved training efficiency, as the model rapidly converges with fewer fine-tuning iterations. The Simpler-Env results further demonstrate the effectiveness of world-model-based post-training for efficient policy adaptation across diverse robotic setups.
While similar effects are observed in latent-action methods~\cite{ye2024latent,chen2024moto, gao2025adaworld}, our world model offers a simpler paradigm without latent actions, achieving better transferability.
\begin{table*}[htbp]
\centering
\caption{\textbf{Post-training enables data-efficient and training-efficient downstream policy learning.}}
\label{tab:post_efficiency}
\begin{subtable}[t]{0.48\textwidth}
\centering
\caption{\textbf{Data efficiency comparison.}}
\resizebox{\linewidth}{!}{%
\begin{tabular}{lcc}
\toprule
\textbf{Method} & \textbf{Data} & \textbf{CALVIN} \\
\midrule
RT-1~\cite{brohan2022rt} & 10\% & 0.34\\
MT-R3M~\cite{nair2022r3m} & 10\% & 0.61\\
HULC~\cite{mees2022matters}  & 10\% & 1.11\\
GR-1~\cite{wu2023unleashing} & 10\% & 2.00\\
RoboVLMS~\cite{li2024towards} & 10\% & 2.52\\
\midrule
\textbf{\methodname{}} (w/o post-train) & 10\% & 0.15 \\
\rowcolor[gray]{0.9}\textbf{\methodname{}} & 10\% & \textbf{3.19} \\
\bottomrule
\end{tabular}
}
\label{tab:data_efficiency}
\end{subtable}
\hfill
\begin{subtable}[t]{0.48\textwidth}
\centering
\caption{\textbf{Training efficiency comparison.}}
\resizebox{\linewidth}{!}{
\begin{tabular}{lccc}
\toprule
\multicolumn{4}{c}{\textbf{Fast convergence (CALVIN)}} \\
Training Iters & 2k & 4k & 8k  \\
\midrule
w/o post-train & 0.37 & 0.82 & 1.46  \\
\rowcolor[gray]{0.9}w/ post-train    &4.21 & 4.56 & 4.61    \\
\midrule
\multicolumn{4}{c}{\textbf{Fast adaptation (SimplerEnv-Bridge)}} \\
Method & Batch size & Iters & Success \\
\midrule
RoboVLMs~\cite{li2024towards} & 128   &50k & 37.5 \\
\rowcolor[gray]{0.9}\textbf{\methodname{}  }    & 128 &12k& 64.6 \\
\bottomrule
\end{tabular}
}
\label{tab:train_efficiency}
\end{subtable}
\vspace{-5mm}
\end{table*}

\begin{table*}[htbp]
    \centering
    \caption{\textbf{Ablation study on the visual prediction and historical context in policy learning.}}
    \begin{subtable}[t]{0.52\linewidth}
        \centering
        \caption{\textbf{Effectiveness of visual prediction.}}
        \resizebox{\linewidth}{!}{%
        \begin{tabular}{cccc}
            \toprule
            \textbf{Post-train} & \textbf{Visual prediction} & \textbf{CALVIN} & \textbf{LIBERO} \\
            \midrule
            \checkmark & & 4.61 & 94.2\\
            \rowcolor[gray]{0.9} & \checkmark & 4.42 & 88.7 \\
             & & 1.46 & 48.5 \\
            \bottomrule
        \end{tabular}
        }
        \label{tab:calvin_abl_visual}
    \end{subtable}
    \hfill
    \begin{subtable}[t]{0.46\linewidth}
        \centering
        \caption{\textbf{Effectiveness of history context.}}
        \resizebox{\linewidth}{!}{%
        \begin{tabular}{ccc}
        \toprule
        \multicolumn{2}{c}{\textbf{Observations}} & \textbf{Avg. Len $\uparrow$} \\
        \cmidrule(lr){1-2}
        \textbf{History Window} & \textbf{Current + History} & \\
        \midrule
        0 & 1 + 0 & 4.26 \\
        \rowcolor[gray]{0.9}10 & 1 + 1 & 4.61 \\
        10 & 1 + 2 & 4.43 \\
        20 & 1 + 2 & 4.47 \\
        \bottomrule
\end{tabular}
        }
        \label{tab:calvin_abl_history}
    \end{subtable}
\end{table*}

\vspace{-2mm}
\paragraph{Effectiveness of Visual Prediction.}
While post-training proves effective, it is also crucial that the model demonstrates strong performance without relying on it. As shown in Table~\ref{tab:calvin_abl_visual}, our findings indicate that, even without post-training, fine-tuning with visual loss supervision—leveraging the autoregressive nature of the model—naturally integrates world model learning into the policy learning process. This approach leads to a significant improvement in the model's performance.
\vspace{-2mm}
\paragraph{Effectiveness of History Context.}
History context—comprising past observations and actions—provides valuable guidance for robot planning. In this section, we investigate the appropriate length of the history window during the fine-tuning stage. As shown in Table~\ref{tab:calvin_abl_history}, our ablation study on the CALVIN benchmark examines the impact of varying history window lengths. Incorporating a history window significantly improves performance (from 4.26 to 4.61). However, extending the window beyond a certain length yields diminishing returns, suggesting that recent observations carry the most predictive value, consistent with the Markov property in sequential planning.

\subsection{Multimodal Capability}
As illustrated in Figure~\ref{fig:visual_demo}, we qualitatively showcase the model’s ability to interleave multiple modalities—action, language, and vision—within a unified framework. This design enables policy learning for embodied control, spatial reasoning through language output, and future state prediction via visual output, highlighting the model’s capacity for generalizable multimodal understanding.
\begin{figure}[tbp]
    \centering
    \includegraphics[width=\textwidth]{images/visual_demo.pdf}
    \vspace{-5mm}
    \caption{\textbf{Multimodal capabilities of \methodname{}}. Top: Action outputs for executing long-horizon tasks in the LIBERO benchmark. Bottom: Visual predictions and spatial grounding demonstrating the model's spatiotemporal understanding. The red box marks the current observation; green boxes indicate predicted object detections.}
    \label{fig:visual_demo}
    \vspace{-4mm}
\end{figure}

\subsection{Broader Applications}
\vspace{-1mm}
\paragraph{End-to-end Learning for Autonomous Driving.}
To further explore the potential of our method, we perform a preliminary transfer to the autonomous driving domain by finetuning the model on the NAVSIM benchmark. Notably, our method is a pure autoregressive, token-based framework, modeling the driving task as causal sequence prediction over discretized multimodal tokens. Despite using only front-view camera inputs—without relying on BEV representations or multi-sensor fusion—our model achieves powerful performance on the NAVSIM test set. Notably, the current performance is not pretrained on driving videos but is only fine-tuned on downstream policy benchmarks. These results highlight the strong potential of our method for broader real-world applications.
\begin{table}[htbp]
  \centering
    \caption{\textbf{Broader applications of \methodname{} for end-to-end autonomous driving on the NAVSIM.} MC: Multi Camera. L: LiDAR. FC: Front Camera.
    }
  \resizebox{0.88\linewidth}{!}{
  \begin{tabular}{lccccccc>{\columncolor{gray!20}}c}
    \toprule
    \textbf{Method} & \textbf{Model} & \textbf{Input} & \textbf{NC$\uparrow$} & \textbf{DAC$\uparrow$} & \textbf{EP$\uparrow$} & \textbf{TTC$\uparrow$} & \textbf{C$\uparrow$} & \textbf{PDMS$\uparrow$} \\
    \midrule
    \textcolor{gray}{Human} & \textcolor{gray}{--} & \textcolor{gray}{--} & \textcolor{gray}{100.0} & \textcolor{gray}{100.0} & \textcolor{gray}{87.5} & \textcolor{gray}{100.0} & \textcolor{gray}{99.9} & \textcolor{gray}{94.8} \\
    \midrule
    Ego Status MLP & -- & Ego State & 93.0 & 77.3 & 62.8 & 83.6 & 100.0 & 65.6 \\
    VADv2~\cite{vadv2}          & BEV-based & MC & 97.9 & 91.7 & 77.6 & 92.9 & 100.0 & 83.0 \\
    UniAD~\cite{uniad}          & BEV-based & MC & 97.8 & 91.9 & 78.8 & 92.9 & 100.0 & 83.4 \\
    Transfuser~\cite{Transfuser}     & BEV-based & MC\&L & 97.7 & 92.8 & 79.2 & 92.8 & 100.0 & 84.0 \\
    \midrule
    \textbf{\methodname{}} & Auto-regressive & FC & 96.9 & 91.1 & 76.8 & 91.7 & 96.7 & 81.7 \\
    \bottomrule
  \end{tabular}}
  \label{tab:driving_navsim}
  \vspace{-5mm}
\end{table}



\section{Related Work}

\textbf{Foundation Models in Robotics.} 
Developing and using foundation models~\citep{bommasani2021opportunities} for robotics has been of great interest recently. 
% non-VLA work that uses LLMs/VLMs 
One common approach is to leverage existing pre-trained foundation models as high-level black-box reasoning modules in conjunction with low-level robot-specific policies~\citep{saycan-2022, innermono-2022, progprompt-2022, driess2023palm, code-as-policies-2022, text2motion-23, groundeddecoding-2024}. This approach allows the robot to plan sequences of low-level skills or motions using the pre-trained foundation model. However, it assumes the availability of these low-level policies and a sufficient interface to connect them to the black-box foundation models. 
% one core limitation of this approach is a lack of end-to-end optimization for the downstream robotics task. 
% VLA work
An alternative approach is to finetune pre-trained foundation models on robotics data to build Vision-Language-Action (VLA) models~\citep{rt1-2022, rt22023arxiv, black2024pi_0, kim24openvla, zheng2025tracevla, wen2024tinyvla, cheang2024gr2vla, li2023vision, zhen20243dvla, huang2023embodied, ye2025latent, yang2025magma}. Instead of enforcing a rigid hierarchy between high-level VLM planning and low-level control, these VLA models allow for end-to-end optimization toward the downstream deployment tasks. We take a similar approach to train \modelname{} and use the Eagle-2 model~\citep{eagle2} as our base Vision Language Model (VLM). We fine-tune our VLM together with a flow-matching~\citep{flowmatching, liu2022flow, hu2024adaflow} action generation model with action chunking~\citep{action_chunking}. In contrast to prior VLA models \citep{black2024pi_0} that use a mixture-of-experts architecture to bridge the base VLM model with the action generation model, we use a simple cross-attention mechanism. This approach provides flexibility regarding the exact architecture of the VLM model and the action generation model we can use. Furthermore, we use embodiment-specific state and action projector modules, which support different robot embodiments, including latent~\citep{ye2025latent} and IDM-based~\citep{baker2022video} actions. The use of these projectors is similar to those in \citet{octo_2023}, though that work did not fine-tune the VLM models. 



\textbf{Datasets for Robot Learning.} 
% Collecting large-scale datasets in robotics is a challenging endeavor, but there have been several prior attempts to address this problem. 
A core challenge in robot learning is the scarcity of large-scale, diverse, and embodied datasets necessary to train generalist robots. 
% TODO: we could consider discussing self-supervised approaches here (e.g. RL)
% Real robot teleoperation, large-scale datasets
One common approach is to use robot teleoperation~\citep{zhang2017deep, mandlekar2018roboturk, mandlekar2019scaling, mandlekar2020human, wu2023gello, action_chunking, aldaco2024aloha, fu2024mobile, iyer2024open, dass2024telemoma}, where a human uses a device such as a smartphone or Virtual Reality (VR) controller, to control a robot to perform tasks of interest. 
The robot sensor streams and robot controls during operation are logged to a dataset, allowing for high-quality task demonstrations to be collected.
Recently, this approach has been scaled by utilizing large teams of human operators and robot fleets over extended periods of time (\eg, months), resulting in large-scale robot manipulation datasets with thousands of hours of demonstrations~\citep{ebert2021bridge, brohan2022rt, ahn2022can, lynch2022interactive, o2024open, contributors2025agibotworld, black2024pi_0}. 
However, collecting data this way requires extensive cost and human effort.
Another line of work, instrumented human demonstrations, uses special hardware to capture robot-relevant observation and action data without explicitly teleoperating the target robot. For example, \cite{chi2024universal, seo2024legato} use hand-held robot grippers, \citet{fang2024airexo} uses a robot-like exoskeleton, and \cite{kareer2024egomimicscalingimitationlearning} uses special glasses to capture human hand motions, which are retargeted to robot action data. These approaches tend to result in faster data collection, though they have a mismatch with the downstream robot compared to direct robot teleoperation.
% % Simulation, data generation
% There are other compelling alternatives to the burden of real-world on-robot data collection.
% Recently, several works~\cite{mandlekar2023mimicgen, james2020rlbench, dalal2023imitating, gu2023maniskill2, ha2023scaling, robocasa2024, jiang2024dexmimicen, wang2023robogen, garrett2024skillmimicgen, yang2025physics} have proposed automated data generation pipelines that can leverage simulation to produce thousands of task demonstrations with minimal human effort. 
% This makes it easy to generate large-scale datasets; however, utilizing these datasets can be challenging due to the simulation-to-reality gap.
% % TODO: If we want to cite robot-free data collection like UMI, this could be one place to do it.
% % Dream data
% Another promising avenue is to leverage advances in generative models, such as video generation models, to augment existing sets of robot demonstrations~\cite{agarwal2025cosmos, mandi2022cacti, yu2023scaling, chen2023genaug}. For example, given a language prompt and an initial camera frame that shows a robot workspace, a video generation model could create new video demonstrations for the robot to learn from. Similarly, such a model could be applied to augment the visuals in existing task demonstrations.
% Human video
A separate line of work makes use of human video datasets~\citep{grauman2024ego, grauman2022ego4d, goyal2017something, damen2018scaling, miech2019howto100m}, which are plentiful and substantially easier to collect than on-robot data, as a source of training data for robots. 
Some works~\citep{nair2022r3m, wu2023unleashing, karamcheti2023language} use human video datasets to pre-train representations that are then used as a feature space for training policies on downstream robot datasets. Other works~\cite{bharadhwaj2024gen2act, bharadhwaj2024track2act, ren2025motion} try to jointly use human video data and robot data through intermediate representations for the motions in the video. \citet{ye2025latent} shows that pretraining VLAs with \textit{latent} actions only on human videos yields positive transfer to downstream robotic tasks.
% Prior works~\citep{nair2022r3m, wu2023unleashing, karamcheti2023language, bharadhwaj2024gen2act, bharadhwaj2024track2act, ren2025motion} have attempted to use human videos~\citep{grauman2024ego, grauman2022ego4d, goyal2017something, damen2018scaling, miech2019howto100m} as a source of training data for robots~\checkthis{make this sentence crispier - \textit{how} they use human videos}. 
% While this is a rich source of data, making use of the data can be difficult due to the embodiment gap between humans and robots. In this work, we use embodiment-specific state and action projector models to enable training on these different embodiments.
% \checkthis{also mention instrumented human data like UMI, LEGATO, EgoMimic? KL: added above}
% Model tries to combine all of them together
Rather than relying on a single type of training data, we developed techniques to effectively learn from a diverse assortment of real-world robot data, human video data, and synthetic data.


\textbf{Synthetic Data Generation in Robotics.} 
% Simulation, data generation
Real-world robot data collection requires large amounts of time and considerable human cost. By contrast, data collection in simulation can be substantially more efficient and less painful, making it a compelling alternative. 
% There are other compelling alternatives to the burden of real-world on-robot data collection.
Recently, several works~\citep{mandlekar2023mimicgen, james2020rlbench, dalal2023imitating, gu2023maniskill2, ha2023scaling, robocasa2024, jiang2024dexmimicen, wang2023robogen, garrett2024skillmimicgen, yang2025physics} have proposed automated data generation pipelines that can leverage simulation to produce thousands of task demonstrations with minimal human effort. 
This makes it easy to generate large-scale datasets; however, utilizing these datasets can be challenging due to the simulation-to-reality gap.
% TODO: If we want to cite robot-free data collection like UMI, this could be one place to do it.
% Dream data

Another promising avenue has been using neural generative models to augment existing sets of robot demonstrations~\citep{mandi2022cacti, yu2023scaling, chen2023genaug}. However, previous work have been limited to utilizing in-painting or text-to-image diffusion models to augment the training data. In our work, we leverage the recent advancements in video generative models~\citep{agarwal2025cosmos, wan2.1} to create entire neural trajectories, at a scale that has never been explored before: $\sim$300k neural trajectories which amounts to 827 hours of robot trajectories.

In our model, we make use of large synthetic simulation datasets generated by MimicGen~\citep{mandlekar2023mimicgen} and DexMimicGen~\citep{jiang2024dexmimicen}, as well as neural-generated video datasets with state-of-the-art video generation models. Our way of co-training with synthetically generated and real-world data sets us from other large-scale VLA efforts. 
% In our model, we make use of both synthetically generated simulation data, and data from generative models.~\checkthis{be more concrete how we generate synthetic data? how we do things differently?}


\section{Conclusion}
In this paper, we present \methodname{}, a unified framework for vision–language–action modeling that bridges heterogeneous modalities through a shared token space and models them autoregressively. The proposed unified design facilitates deeper cross-modal integration and inherently supports flexible multimodal tasks. By leveraging a world model trained to capture dynamics and causality from videos, we observe significant improvements in downstream policy learning, both in terms of performance and efficiency. Extensive simulation experiments further demonstrate the model’s strong generalization ability, efficient policy learning, and broad applicability across diverse domains.
These findings highlight the great potential of our method as a new paradigm for vision–language–action modeling.
\vspace{-2mm}
\paragraph{Limitations and Future Work.}
Due to limited computational resources, our investigation into post-training scalability is still in its early stages. Nonetheless, initial results are promising and indicate potential for scaling to larger video datasets. Furthermore, while the unified multimodal framework exhibits strong capabilities in cross-modal learning, further research is needed to fully integrate it with reinforcement learning paradigms, enabling more robust and adaptive policy learning.

\bibliography{neurips_2023}
\bibliographystyle{plain}

\input{content/checklist}
\setcounter{table}{0}
\renewcommand{\thetable}{A\arabic{table}}
\setcounter{figure}{0}
\renewcommand{\thefigure}{A\arabic{figure}}
\renewcommand\theHtable{Appendix.\thetable}


\section{Text Prompts}
Below we listed the text prompts we used for adaptation and task evaluation.

\begin{table}[h]
\centering
\small

\setlength{\tabcolsep}{4.8pt}
\scalebox{0.8}{

\begin{tabular}{@{}llll@{}}
\toprule
Task             & In-Domain Prompts                         & AnimateDiff Prompts                                        & DreamBooth Identifier                 \\ \midrule
Dog Walking      & a dog/pharaoh hound walking          & a dog/pharaoh hound walking                                & a {[}D{]} dog                         \\
Humanoid Walking & a(n) humanoid/action figure  walking & a(n) humanoid/action figure walking                        & a {[}D{]} action figure               \\
\midrule
Assembly$^*$     & assembly                         & a robot arm placing a ring over a peg                               & \multirow{14}{*}{a {[}D{]} robot arm} \\
Dial Turn$^*$     & dial turn                         & a robot arm turning a dial                               &                                       \\
Reach$^*$     & reach                         & a robot arm reaching a red sphere                               &                                       \\
Peg Unplug Side$^*$     & peg unplug side                         & a robot arm unplugging a gray peg                               &                                       \\
Lever Pull$^*$       & lever pull                           & a robot arm pulling a lever                                &                                       \\ 
Coffee Push$^*$      & coffee push                          & a robot arm pushing a white cup towards a coffee machine &                                       \\
Door Close$^*$       & door close                           & a robot arm closing a door                                 &                                   \\
Door Open        & door open                            & a robot arm opening a door                                 &                                       \\
Window Close     & window close                         & a robot arm closing a window                               &                                       \\
Window Open      & window open                          & a robot arm opening a window                               &                                       \\
Drawer Close     & drawer close                         & a robot arm closing a drawer                               &                                       \\
Drawer Open      & drawer open                          & a robot arm open a drawer                                  &                                       \\
Soccer           & soccer                               & a robot arm pushing a soccer ball into the net             &                                       \\
Button Press     & button press                         & a robot arm pushing a button                               &                                       \\\bottomrule
\end{tabular}
}

\caption{\textbf{Task-Prompt Pairs.} We include a comprehensive list of tasks and their text prompts for adaptation and evaluation. ``$*$'' denotes tasks seen during adaptation.}
\label{table:text_prompts}
\end{table}

\begin{table}[h]
\centering
\small

\setlength{\tabcolsep}{4.8pt}

\begin{tabular}{@{}lll@{}}
\toprule
Task             & In-Domain Prompts                         & AnimateDiff Prompts           \\ \midrule
Spatula in Kitchen$^{*}$      & spatula                      & find the spatula              \\
Toaster in Kitchen$^{*}$       & toaster                   & find the toaster                    \\
Painting in Living Room$^*$     & painting                    & find the painting             \\
Blinds in Bedroom$^*$           & blinds                         & find the blinds             \\
ToiletPaper in Bathroom$^*$     & toilet paper                         & find the toilet paper   \\
Pillow in Living Room          & pillow                         & find the pillow           \\
DeskLamp in Living Room       & desk lamp                           & find the desk lamp     \\ 
Mirror in Bedroom               & mirror                          & find the mirror         \\
Laptop in Bedroom             & laptop                           & find the laptop           \\
\bottomrule
\end{tabular}


\caption{\textbf{Task-Prompt Pairs for iTHOR.} We include a comprehensive list of iTHOR tasks and their text prompts for adaptation and evaluation. ``$*$'' denotes tasks seen during adaptation.}
\label{table:text_prompts_ithor}
\end{table}








\section{Continued Denoising}
\label{sec:cont_denoising}

In diffusion-based policy supervision, rewards are extracted from the procedure of corrupting frames achieved by the policy with some level of Gaussian noise and then making denoising predictions using the video model~\citep{huang2023diffusion, luo2024text}.  For additional insight, we propose a visualization technique called \textbf{\textit{continued denoising}}, and report FVD scores for videos generated in such a manner.  In continued denoising, rather than extracting a scalar from components of the denoising prediction as in Video-TADPoLe, we treat the noised video as an initialization and iteratively continue sampling to produce a final clean video prediction - thus, “continuing” the denoising procedure.  In our experiments we perform continued denoising conditioned on a desired text prompt, a noise level of 700, a total frame length of 16, and 10 denoising steps.

As mentioned in Section~\ref{subsec:policy_supervision_method}, policy supervision does not necessarily require strong free-form generation of in-domain videos; rather it evaluates observed frames achieved by following the current policy.  For qualitative purposes, continued denoising provides us a visual sense of how this evaluation of achieved frames is done (examples in Figure~\ref{fig:mw_continued_denoising_400_unseen}), as well as a sanity check on the integration of in-domain information through adaptation.  Furthermore, it enables quantitative comparison through FVD scores, which provides an idea on the capability of adapted video models to reconstruct in-domain-like videos conditioned on text.  It is intuitive to hypothesize that a lower FVD score correlates with better in-domain adaptation, as it understands how to accurately complete the provided in-domain frames from a heavy noise corruption.

In Table~\ref{table:fvd_scores_contd}, we report the FVD scores for the same set of seen and unseen tasks that are evaluated in free-form generation experiments. 
We discover that the lowest FVD score for continued denoising is achieved by the in-domain model, which is unsurprising as it was explicitly trained on such examples.  The next-best FVD scores are achieved by probabilistic adaptation and its inverse.  This is significant because it supports the finding that with adaptation, generalization to unseen tasks is possible, and suggests that accurate domain-specific rewards can be supplied through policy supervision.  Indeed, this aligns with our result in Table~\ref{table:mw_videotadpole}, where Inverse Probabilistic Adaptation achieves the best overall task performance through policy supervision. 

\begin{table}[h]
\centering
\setlength{\tabcolsep}{4.8pt}
\resizebox{\textwidth}{!}{
\begin{tabular}{@{}lcccccc@{}}
\toprule
FVD Scores (MetaWorld) & Vanilla AnimateDiff & In-Domain-Only & Direct Finetuning & Subject Customization & Prob. Adaptation & Inverse Prob. Adaptation \\ \midrule
Seen                   & 2700.4              & \textbf{602.8}          & 1004.6            & 1078.9                & 622.6            & 627.4                    \\
Unseen                 & 2643.2              & \textbf{610.1}          & 978.5             & 1711.8                & 630.6            & 681.8                    \\ \bottomrule
\end{tabular}
}
\caption[]{\textbf{FVD Scores with Continued Denoising.} We report FVD scores for videos of MetaWorld tasks, produced by Continued Denoising via the video generative models of interest. This is computed for both seen and unseen task sets, each with 7 tasks, aggregating results over 1000 synthetic videos.}
\label{table:fvd_scores_contd}
\end{table}

\begin{figure}[h]
    \centering
    \includegraphics[width=\linewidth]{figures/mw_continued_denoising_400_unseen.pdf}
    \vspace{-15pt}
    \caption{\textbf{Continued Denoising.}  We visualize frames from a task unseen during adaptation, corrupted with a level of Gaussian noise (top row).  We then show the result of continued denoising using an inverse probabilistic adaptation model to verify it can visually generalize to fill in novel in-domain information.  Despite not having seen a button, it is able to reconstruct it conditioned on text.  This figure is for intuition; in practice, a much higher noise level is used, shown in Figure~\ref{fig:mw_continued_denoising_700_unseen}.}
    \label{fig:mw_continued_denoising_400_unseen}
    \vspace{-10pt}
\end{figure}


\begin{figure}[h]
    \centering
    \includegraphics[width=\linewidth]{figures/mw_continued_denoising_700_unseen.pdf}
    \vspace{-1em}
    \caption{\textbf{Continued Denoising (in practice).}  In practice, an aggressive level of Gaussian corruption is usually used on achieved frames for reward computation (700 for MetaWorld).  However, because to the human eye this may look virtually indistinguishable from pure noise, we supply an illustrative example in Figure~\ref{fig:mw_continued_denoising_400_unseen} using a noise level of 400.  Here, we showcase visuals of the same unseen task corrupted with a practical noise level of 700.  We then show the result of continued denoising to visually verify the model integrates adapted in-domain information successfully.  When performing continued denoising from such a high corruption, conditioned on the text prompt ``a robot arm pushing a button”, it is therefore quite surprising the level of detail with which the adapted text-to-video model is able to reconstruct novel in-domain features such as the button - which it has not even seen during adaptation.  The resulting continued denoising video can also be evaluated against in-domain examples via FVD for further insights.}
    \label{fig:mw_continued_denoising_700_unseen}
    \vspace{-1em}
\end{figure}

\section{Video-TADPoLe Reward Computation}
\label{sec:videotadpole_equations}

Video-TADPoLe~\citep{luo2024text} rewards are densely computed for a trajectory achieved by a policy, in terms of their rendered frames.  For arbitrary start index $i$ and end index $j$ inclusive of the trajectory, for $i \leq j$, let $\mathbf{o}_{[i+1:j+1]}$  denote the associated sequence of rendered frames.  Video-TADPoLe then utilizes a source noise vector $\boldsymbol{\epsilon}_0 \sim \mathcal{N}(\boldsymbol{\epsilon};\boldsymbol{0}, \textbf{I}_{j-i+1})$ of the same dimensionality as a Gaussian corruption to produce noisy observation $\tilde{\mathbf{o}}_{[i+1:j+1]}$.  Then, Video-TADPoLe computes a batch of \textit{alignment reward} terms through one inference step of the text-to-video diffusion model as:
$$r_{[i:j]}^\text{align} = \left\lVert\boldsymbol{\hat{\epsilon}}_{\boldsymbol{\phi}}(\tilde{\boldsymbol{o}}_{[i+1:j+1]}; \texttt{t}_\text{noise}, y) - \boldsymbol{\hat{\epsilon}}_{\boldsymbol{\phi}}(\tilde{\boldsymbol{o}}_{[i+1:j+1]}; \texttt{t}_\text{noise})\right\rVert_2^2,$$
and a batch of \textit{reconstruction reward} terms as:
$$r_{[i:j]}^\text{rec} = \left\lVert\boldsymbol{\hat{\epsilon}}_{\boldsymbol{\phi}}(\tilde{\boldsymbol{o}}_{[i+1:j+1]}; \texttt{t}_\text{noise}) - \boldsymbol{\epsilon}_0\right\rVert_2^2 - \left\lVert\boldsymbol{\hat{\epsilon}}_{\boldsymbol{\phi}}(\tilde{\boldsymbol{o}}_{[i+1:j+1]}; \texttt{t}_\text{noise}, y) - \boldsymbol{\epsilon}_0\right\rVert_2^2.$$
For a provided context window of size $n$, Video-TADPoLe calculates the reward at each timestep $t$ utilizing each context window that involves achieved observation $\mathbf{o}_{t+1}$:
$$r_t = \frac{1}{n}\sum_{i = 1}^{n}\texttt{symlog}\left(w_1*r_{[t-i+1:t-i+n]}^\text{align}[i-1]\right) + \texttt{symlog}\left(w_2 * r_{[t-i+1:t-i+n]}^\text{rec}[i-1]\right).$$
A stride term $s$ can be used to make this computation tractable across long trajectories, where the context window skips $s$ timesteps before computing a sequence of Video-TADPoLe rewards again.  The context window $n$, stride $s$, and noise level $\texttt{t}_\text{noise}$ are hyperparameters to be set by the user; in practice, good settings for such hyperparameters can be found in an offline manner through \textit{policy discrimination} (Section~\ref{sec:policy_discrimination}).











\section{Implementation Details}
\label{sec:implementation_details}


We include the default hyperparameters from the TD-MPC implementation in Table~\ref{tab:tdmpc-hparams} for completeness.  We do not modify the default recommended settings for both Humanoid and Dog environments, as well as the Meta-World experiments.








\begin{table}[h]
\centering
\small
\begin{tabular}{@{}ll@{}}
\toprule
Hyperparameter     & Value                           \\ \midrule
Training Objective       &   \texttt{pred\_noise}           \\
Number of Training Steps &  60000                            \\
Loss Type                & L2                                \\
Learning Rate            &  1e-4                              \\
Beta Schedule            &  Linear schedule (0.0085, 0.012)   \\
Timesteps                &  1000                              \\
EMA Decay                &  0.99                              \\
EMA Update Steps         &  10                                \\ 
\bottomrule
\end{tabular}


\caption[]{\textbf{Hyperparameters for In-Domain Model Training.} }
\label{table:hparams_in_domain_model_training}
\end{table}

\begin{table}[]
\centering
\begin{tabular}{@{}lcc@{}}
\toprule
Noise Level              & Humanoid Walking & Dog Walking \\ \midrule
In-Domain Only           & 600              & 600         \\
Direct Finetuning        & 700              & 700         \\
Subject Customization    & 500              & 600         \\
Prob. Adaptation         & 700              & 700         \\
Inverse Prob. Adaptation & 600              & 500         \\ \bottomrule
\end{tabular}
\caption[]{\textbf{VideoTADPoLe Noise Levels for DeepMind Control.}}
\label{table:videotadpole_noise_level_dmc}
\end{table}


\textbf{Visual Planning Hyperparameters:} To generate a video plan with adapted video models, we perform DDIM~\citep{song2021ddim} sampling for 25 steps. We use 7.5 as the text-conditioning guidance scale for directly finetuned AnimateDiff, and use 2.5 for other adaptation techniques. Additionally, we use 0.1 as the prior strength for probabilistic adaptaion and 0.5 for its inverse version.


\textbf{Inverse Dynamics:} We employ a small MLP network as our inverse dynamics model. The model takes in the embeddings of two consecutive video frames, which are extracted using VC-1~\citep{majumdar2023vc1}, and predicts the action that enables the transition between the provided frames. We train the inverse dynamics model on a dataset comprising a mixture of expert and suboptimal trajectories rendered from the environment, using the same set of tasks and data volumn as used for adaptation. For fairness, we reuse the same dynamics model across all adaptation techniques during evaluation. We provide the detailed hyperparameters of inverse dynamics training in Table~\ref{table:inv_dyn_hparams}.


\begin{table}[h]
\centering
\small

\begin{tabular}{@{}ll@{}}
\toprule
Hyperparameter    & Value \\ \midrule
Input Dimension  & 1536      \\
Output Dimension & 4      \\
Training Epochs  & 20      \\
Learning Rate    & 3e-5      \\
Optimizer        & AdamW     \\ \bottomrule
\end{tabular}

\caption{\textbf{Hyperparamters of Inverse Dynamics Model Training}}
\label{table:inv_dyn_hparams}
\end{table}


\begin{table}[h!]
\centering
\begin{minipage}{0.48\textwidth}
\label{tab:sd-hparams}
\vspace{0.05in}
\centering
\begin{tabular}{@{}ll@{}}
\toprule
Component   & \# Parameters (Millions) \\
\midrule
VAE (Encoder) & 34.16 \\
VAE (Decoder) & 49.49 \\
U-Net & 865.91 \\
Text Encoder & 340.39 \\
\bottomrule
\end{tabular}%

\caption{\textbf{StableDiffusion Components.} For completeness, we list sizes of the components of the StableDiffusion v2.1 checkpoint used in Video-TADPoLe experiments. The checkpoint is used purely for inference, and is not modified or updated in any way. Note that the VAE Decoder is not utilized in our framework.}

\hfill

\vspace{0.45in}


\label{tab:ad-hparams}
\vspace{0.05in}
\centering
\begin{tabular}{@{}ll@{}}
\toprule
Component   & \# Parameters (Millions) \\
\midrule
VAE (Encoder) & 34.16 \\
VAE (Decoder) & 49.49 \\
U-Net & 1312.73 \\%865.91 \\
Text Encoder & 123.06 \\
\bottomrule
\end{tabular}%
\caption{\textbf{AnimateDiff Components.} For completeness, we list sizes of the components of the AnimateDiff checkpoint used in Video-TADPoLe experiments.  The checkpoint is used purely for inference, and is not modified or updated in any way. Note that the VAE Decoder is not utilized in our framework.}
\end{minipage}\hfill
\begin{minipage}{0.48\textwidth}
\vspace{0.05in}
\centering
\resizebox{\linewidth}{!}{%
\begin{tabular}{@{}ll@{}}
\toprule
Hyperparameter   & Value \\
\midrule
Discount factor ($\gamma$) & 0.99 \\
Seed steps & $5,000$ \\
Replay buffer size & Unlimited \\
Sampling technique & PER ($\alpha=0.6, \beta=0.4$) \\
Planning horizon ($H$) & $5$ \\
Initial parameters ($\mu^{0}, \sigma^{0}$) & $(0, 2)$ \\
Population size & $512$ \\
Elite fraction & $64$ \\
Iterations & 12 (Humanoid)\\
 & 8 (Dog)\\
Policy fraction & $5\%$ \\
Number of particles & $1$ \\
Momentum coefficient & $0.1$ \\
Temperature ($\tau$) & $0.5$ \\
MLP hidden size & $512$ \\
MLP activation & ELU \\
Latent dimension & 100 (Humanoid, Dog) \\
Learning rate & 3e-4 (Dog)\\
 & 1e-3 (Humanoid) \\
Optimizer ($\theta$) & Adam ($\beta_1=0.9, \beta_2=0.999$) \\
Temporal coefficient ($\lambda$) & $0.5$ \\
Reward loss coefficient ($c_{1}$) & $0.5$ \\
Value loss coefficient ($c_{2}$) & $0.1$ \\
Consistency loss coefficient ($c_{3}$) & $2$ \\
Exploration schedule ($\epsilon$) & $0.5\rightarrow 0.05$ (25k steps) \\
Planning horizon schedule & $1\rightarrow 5$ (25k steps) \\
Batch size & 2048 (Dog) \\
 & 512 (Humanoid) \\
Momentum coefficient ($\zeta$) & $0.99$ \\
Steps per gradient update & $1$ \\
$\theta^{-}$ update frequency & 2 \\
\bottomrule
\end{tabular}%
}
\caption{\textbf{TD-MPC hyperparameters.} We use the official implementation TD-MPC~\citep{hansen2022temporal} with no adjustments to the hyperparameters, but list it below for completeness. We set the number of training steps to 2 million for continuous control experiments using TD-MPC, and 700k steps for MetaWorld experiments.}
\label{tab:tdmpc-hparams}
\end{minipage}
\end{table}


\section{Policy Discrimination}
\label{sec:policy_discrimination}
Rather than performing an expensive sweep over Video-TADPoLe hyperparameters directly by launching policy supervision experiments across each adapted video model technique, which can be expensive, we look for an offline method to determine reasonable hyperparameter settings.  For each environment, we therefore utilize an example expert quality demonstration video as well as an example poor quality demonstration video (with arbitrary quality levels in-between, if available).  Then, we can perform a search over Video-TADPoLe parameters by computing Video-TADPoLe rewards for these trajectories using an adapted video model, conditioned on the task-relevant text prompt, with respect to different context window, stride, and noise level settings.  We seek parameter settings that, through the adapted video model's Video-TADPoLe reward computation, can correctly distinguish between the expert, text-aligned video demonstration from the poor, text-unaligned video demonstration; this can be done by comparing the predicted Video-TADPoLe rewards.  Once identified in this offline manner, we can subsequently use the discovered settings of context window, stride, and noise level for learning text-conditioned policies.  In practice, we have found that these settings can be reused for novel text-conditioning within the same environment without issue.


\section{Policy Supervision with Additional Pretrained Video Models}

\begin{table*}[h]
\centering
\small

\setlength{\tabcolsep}{4.8pt}
 \scalebox{0.9}{
\begin{tabular}{lccccc}\toprule
Success Rate (\%) w/ & Door Close$^*$ & Door Open & Window Close & Window Open & Drawer Close  \\
\midrule
In-Domain-Only & 100.0 $\pm$ 0.0  & 31.1 $\pm$ 44.0  & 0.0 $\pm$ 0.0  & 33.3 $\pm$ 47.1  & 74.4 $\pm$ 36.2    \\
Vanilla AnimateLCM & 100.0 $\pm$ 0.0  & 0.0 $\pm$ 0.0  & 98.9 $\pm$ 1.9  & 33.3 $\pm$ 29.1   & 100.0 $\pm$ 0.0    \\
\midrule
Prob. Adaptation & 100.0 $\pm$ 0.0  & 0.0 $\pm$ 0.0 & 66.7 $\pm$ 57.7  & 0.0 $\pm$ 0.0 & 100.0 $\pm$ 0.0   \\
Inverse Prob. Adaptation & 100.0 $\pm$ 0.0 & 100.0 $\pm$ 0.0  & 100.0 $\pm$ 0.0  & 94.4 $\pm$ 9.6  & 100.0 $\pm$ 0.0    \\
\midrule
\midrule
Success Rate (\%) w/ & Drawer Open & Coffee Push$^*$ & Soccer & Button Press &  \textbf{Overall}  \\
\midrule
In-Domain-Only &  0.0 $\pm$ 0.0 & 0.0 $\pm$ 0.0 & 0.0 $\pm$ 0.0 & 33.3 $\pm$ 47.1  & 30.2    \\
Vanilla AnimateLCM & 0.0 $\pm$ 0.0 & 5.6 $\pm$ 9.6  & 0.0 $\pm$ 0.0  & 0.0 $\pm$ 0.0  &  37.5    \\
\midrule
Prob. Adaptation & 0.0 $\pm$ 0.0 & 32.2 $\pm$ 28.0  & 4.4 $\pm$ 7.7  & 0.0 $\pm$ 0.0  &  33.7     \\
Inverse Prob. Adaptation & 16.7 $\pm$ 29.0 & 41.1 $\pm$ 15.0 & 4.4 $\pm$ 5.1 & 30.0 $\pm$ 52.0  &  \textbf{65.2}   \\
\bottomrule
\end{tabular}
}
\vspace{-5pt}
\caption[]{\textbf{Policy Learning on MetaWorld with AnimateLCM.} We report the mean success rate across 9 manipulation tasks in MetaWorld, aggregated over 3 seeds.}
\label{table:mw_policysupervision_animatelcm}
\vspace{-5pt}
\end{table*}

We also provide policy supervision results on MetaWorld with AnimateLCM~\citep{wang2024animatelcm} in Table~\ref{table:mw_policysupervision_animatelcm}. Similar to AnimateDiff, vanilla AnimateLCM is also able to achieve decent success rates through Video-TADPoLe. Furthermore, we discover that inverse probabilistic adaptation consistently achieves the best performance with both AnimateDiff and AnimateLCM. With AnimateLCM, inverse probabilistic adaptation obtains the highest overall success rate of $\textbf{65.2\%}$, surpassing all other evaluated video models and adaptation techniques, with non-zero success rates across all evaluated tasks.


\section{Visual Planning with Additional Pretrained Video Models}

\begin{table*}[h]
\centering
\small

\setlength{\tabcolsep}{4.8pt}
 \scalebox{0.9}{
\begin{tabular}{lccccc}\toprule
Success Rate (\%) w/ & Door Close$^*$ & Door Open & Window Close & Window Open & Drawer Close  \\
\midrule
In-Domain-Only & 93.3 $\pm$ 14.9 & 0.0 $\pm$ 0.0 & 53.3 $\pm$ 29.8  & 6.7 $\pm$ 14.9 & 20.0 $\pm$ 29.8 \\
Vanilla AnimateLCM & 100.0 $\pm$ 0.0  & 0.0 $\pm$ 0.0  & 0.0 $\pm$ 0.0  & 20.0 $\pm$ 18.3   & 40.0 $\pm$ 27.9    \\
\midrule
Prob. Adaptation & 100.0 $\pm$ 0.0  & 0.0 $\pm$ 0.0 & 53.3 $\pm$ 38.0  & 0.0 $\pm$ 0.0 & 53.3 $\pm$ 29.8   \\
Inverse Prob. Adaptation & 100.0 $\pm$ 0.0 & 0.0 $\pm$ 0.0  & 40.0 $\pm$ 14.9  & 0.0 $\pm$ 0.0  & 93.3 $\pm$ 14.9    \\
\midrule
\midrule
Success Rate (\%) w/ & Drawer Open & Coffee Push$^*$ & Soccer & Button Press &  \textbf{Overall}  \\
\midrule
In-Domain-Only &  0.0 $\pm$ 0.0  & 0.0 $\pm$ 0.0 & 0.0 $\pm$ 0.0 & 40.0 $\pm$ 14.9 &   23.7  \\
Vanilla AnimateLCM & 0.0 $\pm$ 0.0 & 0.0 $\pm$ 0.0  & 0.0 $\pm$ 0.0  & 0.0 $\pm$ 0.0  &  17.8    \\
\midrule
Prob. Adaptation & 0.0 $\pm$ 0.0 & 0.0 $\pm$ 0.0  & 0.0 $\pm$ 0.0  & 6.7 $\pm$ 14.9  &  23.7     \\
Inverse Prob. Adaptation & 0.0 $\pm$ 0.0 & 0.0 $\pm$ 0.0 & 6.7 $\pm$ 14.9 & 26.7 $\pm$ 27.9  &  \textbf{29.6}   \\
\bottomrule
\end{tabular}
}
\vspace{-5pt}
\caption[]{\textbf{Visual Planning on MetaWorld with AnimateLCM.} We report the mean success rate across 9 manipulation tasks in MetaWorld. Each table entry shows the average success rate aggregated from $5$ seeds. }
\label{table:mw_visualplanning_animatelcm}
\vspace{-5pt}
\end{table*}

We provide visual planning results on MetaWorld with an additional video diffusion model, AnimateLCM~\citep{wang2024animatelcm}, in Table~\ref{table:mw_visualplanning_animatelcm}. We observe both probabilistic adaptation and its inverse version bring improvements in overall success rate compared to Vanilla AnimateLCM. Specifically, inverse probabilistic adaptation achieves the best overall performance and outperforms the in-domain-only baseline by 24.9\%, reconfirming the efficacy of adaptation in improving in-domain task performance.  This further demonstrates that adaptation as an approach can be applied flexibly across different backbone text-to-video models for successful downstream robotic applications.

\section{Visual Planning for Object Navigation in iTHOR Environments}

\begin{table*}[h]
\centering
\small

\setlength{\tabcolsep}{4.8pt}
\resizebox{\textwidth}{!}{
\begin{tabular}{lccccc}\toprule
Success Rate (\%) w/ & Spatula in \textit{Kitchen}$^*$ & Toaster in \textit{Kitchen}$^*$ & Painting in \textit{Living Room}$^*$ & Blinds in \textit{Bedroom}$^*$ & ToiletPaper in \textit{Bathroom}$^*$ \\
\midrule
In-Domain-Only & 13.3 $\pm$ 29.8 & 33.3 $\pm$ 33.3 & 0.0 $\pm$ 0.0  & 13.3 $\pm$ 29.8 & 40.0 $\pm$ 36.5 \\
\midrule
Prob. Adaptation & 20.0 $\pm$ 29.8  & 60.0 $\pm$ 27.9 & 0.0 $\pm$ 0.0  & 26.7 $\pm$ 36.5 & 73.3 $\pm$ 14.9   \\
Inverse Prob. Adaptation & 13.3 $\pm$ 18.3 & 33.3 $\pm$ 23.6  & 0.0 $\pm$ 0.0  & 33.3 $\pm$ 33.3  & 40.0 $\pm$ 14.9    \\
\midrule
\midrule
Success Rate (\%) w/ & Pillow in \textit{Living Room} & DeskLamp in \textit{Living Room} & Mirror in \textit{Bedroom} &  Laptop in \textit{Bedroom} &  \textbf{Overall}  \\
\midrule
In-Domain-Only &  6.7 $\pm$ 14.9  & 6.7 $\pm$ 14.9 & 0.0 $\pm$ 0.0 & 26.7 $\pm$ 27.9 &   15.6  \\
\midrule
Prob. Adaptation & 13.3 $\pm$ 18.3 & 13.3 $\pm$ 18.3  & 0.0 $\pm$ 0.0  & 53.3 $\pm$ 29.8  &  \textbf{28.9}     \\
Inverse Prob. Adaptation & 6.7 $\pm$ 14.9 & 13.3 $\pm$ 18.3 & 6.7 $\pm$ 14.9 & 60.0 $\pm$ 27.9  &  23.0   \\
\bottomrule
\end{tabular}
}
\vspace{-5pt}
\caption[]{\textbf{Visual Planning on iTHOR.} We report the mean success rate across 9 object navigation tasks in iTHOR. Each table entry shows the average success rate aggregated from $5$ seeds. ``$*$'' denotes seen tasks during adaptation.} %
\label{table:mw_videoplanning_ithor}
\vspace{-5pt}
\end{table*}

We provide additional experimentation of adaptation techniques on iTHOR~\citep{eric2017ai2thor}, in which a mobile robotic agent is asked to perform egocentric navigation to a specified target object in different scenes. This benchmark poses challenges of navigating in partially observable settings and allows us to further evaluate adaptation methods on in-domain video generation from egocentric views. To perform adaptation, we reuse the video dataset provided by AVDC~\citep{ko2024avdc}, which spans 12 target objects and includes 25 successful navigation trajectories for each object. We report the success rates of visual planning across 9 navigation tasks in Table~\ref{table:mw_videoplanning_ithor}, in which 4 tasks are unseen during adaptation. We provide a detailed list of iTHOR tasks along with their corresponding text prompts in Table~\ref{table:text_prompts_ithor}. In Table~\ref{table:mw_videoplanning_ithor}, we again observe that the overall performance of both probabilistic adaptation and its inverse outperform that of in-domain-only baseline by a large margin, highlighting that the internet knowledge of pretrained video models can be effectively utilized for various downstream robotic applications through proper adaptation.  This result further highlights how adaptation can be flexibly applied across varied robotic settings.



\section{Step Counts to Task Success in Close-loop Visual Planning}


\begin{table*}[h]
\centering
\small

\setlength{\tabcolsep}{4.8pt}
 \scalebox{0.9}{
\begin{tabular}{lccccc}\toprule
Step Count  w/ & Door Close$^*$ & Door Open & Window Close & Window Open & Drawer Close  \\
\midrule
In-Domain-Only & 80.0 & - & 176.0  & 344.0 & 25.3 \\
Vanilla AnimateDiff & 122.3  & -  & 323.0  & 217.3   & 160.9    \\
\midrule
Direct Finetuning & 96.0 & - & 159.2 & 333.3 & 63.8 \\
Subject Customization  & 150.5 & - & - & 297.3 & 238.7 \\
Prob. Adaptation & 75.4   & - & 171.5  & 312.0 & 31.0   \\
Inverse Prob. Adaptation & 87.0 & -  & 222.6  & -  & 35.8    \\
\midrule
\midrule
Step Count  w/ & Drawer Open & Coffee Push$^*$ & Soccer & Button Press &    \\
\midrule
In-Domain-Only &  -  & - & - & 204.0 &     \\
Vanilla AnimateDiff & - & -  & -  & -  &     \\
\midrule
Direct Finetuning & - & - & - & - & \\
Subject Customization & - & 155.0 & - & - & \\
Prob. Adaptation & 244.0 & 52.0  & -  & 183.2  &      \\
Inverse Prob. Adaptation & - & - & 144.0 & 192.0 &     \\
\bottomrule
\end{tabular}
}
\vspace{-5pt}
\caption[]{\textbf{Step Counts of Visual Planning on MetaWorld.} We report the average number of taken steps in successful evaluation rollouts across 9 manipulation tasks in MetaWorld. Unsuccessful rollouts are omitted. We observed that probabilistic adaptation in general achieves task success using fewer number of steps.} %
\label{table:mw_visualplanning_stepcount}
\vspace{-5pt}
\end{table*}







\end{document}