\section{Experiments}
\subsection{Dataset}
\paragraph{CALVIN.}
\emph{CALVIN}~\cite{mees2022calvin} is a simulated benchmark tailored for evaluating long-horizon, language-conditioned robotic manipulation. It comprises four simulated environments (A, B, C, and D), each containing demonstration trajectories collected via human teleoperation. The benchmark encompasses 34 distinct manipulation tasks with a total of 1,000 unique language instructions. Performance is measured by the average number of successfully completed sub-tasks within a sequence. Standard evaluation protocols include the \emph{ABC$\rightarrow$D} and \emph{ABCD$\rightarrow$D} settings, which test a model’s ability to generalize to unseen environments and compositions of long-horizon tasks.
\vspace{-2mm}
\paragraph{LIBERO.} 
The \emph{LIBERO} benchmark~\cite{liu2023libero} is a comprehensive suite for lifelong robotic manipulation, comprising four task suites with 10 tasks and 50 human demonstrations each. These suites are designed to evaluate different generalization abilities: \emph{LIBERO-Spatial} tests spatial reasoning by varying layouts with fixed objects; \emph{LIBERO-Object} assesses object-level generalization with varying objects in a fixed scene; \emph{LIBERO-Goal} targets goal-conditioned behavior by varying task goals; and \emph{LIBERO-Long} (\emph{LIBERO-10}) features long-horizon, compositional tasks with diverse objects, layouts, and goals, challenging temporal and compositional reasoning.
\vspace{-2mm}
\paragraph{SimplerEnv.} 
SimplerEnv~\cite{li2024evaluating} serves as a simulation benchmark designed to evaluate the transferability and generalization capabilities of models trained on real-world video data. It incorporates diverse manipulation setups across both the WidowX and Google Robot platforms, encompassing variations in lighting conditions, object textures, color distributions, and camera viewpoints.

% ----------------------------------------------------------------------------------------------
\subsection{Implementation Details}
The model adopts a purely autoregressive Transformer architecture with 8.5 billion parameters, identical to Emu3~\cite{wang2024emu3}. Images are tokenized using a VQ-based image encoder with a spatial compression factor of 8. For action encoding, we use the relative differences between consecutive frames. We first apply 1st and 99th percentile normalization, and then utilize the FAST tokenizer~\cite{pertsch2025fast}, which has a vocabulary size of 1024 and replaces the final 1024 token IDs of the language tokenizer.

\vspace{-2mm}
\paragraph{Post-training Stage.}
In the post-training stage, we leverage large-scale robot-centric video datasets to study the effects of various post-training strategies on downstream policy learning. The model is initialized with pre-trained weights from the first stage of Emu3~\cite{wang2024emu3}. We curate a total of 622K videos from existing robotics datasets (details provided in the appendix), and identify the world model as the most effective post-training approach. During training, supervision is applied solely on the vision tokens. The model is trained for 30K steps with a batch size of 64.

\vspace{-2mm}
\paragraph{Fine-tuning Stage.}
During fine-tuning, the model is initialized with weights from the post-training stage and trained using a two-frame interleaved vision-action sequence with an action chunk size of 10. A cosine annealing learning rate schedule is applied, starting at \(8 \times 10^{-5}\), and the loss is computed solely over action tokens.
For the CALVIN benchmark, RGB observations from both third-person (\(200 \times 200\)) and wrist-view (\(80 \times 80\)) cameras are used. Training is conducted on A100 GPUs with a batch size of 192 for 8k steps.
For the LIBERO benchmark, third-person and wrist-view RGB images (both at \(200 \times 200\)) are used to train a unified model with a batch size of 192 for 8k steps. A single model is evaluated across four task suites.
For the SimplerEnv benchmark, single-view RGB observations are used with input resized to \(256 \times 256\). Training is conducted on the Bridge-WidowX setup using a batch size of 128 for 20k steps, with an action chunk size of 5.

Additional implementation details on the post-training strategy, real-robot fine-tuning procedures, and autonomous driving experiments are provided in the appendix.
% ----------------------------------------------------------------------------------------------
\begin{table}[ht]
    \centering
    \caption{\textbf{Long-horizon robotic manipulation evaluation on the CALVIN benchmark.}}
      \resizebox{0.86\textwidth}{!}{
    \begin{tabular}{l l c c c c c c}
        \toprule
        \textbf{Method}  & \textbf{Task} & \multicolumn{5}{c}{\textbf{Tasks Completed in a Row}} & \textbf{Avg. Len $\uparrow$} \\
        \cmidrule(lr){3-7}
        &  & 1 & 2 & 3 & 4 & 5 & \\
        \midrule
        MCIL~\cite{lynch2020language} & ABCD$\rightarrow$D &0.373 & 0.027 & 0.002 & 0.000 & 0.000 & 0.40\\
        RT-1~\cite{brohan2022rt}  & ABCD$\rightarrow$D & 0.844 & 0.617 & 0.438 & 0.323 & 0.227 & 2.45 \\
        Robo-Flamingo~\cite{li2024vision} & ABCD$\rightarrow$D & 0.964 & 0.896 & 0.824 & 0.740 & 0.660 & 4.09 \\
        GR-1~\cite{wu2023unleashing}  & ABCD$\rightarrow$D & 0.949 & 0.896 & 0.844 & 0.789 & 0.731 & 4.21 \\
        UP-VLA~\cite{zhang2025up}  & ABCD$\rightarrow$D & 0.962 & 0.921 & 0.879 & 0.842 & 0.812 & 4.42 \\
        RoboVLMs~\cite{li2024towards} & ABCD$\rightarrow$D & 0.967 & 0.930 & 0.899 & 0.865 & 0.826 & 4.49 \\
        \rowcolor[gray]{0.9} \textbf{\methodname{}} & ABCD$\rightarrow$D & \textbf{0.985} & \textbf{0.961} & \textbf{0.931} & \textbf{0.899} & \textbf{0.851} & \textbf{4.63}\\
        \midrule
        MCIL~\cite{lynch2020language} & ABC$\rightarrow$D &0.304	&0.013&	0.002	&0.000	&0.000&	0.31\\
        Robo-Flamingo~\cite{li2024vision} & ABC$\rightarrow$D & 0.824 & 0.619 & 0.466 & 0.331 & 0.235 & 2.47 \\
        SuSIE~\cite{black2023zero}& ABC$\rightarrow$D& 0.870 &0.690 &0.490 &0.380 &0.260 &2.69\\
        GR-1~\cite{wu2023unleashing}  & ABC$\rightarrow$D & 0.854 & 0.712 &0.596	&0.497	&0.401	&3.06 \\
        UP-VLA~\cite{zhang2025up}  & ABC$\rightarrow$D & 0.928 &0.865& 0.815& 0.769& 0.699& 4.08 \\
        RoboVLMs~\cite{li2024towards} & ABC$\rightarrow$D &
        0.980 &0.936 &0.854& 0.778 &0.704 &4.25 \\
        Seer-Large~\cite{tian2024predictive} & ABC$\rightarrow$D &0.963 &0.916 &0.861 &0.803 &0.740 & 4.28\\
         \rowcolor[gray]{0.9} \textbf{\methodname{}}  & ABC$\rightarrow$D & \textbf{0.989} & \textbf{0.948} & \textbf{0.890} & \textbf{0.828} & \textbf{0.751} & \textbf{4.41}\\
        \bottomrule
    \end{tabular}}
        \label{tab:calvin_results}
\end{table}


\subsection{Main Results}
In this section, we evaluate our method on three simulation benchmarks: CALVIN (long-horizon tasks), LIBERO (diverse generalization), and SimplerEnv (real-to-sim manipulation). Our approach consistently achieves state-of-the-art performance across all settings.
\vspace{-2mm}
\paragraph{CALVIN Simulation Evaluation.}
Table~\ref{tab:calvin_results} presents the experimental results in the CALVIN benchmark. Our method achieves the highest performance on both the ABC$\rightarrow$D and ABCD$\rightarrow$D tasks, significantly outperforming previous approaches and demonstrating strong capabilities in multi-task learning and long-horizon planning.
\vspace{-2mm}
\paragraph{LIBERO Simulation Evaluation.}
Following~\cite{zhao2025cot}, we report the average success rate over 500 episodes for each task suite (Spatial, Object, Goal, Long). As shown in Table~\ref{tab:libero}, \methodname{} achieves the best overall performance across all LIBERO benchmark suites, with particularly significant gains on long-horizon tasks—improving the previous state of the art from 69.0\% to 94.0\%. Compared to $\pi_0$~\cite{pertsch2025fast}, our method demonstrates superior performance on long-horizon tasks.
\begin{table}[ht]
\centering
\caption{\textbf{Comparison of different methods on the LIBERO benchmark.}}
\resizebox{0.68\textwidth}{!}{
\begin{tabular}{lccccc}
\toprule
\textbf{Method} & \textbf{SPATIAL} & \textbf{OBJECT} & \textbf{GOAL} & \textbf{LONG} & \textbf{Average} \\
\midrule
DP*~\cite{chi2023diffusion}         & 78.3\% & 92.5\% & 68.3\% & 50.5\% & 72.4\% \\
Octo~\cite{team2024octo}      & 78.9\% & 85.7\% & 84.6\% & 51.1\% & 75.1\% \\
OpenVLA~\cite{kim2024openvla}& 84.9\% & 88.4\% & 79.2\% & 53.7\% & 76.5\% \\
SpatialVLA~\cite{qu2025spatialvla} & 88.2\% & 89.9\% & 78.6\% & 55.5\% & 78.1\% \\
CoT-VLA~\cite{zhao2025cot} & 87.5\% & 91.6\% & 87.6\% & 69.0\% & 81.1\%\\
$\pi_0$-FAST~\cite{pertsch2025fast} &\textbf{96.4\%}	& 96.8\%&	88.6\%&	60.2\%&	85.5\% \\
% \rowcolor[gray]{0.9} \textbf{\methodname{}} &\textbf{96.6\%} & \textbf{99.4\%} & \textbf{91.6\%} & \textbf{89.2\%} & \textbf{94.2\%}\\
\rowcolor[gray]{0.9} \textbf{\methodname{}} &95.4\% & \textbf{98.8\%} & \textbf{93.6\%} & \textbf{94.0\%} & \textbf{95.5\%}\\
\bottomrule
\end{tabular}}
\label{tab:libero}
\vspace{-3mm}
\end{table}

\vspace{-2mm}
\paragraph{SimplerEnv Simulation Evaluation.}
Table~\ref{tab:simplerenv_bridge} summarizes the performance across various manipulation policies on the Bridge-WidowX setup. Our approach demonstrates a significant improvement over prior methods, raising the average success rate from 42.7\% to 69.8\%. In particular, it shows marked improvements on previously difficult tasks, including stack block, put carrot and put spoon.
\begin{table}[htbp]
\centering
\caption{\textbf{Evaluation on SimplerEnv-WidowX across various manipulation tasks.}}
\label{tab:simplerenv_bridge}
\resizebox{\textwidth}{!}{
\begin{tabular}{l|cc|cc|cc|cc|c}
\toprule
\multirow{2}{*}{\textbf{Model}} 
& \multicolumn{2}{c|}{\textbf{Put Spoon on Towel}} 
& \multicolumn{2}{c|}{\textbf{Put Carrot on Plate}} 
& \multicolumn{2}{c|}{\textbf{Stack Green on Yellow Block}} 
& \multicolumn{2}{c|}{\textbf{Put Eggplant in Yellow Basket}} 
& \textbf{Overall} \\
\cmidrule(lr){2-3} \cmidrule(lr){4-5} \cmidrule(lr){6-7} \cmidrule(lr){8-9}
& \multicolumn{1}{c}{Grasp} & \multicolumn{1}{c|}{Success} 
& \multicolumn{1}{c}{Grasp} & \multicolumn{1}{c|}{Success} 
& \multicolumn{1}{c}{Grasp} & \multicolumn{1}{c|}{Success} 
& \multicolumn{1}{c}{Grasp} & \multicolumn{1}{c|}{Success} 
& \multicolumn{1}{c}{Success} \\
\midrule
RT-1-X~\cite{brohan2023rt}     
& 16.7\% & 0.0\%   
& 20.8\% & 4.2\%   
& 8.3\%  & 0.0\%   
& 0.0\%  & 0.0\%   
& 1.1\% \\
Octo-Base~\cite{octo_2023} 
& 34.7\% & 12.5\%  
& 52.8\% & 8.3\%   
& 31.9\% & 0.0\%   
& 66.7\% & 43.1\%  
& 16.0\% \\
Octo-Small~\cite{octo_2023}
& 77.8\% & 47.2\%  
& 27.8\% & 9.7\%   
& 40.3\% & 4.2\%   
& 87.5\% & 56.9\%  
& 29.5\% \\
OpenVLA~\cite{kim2024openvla}  
& 4.1\%  & 0.0\%   
& 33.3\% & 0.0\%   
& 12.5\% & 0.0\%   
& 8.3\%  & 4.1\%   
& 1.0\% \\
RoboVLMs~\cite{li2024towards}
& 70.8\% & 45.8\%  
& 33.3\% & 20.8\%  
& 54.2\% & 4.2\%  
& 91.7\% & 79.2%  
& 37.5\% \\
SpatialVLA~\cite{qu2025spatialvla} 
& 20.8\% & 16.7\% 
& 29.2\% & 25.0\% 
& 62.5\% & 29.2\% 
& 100\% & \textbf{100\%} 
& 42.7\% \\
% \rowcolor[gray]{0.9} \textbf{\methodname{}} 
% & \textbf{91.7\%} & \textbf{70.8\%} 
% & \textbf{66.7\%} & \textbf{58.3\%} 
% & \textbf{79.2\%} & \textbf{33.3\%} 
% & 95.8\% & 95.8\%  
% & \textbf{64.6\%} \\
\rowcolor[gray]{0.9} \textbf{\methodname{}} 
& \textbf{83.3\%} & \textbf{83.3\%} 
& \textbf{74.0\%} & \textbf{66.7\%} 
& \textbf{95.8\%} & \textbf{33.3\%} 
& \textbf{100.0\%} & 95.8\%  
& \textbf{69.8\%} \\
\bottomrule
\end{tabular}}
\vspace{-3mm}
\end{table}

% \input{tables/simplerenv_google}
% -----------------------------------------------------------------------
\subsection{In-Depth Analysis}
In this section, we provide an in-depth analysis within our unified framework, which may offer key insights for the design of future VLA models.
We first analyze how post-training enhances downstream policy learning in terms of both performance (Table~\ref{abl:post_train}) and training efficiency (Table~\ref{tab:post_efficiency}), highlighting the potential of world models as a general post-training strategy for robotics.
We then investigate that even without post-training stage, incorporating visual prediction loss (Table~\ref{tab:calvin_abl_visual}) and historical context (Table~\ref{tab:calvin_abl_history}) still contributes positively to policy learning. 
\vspace{-2mm}
\paragraph{Effectiveness of World Model Post-Training.}
Table~\ref{abl:post_train} investigates the effects of different post-training strategies on downstream policy learning across various simulation benchmarks. The results reveal that, due to inconsistencies in the action space across tasks, action-only learning exhibits low transferability, leading to a negative impact on performance. In contrast, most post-training approaches significantly enhance policy learning, highlighting the crucial role of visual learning in transferability.
Among these, the world model post-training approach yields the most substantial gains, enhancing both generalization and long-horizon planning capabilities.
A comparison with text-to-image (T2I) training emphasizes the importance of modeling temporal dynamics in video data, while contrasting with video-only training highlights the essential role of textual guidance in state transitions. Notably, this world model training requires no action annotations, enabling scalable learning from large-scale video data and providing a promising direction for future VLA research.
\definecolor{DeepGreen}{rgb}{0.0, 0.8, 0.0}

\begin{table}[tbp]
\centering
\caption{\textbf{Effectiveness of World Model Post-Training.} We compare different post-training strategies by fine-tuning only with action prediction on the downstream benchmarks.}
\resizebox{\linewidth}{!}{
\begin{tabular}{ccc|cc|cc}
\toprule
\multicolumn{3}{c|}{\textbf{Post-training Stage}} & \multicolumn{2}{c|}{\textbf{Generalization}} & \multicolumn{2}{c}{\textbf{Long-horizon}} \\
\textbf{Strategy} & \textbf{Sequence} & \textbf{Supervision} & \textbf{LIBERO} & \textbf{SimplerEnv-WidowX} & \textbf{LIBERO-Long} & \textbf{CALVIN} \\
\midrule
& & &  48.5 & 0.0 & 17.4 & 1.46 \\
action prediction & $T, I, A$ & action &43.9 (\textcolor{red}{-4.6}) & 0.0 & 10.6 (\textcolor{red}{-6.8})& 0.52(\textcolor{red}{-0.94})\\
text-to-image    & $T,I$ & vision & 69.8 (\textcolor{DeepGreen}{+21.3}) & 6.3 (\textcolor{DeepGreen}{+6.3}) & 55.8 (\textcolor{DeepGreen}{+38.4}) & 3.79 (\textcolor{DeepGreen}{+2.33}) \\
video prediction & $I_1,...,I_t$ & vision & 78.9 (\textcolor{DeepGreen}{+30.4})& 17.7 (\textcolor{DeepGreen}{+17.7}) & 80.8 (\textcolor{DeepGreen}{+63.4}) & 3.59 (\textcolor{DeepGreen}{+2.13}) \\
\rowcolor[gray]{0.9} world model      & $T,I_1,...,I_t$ & vision & \textbf{94.2} (\textcolor{DeepGreen}{+45.7}) & \textbf{64.6} (\textcolor{DeepGreen}{+64.6}) & \textbf{89.2} (\textcolor{DeepGreen}{+71.8}) & \textbf{4.61} (\textcolor{DeepGreen}{+3.15}) \\
\bottomrule
\end{tabular}}
\label{abl:post_train}
\vspace{-5mm}
\end{table}

\vspace{-2mm}
\paragraph{Data and Training Efficiency.}
Table~\ref{tab:post_efficiency} shows that post-training substantially enhances downstream policy learning efficiency. On the CALVIN benchmark (Table~\ref{tab:data_efficiency}), our method achieves higher success rates using only 10\% of the fine-tuning data, outperforming prior approaches such as GR-1~\cite{wu2023unleashing} and RoboVLMs~\cite{li2024towards}. In addition, Table~\ref{tab:train_efficiency} highlights improved training efficiency, as the model rapidly converges with fewer fine-tuning iterations. The Simpler-Env results further demonstrate the effectiveness of world-model-based post-training for efficient policy adaptation across diverse robotic setups.
While similar effects are observed in latent-action methods~\cite{ye2024latent,chen2024moto, gao2025adaworld}, our world model offers a simpler paradigm without latent actions, achieving better transferability.
\begin{table*}[htbp]
\centering
\caption{\textbf{Post-training enables data-efficient and training-efficient downstream policy learning.}}
\label{tab:post_efficiency}
\begin{subtable}[t]{0.48\textwidth}
\centering
\caption{\textbf{Data efficiency comparison.}}
\resizebox{\linewidth}{!}{%
\begin{tabular}{lcc}
\toprule
\textbf{Method} & \textbf{Data} & \textbf{CALVIN} \\
\midrule
RT-1~\cite{brohan2022rt} & 10\% & 0.34\\
MT-R3M~\cite{nair2022r3m} & 10\% & 0.61\\
HULC~\cite{mees2022matters}  & 10\% & 1.11\\
GR-1~\cite{wu2023unleashing} & 10\% & 2.00\\
RoboVLMS~\cite{li2024towards} & 10\% & 2.52\\
\midrule
\textbf{\methodname{}} (w/o post-train) & 10\% & 0.15 \\
\rowcolor[gray]{0.9}\textbf{\methodname{}} & 10\% & \textbf{3.19} \\
\bottomrule
\end{tabular}
}
\label{tab:data_efficiency}
\end{subtable}
\hfill
\begin{subtable}[t]{0.48\textwidth}
\centering
\caption{\textbf{Training efficiency comparison.}}
\resizebox{\linewidth}{!}{
\begin{tabular}{lccc}
\toprule
\multicolumn{4}{c}{\textbf{Fast convergence (CALVIN)}} \\
Training Iters & 2k & 4k & 8k  \\
\midrule
w/o post-train & 0.37 & 0.82 & 1.46  \\
\rowcolor[gray]{0.9}w/ post-train    &4.21 & 4.56 & 4.61    \\
\midrule
\multicolumn{4}{c}{\textbf{Fast adaptation (SimplerEnv-Bridge)}} \\
Method & Batch size & Iters & Success \\
\midrule
RoboVLMs~\cite{li2024towards} & 128   &50k & 37.5 \\
\rowcolor[gray]{0.9}\textbf{\methodname{}  }    & 128 &12k& 64.6 \\
\bottomrule
\end{tabular}
}
\label{tab:train_efficiency}
\end{subtable}
\vspace{-5mm}
\end{table*}

\begin{table*}[htbp]
    \centering
    \caption{\textbf{Ablation study on the visual prediction and historical context in policy learning.}}
    \begin{subtable}[t]{0.52\linewidth}
        \centering
        \caption{\textbf{Effectiveness of visual prediction.}}
        \resizebox{\linewidth}{!}{%
        \begin{tabular}{cccc}
            \toprule
            \textbf{Post-train} & \textbf{Visual prediction} & \textbf{CALVIN} & \textbf{LIBERO} \\
            \midrule
            \checkmark & & 4.61 & 94.2\\
            \rowcolor[gray]{0.9} & \checkmark & 4.42 & 88.7 \\
             & & 1.46 & 48.5 \\
            \bottomrule
        \end{tabular}
        }
        \label{tab:calvin_abl_visual}
    \end{subtable}
    \hfill
    \begin{subtable}[t]{0.46\linewidth}
        \centering
        \caption{\textbf{Effectiveness of history context.}}
        \resizebox{\linewidth}{!}{%
        \begin{tabular}{ccc}
        \toprule
        \multicolumn{2}{c}{\textbf{Observations}} & \textbf{Avg. Len $\uparrow$} \\
        \cmidrule(lr){1-2}
        \textbf{History Window} & \textbf{Current + History} & \\
        \midrule
        0 & 1 + 0 & 4.26 \\
        \rowcolor[gray]{0.9}10 & 1 + 1 & 4.61 \\
        10 & 1 + 2 & 4.43 \\
        20 & 1 + 2 & 4.47 \\
        \bottomrule
\end{tabular}
        }
        \label{tab:calvin_abl_history}
    \end{subtable}
\end{table*}

\vspace{-2mm}
\paragraph{Effectiveness of Visual Prediction.}
While post-training proves effective, it is also crucial that the model demonstrates strong performance without relying on it. As shown in Table~\ref{tab:calvin_abl_visual}, our findings indicate that, even without post-training, fine-tuning with visual loss supervision—leveraging the autoregressive nature of the model—naturally integrates world model learning into the policy learning process. This approach leads to a significant improvement in the model's performance.
\vspace{-2mm}
\paragraph{Effectiveness of History Context.}
History context—comprising past observations and actions—provides valuable guidance for robot planning. In this section, we investigate the appropriate length of the history window during the fine-tuning stage. As shown in Table~\ref{tab:calvin_abl_history}, our ablation study on the CALVIN benchmark examines the impact of varying history window lengths. Incorporating a history window significantly improves performance (from 4.26 to 4.61). However, extending the window beyond a certain length yields diminishing returns, suggesting that recent observations carry the most predictive value, consistent with the Markov property in sequential planning.

\subsection{Multimodal Capability}
As illustrated in Figure~\ref{fig:visual_demo}, we qualitatively showcase the model’s ability to interleave multiple modalities—action, language, and vision—within a unified framework. This design enables policy learning for embodied control, spatial reasoning through language output, and future state prediction via visual output, highlighting the model’s capacity for generalizable multimodal understanding.
\begin{figure}[tbp]
    \centering
    \includegraphics[width=\textwidth]{images/visual_demo.pdf}
    \vspace{-5mm}
    \caption{\textbf{Multimodal capabilities of \methodname{}}. Top: Action outputs for executing long-horizon tasks in the LIBERO benchmark. Bottom: Visual predictions and spatial grounding demonstrating the model's spatiotemporal understanding. The red box marks the current observation; green boxes indicate predicted object detections.}
    \label{fig:visual_demo}
    \vspace{-4mm}
\end{figure}

\subsection{Broader Applications}
\vspace{-1mm}
\paragraph{End-to-end Learning for Autonomous Driving.}
To further explore the potential of our method, we perform a preliminary transfer to the autonomous driving domain by finetuning the model on the NAVSIM benchmark. Notably, our method is a pure autoregressive, token-based framework, modeling the driving task as causal sequence prediction over discretized multimodal tokens. Despite using only front-view camera inputs—without relying on BEV representations or multi-sensor fusion—our model achieves powerful performance on the NAVSIM test set. Notably, the current performance is not pretrained on driving videos but is only fine-tuned on downstream policy benchmarks. These results highlight the strong potential of our method for broader real-world applications.
\begin{table}[htbp]
  \centering
    \caption{\textbf{Broader applications of \methodname{} for end-to-end autonomous driving on the NAVSIM.} MC: Multi Camera. L: LiDAR. FC: Front Camera.
    }
  \resizebox{0.88\linewidth}{!}{
  \begin{tabular}{lccccccc>{\columncolor{gray!20}}c}
    \toprule
    \textbf{Method} & \textbf{Model} & \textbf{Input} & \textbf{NC$\uparrow$} & \textbf{DAC$\uparrow$} & \textbf{EP$\uparrow$} & \textbf{TTC$\uparrow$} & \textbf{C$\uparrow$} & \textbf{PDMS$\uparrow$} \\
    \midrule
    \textcolor{gray}{Human} & \textcolor{gray}{--} & \textcolor{gray}{--} & \textcolor{gray}{100.0} & \textcolor{gray}{100.0} & \textcolor{gray}{87.5} & \textcolor{gray}{100.0} & \textcolor{gray}{99.9} & \textcolor{gray}{94.8} \\
    \midrule
    Ego Status MLP & -- & Ego State & 93.0 & 77.3 & 62.8 & 83.6 & 100.0 & 65.6 \\
    VADv2~\cite{vadv2}          & BEV-based & MC & 97.9 & 91.7 & 77.6 & 92.9 & 100.0 & 83.0 \\
    UniAD~\cite{uniad}          & BEV-based & MC & 97.8 & 91.9 & 78.8 & 92.9 & 100.0 & 83.4 \\
    Transfuser~\cite{Transfuser}     & BEV-based & MC\&L & 97.7 & 92.8 & 79.2 & 92.8 & 100.0 & 84.0 \\
    \midrule
    \textbf{\methodname{}} & Auto-regressive & FC & 96.9 & 91.1 & 76.8 & 91.7 & 96.7 & 81.7 \\
    \bottomrule
  \end{tabular}}
  \label{tab:driving_navsim}
  \vspace{-5mm}
\end{table}


